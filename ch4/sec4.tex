\begin{section}{Inner Product Space Isomorphisms}

	This section deals with unitary transformations
	and inner product space isomorphisms. It concludes
	with a discussion of $\ltwo$ spaces.
	
%%%%%%%%%%%%%%%%%%%%%%%%%%%%%%%%%%%%%%%%%%%%%%%%%%%%%%%%%%%%
%% Defn of inner product space isomorphism
%%%%%%%%%%%%%%%%%%%%%%%%%%%%%%%%%%%%%%%%%%%%%%%%%%%%%%%%%%%%

\begin{defn}
	Let $V$ and $W$ be real or complex inner product spaces. A
	linear map $T:V \rightarrow W$ is called an inner
	product space \emph{isomorphism} from $V$ to $W$ if it is
	one-to-one, onto and preserves inner products (that is,
	$\innrprod{x}{y}=\innrprod{Tx}{Ty}$ for all $x,y \in V$).
	Here it is understood that the inner product on left is
	the one in $V$ and the one on the right is the one in $W$.
	A map $T:V \rightarrow W$ which satisfies all the properties
	of an inner product space isomorphism except for being
	onto is called an \emph{embedding} of $V$ into $W$.
\end{defn}
	
%%%%%%%%%%%%%%%%%%%%%%%%%%%%%%%%%%%%%%%%%%%%%%%%%%%%%%%%%%%%
%% Defn of unitary transformation
%%%%%%%%%%%%%%%%%%%%%%%%%%%%%%%%%%%%%%%%%%%%%%%%%%%%%%%%%%%%

\begin{defn}
	Let $V$ and $W$ be real or complex inner product spaces. A
	linear map $T:V \rightarrow W$ is called a \emph{unitary
	transformation} from $V$ to $W$ provided it preserves norms
	(that is, $\norm[]{x}=\norm[]{Tx}$ for all $x \in V$). Here
	the norm on the left is the one induced from the inner product
	in $V$ and the one on the right is the one induced from the
	inner product in $W$.
\end{defn}
	
%%%%%%%%%%%%%%%%%%%%%%%%%%%%%%%%%%%%%%%%%%%%%%%%%%%%%%%%%%%%
%% Main theorem on unitary operators
%%%%%%%%%%%%%%%%%%%%%%%%%%%%%%%%%%%%%%%%%%%%%%%%%%%%%%%%%%%%

\begin{thrm}\label{thrm:Unitary}
	Let $V$ and $W$ be real or complex inner product spaces
	and $T$ a linear map from $V$ to $W$. Then $T$ is an 
	inner product space isomorphism
	iff it is an onto unitary transformation from $V$ to $W$.
\end{thrm}

\begin{proof}
	If $T$ preserves inner products then it clearly preserves norms
	because $\norm[]{x}^2 = \innrprod{x}{x} = \innrprod{Tx}{Tx}
	= \norm[]{Tx}^2$ for all $x \in V$. Also, $T$ is onto by the very
	definition of isomorphism. So $T$ is unitary and onto. Conversely, 
	if $T$ preserves
	norms and if $V$ and $W$ are real inner product spaces then it
	is easily seen that $T$ preserves inner products because
	of the identity
		\begin{displaymath}
			\innrprod{x}{y} = \frac{1}{4} \left(\norm[]{x+y}^2
				- \norm[]{x-y}^2 \right)
		\end{displaymath}
	Now, to show that $T$ preserves inner products when $V$ and
	$W$ are complex inner product spaces, first note that
		\begin{IEEEeqnarray*}{rCl}
			\norm[]{x-iy}^2 & = & \norm[]{x}^2+i\innrprod{x}{y}
				-i\innrprod{y}{x}+\norm[]{y}^2 \\
			& = & \norm[]{x}^2+i\innrprod{x}{y}-i\conj{\innrprod{x}{y}}
				+ \norm[]{y}^2 \\
			& = & \norm[]{x}^2 + 2i \text{Im}(\innrprod{x}{y})
				+ \norm[]{y}^2 \; (\ast)
		\end{IEEEeqnarray*}
	and similarly,
		\begin{displaymath}
			\norm[]{T(x-iy)}^2 = \norm[]{Tx-iTy}^2
				= \norm[]{Tx}^2 + 2i \text{Im}(\innrprod{Tx}{Ty})
				+ \norm[]{Ty}^2 \; (\dagger)
		\end{displaymath}
	Since $T$ preserves norms, by assumption, comparing $(\ast)$ to
	$(\dagger)$, we see that Im$(\innrprod{x}{y})=$ Im$(\innrprod{Tx}{Ty})$.
	In a similar fashion,
		\begin{displaymath}
			\norm[]{x+y}^2 = \norm[]{x}^2+\innrprod{x}{y}
				+\innrprod{y}{x}+\norm[]{y}^2
				= \norm[]{x}^2+2\text{Re}(\innrprod{x}{y})
				+\norm[]{y}^2 \; (\ast \ast)
		\end{displaymath}
	and
		\begin{displaymath}
			\norm[]{T(x+y)}^2 = \norm[]{Tx+Ty}^2 = \norm[]{Tx}^2
				+ 2\text{Re}(\innrprod{Tx}{Ty})+\norm[]{Ty}^2
				\; (\ddagger)
		\end{displaymath}
	Comparing $(\ast \ast)$ to $(\ddagger)$ we see that Re$(
	\innrprod{x}{y})$=Re$(\innrprod{Tx}{Ty})$. Thus $T$ preserves
	inner products. Finally note that $T$ is onto by assumption
	and clearly one-to-one since $\norm[]{Tx}=\norm[]{x} \neq 0$
	if $x \neq 0$. So $T$ is an inner product space isomorphism
	from $V$ to $W$.
\end{proof}
	
%%%%%%%%%%%%%%%%%%%%%%%%%%%%%%%%%%%%%%%%%%%%%%%%%%%%%%%%%%%%
%% Inner product space Iso. maps basis to basis
%%%%%%%%%%%%%%%%%%%%%%%%%%%%%%%%%%%%%%%%%%%%%%%%%%%%%%%%%%%%

\begin{prop}
	If $T:V \rightarrow W$ is an inner product space isomorphism then
	it maps each basis of $V$ to a basis of $W$.
\end{prop}

\begin{proof}
	Let $\{b_k\}_{k=1}^\infty$ be a basis for $V$. Then $\innrprod
	{Tb_j}{Tb_k}=\innrprod{b_j}{b_k}=\delta_{jk}$ so $\{Tb_k\}_{k=1}
	^\infty$ is orthonormal. To see that it is a basis, given
	$x \in V$ and $\epsilon > 0$ choose $N > 0$ such that if $n > N$
		\begin{displaymath}
			\norm[]{x-\sum_{k=1}^n \innrprod{x}{b_k}b_k} < \epsilon
		\end{displaymath}
	Then, by Theorem \ref{thrm:Unitary}, since $T$ preserves inner 
	products it preserves norms, so
		\begin{IEEEeqnarray*}{rCl}
			\norm[]{Tx-\sum_{k=1}^n \innrprod{Tx}{Tb_k}Tb_k} & = &
				\norm[]{T(x-\sum_{k=1}^n \innrprod{Tx}{Tb_k}b_k)} \\
			& = & \norm[]{x-\sum_{k=1}^n \innrprod{Tx}{Tb_k}b_k} \\
			& = & \norm[]{x-\sum_{k=1}^n \innrprod{x}{b_k}b_k} \\
			& < & \epsilon
		\end{IEEEeqnarray*}
	So $\{Tb_k\}_{k=1}^\infty$ is a basis for $W$ as required.
\end{proof}

%%%%%%%%%%%%%%%%%%%%%%%%%%%%%%%%%%%%%%%%%%%%%%%%%%%%%%%%%%%%
%% First example of inner product space isomorphism
%%%%%%%%%%%%%%%%%%%%%%%%%%%%%%%%%%%%%%%%%%%%%%%%%%%%%%%%%%%%

\begin{ex}
	Let $T = \{\iscmplx \; | \; \modulus{z}=1\}$. Then given $f:T \rightarrow \cmplx$, 
	define $\tilde{f}:\real \rightarrow \cmplx$ by $\tilde{f}(t)=f(e^{it})$.
	Let $\mathcal{R}_T = \{f:T \rightarrow \cmplx \; | \; \tilde{f} \in \riemmint\}$
	and define $\innerprod{f}{g}=\innerprod{\tilde{f}}{\tilde{g}}$ for $f,g
	\in \mathcal{R}_T$. It is easily verified that this defines an inner
	product on $\mathcal{R}_T$ and that the map $f \mapsto \tilde{f}$ is
	an inner product space isomorphism from $\mathcal{R}_T$ to $\riemmint$.
\end{ex}

%%%%%%%%%%%%%%%%%%%%%%%%%%%%%%%%%%%%%%%%%%%%%%%%%%%%%%%%%%%%
%% l2 spaces
%%%%%%%%%%%%%%%%%%%%%%%%%%%%%%%%%%%%%%%%%%%%%%%%%%%%%%%%%%%%

	We conclude this section with a brief discussion of $\ltwo$ spaces.
	
\begin{defn}
	Denote the set of all complex sequences (or equivalently complex
	valued functions) $\{c_k\}_{k=-\infty}^\infty$ which satisfy
		\begin{displaymath}
			\sum_{k=-\infty}^\infty \modulus{c_k}^2 < \infty
		\end{displaymath}
	by $\ltwo$.
\end{defn}

%%%%%%%%%%%%%%%%%%%%%%%%%%%%%%%%%%%%%%%%%%%%%%%%%%%%%%%%%%%%
%% l2 is a vector space
%%%%%%%%%%%%%%%%%%%%%%%%%%%%%%%%%%%%%%%%%%%%%%%%%%%%%%%%%%%%

\begin{prop}
	$\ltwo$ is a vector space under pointwise addition and scalar
	multiplication.
\end{prop}

\begin{proof}
	Suppose $c=\{c_k\}_{k=-\infty}^\infty$ and $d=\{d_k\}_{k=-\infty}^\infty$
	are sequences in $\ltwo$ and $\iscmplx[\lambda]$. Then, using the
	identity
		\begin{displaymath}
			2\modulus{c_k\conj{d_k}} = 2\modulus{c_k d_k} \leq \modulus{c_k}^2
				+ \modulus{d_k}^2
		\end{displaymath}
	we get
		\begin{IEEEeqnarray*}{rCl}
			\modulus{\sum_{k=-\infty}^\infty \modulus{\lambda c_k+d_k}^2}
				& \leq & \modulus{\modulus{\lambda}^2 \sum_{k=-\infty}^\infty
				\modulus{c_k}^2} + \modulus{\sum_{k=-\infty}^\infty
				\modulus{d_k}^2} + \modulus{\lambda \sum_{k=-\infty}^\infty
				c_k \conj{d_k}} + \modulus{\conj{\lambda} \sum_{k=-\infty}^\infty
				\conj{c_k}d_k} \\
			& \leq & \modulus{\modulus{\lambda}^2 \sum_{k=-\infty}^\infty
				\modulus{c_k}^2} + \modulus{\sum_{k=-\infty}^\infty
				\modulus{d_k}^2} + \modulus{\lambda \sum_{k=-\infty}^\infty
				\left(\frac{\modulus{c_k}^2}{2}+\frac{\modulus{d_k}^2}{2}\right)} \\
			& & + \; \modulus{\conj{\lambda} \sum_{k=-\infty}^\infty
				\left(\frac{\modulus{c_k}^2}{2}+\frac{\modulus{d_k}^2}{2}\right)} \\
			& < & \infty
		\end{IEEEeqnarray*}
	since $\sum_{k=-\infty}^\infty \modulus{c_k}^2 < \infty$ and $\sum_{k=-\infty}^\infty
	\modulus{d_k}^2 < \infty$. Thus, if $c,d \in \ltwo$ and $\iscmplx[\lambda]$, then
	$\lambda c + d \in \ltwo$. So $\ltwo$ is a complex vector space.
\end{proof}

%%%%%%%%%%%%%%%%%%%%%%%%%%%%%%%%%%%%%%%%%%%%%%%%%%%%%%%%%%%%
%% l2 is an inner product space
%%%%%%%%%%%%%%%%%%%%%%%%%%%%%%%%%%%%%%%%%%%%%%%%%%%%%%%%%%%%

\begin{prop}
	For $c=\{c_k\}_{k=-\infty}^\infty$ and $d=\{d_k\}_{k=-\infty}^\infty$ in
	$\ltwo$, define
		\begin{displaymath}
			\innrprod{c}{d} = \sum_{k=-\infty}^\infty c_k \conj{d_k}
		\end{displaymath}
	Then this function is well-defined and is, in fact, an inner product
	on $\ltwo$.
\end{prop}

\begin{proof}
	We will divide the proof into steps.
		\begin{enumerate}[{Step} 1.]
		
			%% The inner product is absolutely convergent.
			\item
				Again, using the identity
					\begin{displaymath}
						2\modulus{c_k\conj{d_k}} = 2\modulus{c_k d_k} 
							\leq \modulus{c_k}^2 + \modulus{d_k}^2
					\end{displaymath}
				we see that $\innrprod{c}{d}$ converges absolutely
				and hence is well-defined.
				
			%% It is non-negative
			\item
				If $c \in \ltwo$ then
					\begin{displaymath}
						\innrprod{c}{c} = \sum_{k=\infty}^\infty \modulus{c_k}^2
							\geq 0
					\end{displaymath}
				with equality iff $c_k = 0$ for all $\isintgr[k]$.
				
			%% It is linear
			\item
				If $a,b,c \in \ltwo$ and $\iscmplx[\lambda]$ then
					\begin{IEEEeqnarray*}{rCl}
						\innrprod{\lambda a+b}{c} & = & \sum_{k=-\infty}^\infty
							(\lambda a_k+b_k)\conj{c_k} \\
						& = & \lambda \sum_{k=-\infty}^\infty a_k\conj{c_k}
							+ \sum_{k=-\infty}^\infty b_k\conj{c_k} \\
						& = & \lambda \innrprod{a}{c}+\innrprod{b}{c}
					\end{IEEEeqnarray*}
				
			%% Conjugation
			\item
				Suppose $c,d \in \ltwo$. Then since the conjugation function
				is continuous, we have
					\begin{displaymath}
						\conj{\sum_{k=-n}^n c_k\conj{d_k}} \rightarrow
							\conj{\innrprod{c}{d}}
					\end{displaymath}
				as $n \rightarrow \infty$. But
					\begin{displaymath}
						\conj{\sum_{k=-n}^n c_k\conj{d_k}} = \sum_{k=-n}^n
							\conj{c_k\conj{d_k}} = \sum_{k=-n}^n d_k \conj{c_k}
							\rightarrow \innrprod{d}{c}
					\end{displaymath}
				as $n \rightarrow \infty$. Thus $\conj{\innrprod{c}{d}} = 
				\innrprod{d}{c}$.
		
		\end{enumerate}
\end{proof}

	As usual, we will denote $\sqrt{\innrprod{c}{c}}$ by $\norm[]{c}$.
	
%%%%%%%%%%%%%%%%%%%%%%%%%%%%%%%%%%%%%%%%%%%%%%%%%%%%%%%%%%%%
%% The map f to f hat is an embedding and the image is dense
%%%%%%%%%%%%%%%%%%%%%%%%%%%%%%%%%%%%%%%%%%%%%%%%%%%%%%%%%%%%
				
\begin{prop}
	The map $f \mapsto \f$ is an embedding from $\riemmint$ into
	$\ltwo$ and the set $\{\f \; | \; f \in \riemmint\}$ is dense
	in $\ltwo$.
\end{prop}

\begin{proof}
	By Corollary \ref{cor:OrthoBasis}, this map is linear and preserves
	inner products. Moreover, if $\f(k)=0$ for all $\isintgr[k]$ then
	$f=0$. So this map is one-to-one. Therefore it is an embedding from
	$\riemmint$ into $\ltwo$.
	
	To show that the image of this map is dense in $\ltwo$, define
	$b_k(j)=\delta_{jk}$ for $j,k \in \intgr$. If
	$c=\{c_k\}_{k=-\infty}^\infty$ is in $\ltwo$ then $c =
	\sum_{k=-\infty}^\infty c_k b_k$. Hence $\{b_k\}_{k=-\infty}
	^\infty$ is a basis for $\ltwo$. Moreover, if $\{e_k\}
	_{k=-\infty}^\infty$ is the basis for $\riemmint$ described
	in the previous section, then $\f[e]_k=b_k$ for all $\isintgr[k]$.
	So given $c \in \ltwo$, let $f_n = \sum_{k=-n}^n c_k e_k$. Then
	$f_n \in \riemmint$ and
		\begin{IEEEeqnarray*}{rCl}
			\norm[]{c-\f_n}^2 & = & \norm[]{\sum_{k=-\infty}^\infty c_k b_k
				- \sum_{k=-n}^n c_k \f[e]_k}^2 \\
			& = & \norm[]{\sum_{k=-\infty}^\infty c_k b_k 
				- \sum_{k=-n}^n c_k b_k}^2 \\
			& = & \norm[]{\sum_{\substack{\modulus{k}>n}} c_k b_k}^2 \\
			& = & \sum_{\substack{\modulus{k}>n}} \modulus{c_k}^2
		\end{IEEEeqnarray*}
	$\rightarrow 0$ as $n \rightarrow \infty$. Hence the set 
	$\{\f \; | \; f \in \riemmint\}$ is dense in $\ltwo$.
\end{proof}
	
%%%%%%%%%%%%%%%%%%%%%%%%%%%%%%%%%%%%%%%%%%%%%%%%%%%%%%%%%%%%
%% l2 is complete
%%%%%%%%%%%%%%%%%%%%%%%%%%%%%%%%%%%%%%%%%%%%%%%%%%%%%%%%%%%%

	The final property of $\ltwo$ we will prove is that it is complete
	with respect to the norm induced from its inner product. To prove
	this result, we will need the following inequality, which will be
	presented without proof.

%% Minkowski's Inequality
\begin{prop}[Minkowski's Inequality]\label{prop:Minkowski}
	Suppose $\{A_{kn}\}$ is an array of complex numbers
	which satisfy
		\begin{displaymath}
			\sum_{k=-\infty}^\infty \modulus{A_{kn}} < \infty
		\end{displaymath}
	for all $\isintgr[n]$. Then
		\begin{displaymath}
			\left(\sum_{n=-\infty}^\infty \modulus{
				\sum_{k=-\infty}^\infty A_{kn}}^2 \right)^{1/2}
				= \sum_{k=-\infty}^\infty \left(
				\sum_{n=-\infty}^\infty \modulus{A_{kn}}^2 \right)^{1/2}
		\end{displaymath}
\end{prop}

%% l2 is complete
\begin{thrm}
	$\ltwo$ is complete with respect to the norm induced from its inner
	product.
\end{thrm}

\begin{proof}
	We will divide the proof into steps.
		\begin{enumerate}[{Step} 1]
		
			%% Find candidate for limit of the sequence
			\item
				Let $\{c_k\}_{k=1}^\infty$ be an infinite Cauchy 
				sequence in $\ltwo$ and	for each $k$ denote the 
				terms of $c_k$ by $\{c_{k,n}\}_{n=-\infty}^\infty$.
				Then given $\epsilon > 0$ there exists $K > 0$ 
				such that if $j,k > K$ and $\isintgr[n]$ is fixed 
				we have
					\begin{displaymath}
						\modulus{c_{k,n}-c_{j,n}} 
							\leq \sum_{m=-\infty}^\infty
							\modulus{c_{k,m}-c_{j,m}}^2 
							= \norm[]{c_k-c_j}^2
							< \epsilon
					\end{displaymath}
				Hence for each $\isintgr[n]$, $\{c_{k,n}\}_{k=1}^\infty$ 
				is a complex Cauchy sequence. Therefore this sequence 
				converges to some $d_n$. Let $d=\{d_n\}_{n=-\infty}^\infty$. 
			
			%% Show that the candidate has finite norm
			\item	
				We will now show that $d \in \ltwo$. Since $\{c_k\}_{k=1}^\infty$
				is Cauchy it is bounded and hence there exists $M > 0$ 
				such that $\norm[]{c_k} < M$ for all $k \geq 1$. Then
					\begin{IEEEeqnarray*}{rCl}
						\norm[]{d}^2 & = & \lim_{N \rightarrow \infty}
							\sum_{\substack{\modulus{n}<N}} \modulus{d_n}^2 \\
						& = & \lim_{k,N \rightarrow \infty}
							\sum_{\substack{\modulus{n}<N}} \modulus{c_{k,n}}^2 \\
						& = & \lim_{k \rightarrow \infty} \norm[]{c_k}^2 \\
						& < & M^2
					\end{IEEEeqnarray*}
			
			%% This step and next show that the sequence in l2 does indeed
			%% converge to the desired limit
			\item
				Choose a strictly increasing sequence of positive integers
				$\{m_k\}_{k=1}^\infty$ such that
				if $m > m_k$ then $\norm[]{c_m-c_{m_k}} < 2^{-k-1}$. Again, this can 
				be done since the sequence $\{c_k\}_{k=1}^\infty$ is Cauchy. Then for
				fixed $\isintgr[n]$ and fixed $K > 0$,
					\begin{displaymath}
						d_n = c_{m_K,n} + \sum_{k=K}^\infty (c_{m_{k+1},n}-c_{m_k,n})
					\end{displaymath}
				Therefore,
					\begin{IEEEeqnarray*}{rCl}
						\norm[]{d-c_{m_K}} & = & \left( \sum_{n=-\infty}^\infty
							\modulus{d_n-c_{m_K,n}}^2 \right)^{1/2} \\
						& = & \left( \sum_{n=-\infty}^\infty \modulus{
							\sum_{k=K}^\infty (c_{m_{k+1},n}-c_{m_k,n})}^2 
							\right)^{1/2} \\
						& = & \sum_{k=K}^\infty \left( \sum_{n=-\infty}^\infty
							\modulus{c_{m_{k+1},n}-c_{m_k,n}}^2 \right)^{1/2} 
							\; \text{by Proposition \ref{prop:Minkowski}} \\
						& = & \sum_{k=K}^\infty \norm[]{c_{m_{k+1}}-c_{m_k}} \\
						& < & \sum_{k=K}^\infty 2^{-k-1} \; \text{by the choice
							of } m_k \\
						& = & 2^{-K}
					\end{IEEEeqnarray*}
			
			\item
				For the final step, given $\epsilon > 0$ choose $K > 0$ such
				that $2^{-K} < \epsilon/2$. Then if $m > m_K$ we have
					\begin{displaymath}
						\norm[]{d-c_m} \leq \norm[]{d-c_{m_K}}
							+ \norm[]{c_m-c_{m_K}}
							< 2^{-K} + 2^{-K-1}
							< \frac{\epsilon}{2}+\frac{\epsilon}{2}
							= \epsilon
					\end{displaymath}
		
		\end{enumerate}
\end{proof}

\end{section}