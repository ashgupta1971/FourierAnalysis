\begin{section}{\fejers Example}

	In this section, we present an example
	of a continuous, $2\pi$-periodic function
	whose Fourier series diveges at zero. But
	we first need to develop some preliminary
	lemmas.
		
%%%%%%%%%%%%%%%%%%%%%%%%%%%%%%%%%%%%%%%%%%%%%%%%
%% First Lemma
%%%%%%%%%%%%%%%%%%%%%%%%%%%%%%%%%%%%%%%%%%%%%%%%

\begin{lemma}\label{lemma:FejersExample1}
	Suppose $\isreal, x \neq 2\pi k, \isintgr[k]$ so
	that $e^{ix} \neq 1$ and $\sin x/2 \neq 0$. Then
	for all $\isnatrl$,
		\begin{enumerate}[i)]
		
			\item
				\begin{displaymath}
					\sum_{k=1}^n e^{ikx} = \left(
						\frac{\sin nx/2}{\sin x/2}
						\right) e^{i(n+1)x/2}
				\end{displaymath}
				
			\item
				\begin{displaymath}
					\modulus{\sum_{k=1}^n e^{ikx}} \leq 
						\frac{1}{\sin x/2}
				\end{displaymath}
				
			\item
				\begin{displaymath}
					\sum_{k=1}^n \sin kx = \frac{
						(\sin nx/2)(\sin (n+1)x/2)}
						{\sin x/2}
				\end{displaymath}
				
			\item
				\begin{displaymath}
					\modulus{\sum_{k=1}^n \sin kx} \leq
						\frac{1}{\modulus{\sin x/2}}
				\end{displaymath}
		
		\end{enumerate}
\end{lemma}

\begin{proof}
	\begin{enumerate}[i)]
	
		\item
			If $x$ satisfies the hypothesis of the lemma
			and $\isnatrl$, then
				\begin{IEEEeqnarray*}{rCl}
					\sum_{k=1}^n e^{ikx} & = & e^{ix} \left(
						\frac{1-e^{inx}}{1-e^{ix}} \right) \\
					& = & \frac{e^{ix} e^{inx/2}}{e^{ix/2}} \left(
						\frac{e^{-inx/2}-e^{inx/2}}{e^{-ix/2} -
						e^{ix/2}} \right) \\
					& = & e^{i(n+1)x/2} \left( \frac{-2\sin nx/2}
						{-2\sin x/2} \right) \\
					& = & e^{i(n+1)x/2} \frac{\sin nx/2}{\sin x/2}
				\end{IEEEeqnarray*}
		
		\item
			This follows from part i).
			
		\item
			This follows from part i) by taking imaginary parts
			of both sides of the identity.
			
		\item
			This follows from part iii).
			
	\end{enumerate}
\end{proof}

%%%%%%%%%%%%%%%%%%%%%%%%%%%%%%%%%%%%%%%%%%%%%%%%
%% Example 1 using Lemma 1
%%%%%%%%%%%%%%%%%%%%%%%%%%%%%%%%%%%%%%%%%%%%%%%%

\begin{ex}
	\begin{enumerate}[i)]
		\item
			Suppose $\{b_k\}_{k=1}^\infty$ is a real
			sequence decreasing to zero. Show that
				\begin{displaymath}
					\sum_{k=1}^\infty b_k \sin kx
				\end{displaymath}
			converges for all $\isreal$.
		\item
			Show that the given series converges 
			uniformly on $0 < a \leq x \leq b < \pi$.
	\end{enumerate}
\end{ex}

\begin{soln}
	\begin{enumerate}[i)]
		\item
			Clearly, the given series converges if $x = 2\pi k$
			for some $\isintgr[k]$. So assume $x$ is not a multiple
			of $2\pi$. Then by part iv) of Lemma 
			\ref{lemma:FejersExample1},
				\begin{displaymath}
					\modulus{\sum_{k=1}^n \sin kx}
						< \frac{1}{\modulus{\sin x/2}}
				\end{displaymath}
			Now given $\epsilon > 0$, choose $\isnatrl[p]$ such
			that $b_p < (\epsilon \modulus{\sin x/2})/2$. Then by Theorem 
			\ref{thrm:AbelsLemma}, for $q > p$,
				\begin{IEEEeqnarray*}{rCl}
					\modulus{\sum_{k=p}^q b_k \sin kx} & < & 
						2 \left(\frac{1}{\modulus{\sin x/2}}\right)b_p \\
					& < & \epsilon
				\end{IEEEeqnarray*}
			So the given series is Cauchy and hence converges.
		\item
			By part iv) of Lemma \ref{lemma:FejersExample1},
				\begin{IEEEeqnarray*}{rCl}
					\modulus{\sum_{k=1}^n \sin kx} & < &
						\frac{1}{\modulus{\sin x/2}} \\
					& \leq & \frac{1}{\sin a/2}
				\end{IEEEeqnarray*}
			So by Theorem \ref{thrm:DirTestUniConv}, the convergence
			of the given series is uniform.
	\end{enumerate}
\end{soln}		

%%%%%%%%%%%%%%%%%%%%%%%%%%%%%%%%%%%%%%%%%%%%%%%%
%% Example 2 using Lemma 1
%%%%%%%%%%%%%%%%%%%%%%%%%%%%%%%%%%%%%%%%%%%%%%%%

\begin{ex}
	Show that the series
		\begin{displaymath}
			\sum_{k=1}^\infty \frac{(-1)^k}{\sqrt{k}}
				\sin (1 + x/k)
		\end{displaymath}
	converges uniformly on $\real$.
\end{ex}

\begin{soln}
\end{soln}

%%%%%%%%%%%%%%%%%%%%%%%%%%%%%%%%%%%%%%%%%%%%%%%%
%% Second Lemma
%%%%%%%%%%%%%%%%%%%%%%%%%%%%%%%%%%%%%%%%%%%%%%%%

\begin{lemma}
	Suppose $0 < \delta < \pi$. Then
		\begin{displaymath}
			\sum_{k=1}^\infty \frac{1}{k}\sin kx
		\end{displaymath}
	converges uniformly for $\delta \leq x \leq 2\pi-\delta$.
\end{lemma}

\begin{proof}
	By part iv) of Lemma \ref{lemma:FejersExample1},
		\begin{displaymath}
			\modulus{\sum_{k=1}^n \sin kx} \leq \frac{1}{\sin x/2}
				\leq \frac{1}{\sin \delta/2}
		\end{displaymath}
	if $\delta \leq x \leq 2\pi-\delta$. Therefore the result
	follows by Theorem \ref{thrm:DirTestUniConv}.
\end{proof}

%%%%%%%%%%%%%%%%%%%%%%%%%%%%%%%%%%%%%%%%%%%%%%%%
%% Final Lemma for Fejer's Example
%%%%%%%%%%%%%%%%%%%%%%%%%%%%%%%%%%%%%%%%%%%%%%%%

\begin{lemma}\label{lemma:FejersExample3}
	If $\isreal, \isnatrl$, then
		\begin{displaymath}
			\modulus{\sum_{k=1}^n \frac{\sin kx}{k}} 
				\leq 9\sqrt{\pi}
		\end{displaymath}
\end{lemma}

\begin{proof}
	We may assume, without loss of generality, that $0 < x < 2\pi$.
	Choose $\isnatrl[p]$ such that $p \leq \sqrt{\pi}/x < p+1$.
	Suppose $p < n$. Then
		\begin{IEEEeqnarray*}{rCl}
			\modulus{\sum_{k=1}^n \frac{\sin kx}{k}}
				& \leq & \sum_{k=1}^p \modulus{\frac{\sin kx}{k}}
				+ \modulus{\sum_{k=p+1}^n \frac{\sin kx}{k}} \\
			& = & \sum_{k=1}^p \modulus{\frac{\sin kx}{kx}}x
				+ \modulus{\sum_{k=p+1}^n \frac{\sin kx}{k}} \\
			& \leq & px + \modulus{\sum_{k=p+1}^n \frac{\sin kx}{k}}
		\end{IEEEeqnarray*}
	It follows from part iii) of Lemma \ref{lemma:FejersExample1}
	that
		\begin{displaymath}
			\modulus{\sum_{k=p+1}^n \frac{\sin kx}{k}}
				\leq \frac{2}{\sin x/2}
		\end{displaymath}
	and hence, by a slight modification of Theorem 
	\ref{thrm:AbelsLemma} that
		\begin{displaymath}
			\modulus{\sum_{k=p+1}^n \frac{\sin kx}{k}}
				\leq 2 \left(\frac{2}{\sin x/2}\right)
				\frac{1}{p+1}
		\end{displaymath}
	Therefore,
		\begin{IEEEeqnarray*}{rCl}
			\modulus{\sum_{k=1}^n \frac{\sin kx}{k}} & \leq &
				px + \frac{4}{(p+1)\sin x/2} \\
			& \leq & \sqrt{\pi} + \frac{4}{p+1}\frac{x/2}{\sin x/2}
				\frac{2}{x} \\
			& \leq & \sqrt{\pi} + \frac{8}{(p+1)x}\left(\frac{x/2}
				{\sin x/2}\right) \\
			& \leq & \sqrt{\pi} + \frac{8}{\sqrt{\pi}}\pi \\
			& = & 9\sqrt{\pi}
		\end{IEEEeqnarray*}
\end{proof}

%%%%%%%%%%%%%%%%%%%%%%%%%%%%%%%%%%%%%%%%%%%%%%%%
%% Fejer's Example
%%%%%%%%%%%%%%%%%%%%%%%%%%%%%%%%%%%%%%%%%%%%%%%%

\begin{thrm}[\fejers Example]
	Given $\mu, n \in \natrl$ and $\isreal$, define
		\begin{IEEEeqnarray*}{rCl}
			Q(x,\mu,n) & = & \frac{\cos \mu x}{n} +
				\frac{\cos (\mu+1)x}{n-1} + \ldots +
				\frac{\cos (\mu+n-1)x}{1} \\
			& & - \frac{\cos (\mu+n+1)x}{1} -
				\frac{\cos (\mu+n+2)x}{2} - \ldots
				- \frac{\cos (\mu+2n)x}{n}
		\end{IEEEeqnarray*}
	Now suppose $\{\mu_k\}_{k=1}^\infty$ and $\{n_k\}_{k=1}^\infty$
	are two increasing sequences in $\natrl$ satisfying
	$\mu_k+2n_k < \mu_{k+1}$ and also that $\{a_k\}_{k=1}^\infty$
	is a sequence of positive reals such that
		\begin{displaymath}
			\sum_{k=1}^\infty a_k
		\end{displaymath}
	converges. Then the series
		\begin{displaymath}
			\sum_{k=1}^\infty a_k Q(x,\mu_k,n_k)
		\end{displaymath}
	converges uniformly and absolutely to a continuous, 
	$2\pi$-periodic function $f$. Moreover, the $\mu_k, n_k$
	and $a_k$ can be chosen so that the Fourier series
	of $f$ diverges at 0.
\end{thrm}

\begin{proof}
	We will divide the proof into multiple steps.
		\begin{enumerate}[{Step} 1.]
		
			% Step 1. Show that the series converges uniformly.
			\item
				Using the identity $\cos(a-b)-\cos(a+b) 
				= 2\sin a \sin b$, we obtain
					\begin{IEEEeqnarray*}{rCl}
						Q(x,\mu_k,n_k) & = & \frac{\cos \mu_k x}
							{n_k} + \frac{\cos (\mu_k+1)x}
							{n_k-1} + \ldots +
							\frac{\cos (\mu_k+n_k-1)x}{1} \\
						& & - \frac{\cos (\mu_k+n_k+1)x}{1} -
							\frac{\cos (\mu_k+n_k+2)x}{2} - \ldots
							- \frac{\cos (\mu_k+2n_k)x}{n_k} \\
						& = & \sum_{k=1}^{n_k} \frac{1}{k} \left(
							\cos(\mu_k+n_k-k)x-\cos(\mu_k+n_k+k)x
							\right) \\
						& = & 2 \sin(\mu_k+n_k)\sum_{k=1}^{n_k}
							\frac{\sin kx}{k}
					\end{IEEEeqnarray*}
				But by Lemma \ref{lemma:FejersExample3}
					\begin{displaymath}
						\modulus{\sum_{k=1}^{n_k} \frac{\sin kx}{k}}
							\leq 9\sqrt{\pi}
					\end{displaymath}
				and so, $\modulus{Q(x,\mu_k,n_k)} < 18\sqrt{\pi}$.
				Now, given $\epsilon > 0$ choose $N > 0$ such
				that
					\begin{displaymath}
						\sum_{k=p}^q a_k < \frac{\epsilon}
							{18\sqrt{\pi}}
					\end{displaymath}
				for $q > p > N$. Then for this choice of $p$ and
				$q$ and all $\isreal$, we have
					\begin{IEEEeqnarray*}{rCl}
						\modulus{\sum_{k=p}^q a_k Q(x,\mu_k,n_k)}
							& < & 18\sqrt{\pi} \sum_{k=p}^q a_k \\
						& < & 18\sqrt{\pi} \frac{\epsilon}
							{18\sqrt{\pi}} \\
						& = & \epsilon
					\end{IEEEeqnarray*}
				Therefore, the series
					\begin{displaymath}
						\sum_{k=1}^\infty a_k Q(x,\mu_k,n_k)
					\end{displaymath}
				satisfies the uniform Cauchy criterion on $\real$, 
				and hence converges uniformly and absolutely to a 
				continuous, $2\pi$-periodic function $f$.
	
			% Step 2. Find the Fourier series of f.
			\item
				Notice that because $\mu_k+2n_k < \mu_{k+1}$, the
				terms of $Q(x,\mu_k,n_k)$ and $Q(x,\mu_j,n_j)$ 
				do not "overlap" when $j \neq k$. Also notice that
					\begin{displaymath}
						f(x) = \sum_{k=1}^\infty a_k Q(x,\mu_k,n_k)
					\end{displaymath}
				is obtained by "bracketing" a trigonometric series 
				of the form
					\begin{displaymath}
						\frac{\alpha_0}{2}+\sum_{\nu=1}^\infty 
							\alpha_\nu \cos \nu x \; (\ast)
					\end{displaymath}
				Let $s_n, \isnatrl$ be the $n$th partial sum of this
				series.	Then $s_{\mu_k-1}(x)$ converges uniformly on
				$\real$ to $f$. We will show that the series $(\ast)$
				is indeed the Fourier series of $f$. To see this, 
				first suppose $\mu_j \leq \nu \leq \mu_j+2n_j$. Then
					\begin{IEEEeqnarray*}{rCl}
						\frac{1}{\pi}\myintb[\pi]{f(x)\cos \nu x}{x}
							& = & \frac{1}{\pi}\myintb[\pi]
							{\sum_{k=1}^\infty a_k Q(x,\mu_k,n_k)
							\cos \nu x}{x} \\
						& = & \frac{1}{\pi} \sum_{k=1}^\infty
							a_k \myintb[\pi]{Q(x,\mu_k,n_k)
							\cos \nu x}{x} \\
						& = & \frac{a_j}{\pi} \myintb[\pi]{Q(x,\mu_j,n_j) 
							\cos \nu x}{x} \\
						& = & \alpha_\nu
					\end{IEEEeqnarray*}
				Next, suppose
					\begin{displaymath}
						\nu \notin \bigcup_{j=1}^\infty
							\{\mu_j,\ldots,\mu_j+n_j-1,
							\mu_j+n_j+1,\ldots,\mu_j+2n_j\}
					\end{displaymath}
				Then
					\begin{IEEEeqnarray*}{rCl}
						\frac{1}{\pi}\myintb[\pi]{f(x)\cos \nu x}{x}
							& = & \frac{1}{\pi}\myintb[\pi]
							{\sum_{k=1}^\infty a_k Q(x,\mu_k,n_k)
							\cos \nu x}{x} \\
						& = & 0
					\end{IEEEeqnarray*}
				which proves the assertion.	

			% Step 3. Show that the Fourier series diverges at 0.
			\item
				Finally, we will show that we can choose the $\mu_k,
				n_k$ and $a_k$ so that the Fourier series of $f$
				diverges at zero. First notice that
					\begin{IEEEeqnarray*}{rCl}
						\modulus{s_{\mu_k+n_k-1}(0) - s_{\mu_k-1}(0)}
							& = & \sum_{\nu=\mu_k}^{\mu_k+n_k-1}
							\alpha_\nu \cos \nu 0 \\
						& = & a_k \left(\frac{1}{n_k}+\frac{1}{n_k-1}
							+ \ldots + 1 \right) \\
						& > & a_k \log n_k
					\end{IEEEeqnarray*}
				Hence $\{s_n(0)\}_{n=1}^\infty$ will not be Cauchy, and hence
				divergent, if we choose	$\{a_k\}_{k=1}^\infty$ and 
				$\{n_k\}_{k=1}^\infty$ so that $\{a_k \log n_k\}_{k=1}^\infty$
				does not converge to zero. It is easy to see that if we let
				$a_k = 1/k^2$ and $\mu_k = n_k = 2^{k^3}$ then this requirement
				is met as well as the condition $\mu_k+2n_k < \mu_{k+1}$.
			
		\end{enumerate}
\end{proof}

	We conclude this section with the statement of two theorems
	closely related to the previous one. However, we will not prove
	these theorems here.
	
%%%%%%%%%%%%%%%%%%%%%%%%%%%%%%%%%%%%%%%%%%%%%%%%
%% Statement of 2 theorems similar to Fejers
%% Example
%%%%%%%%%%%%%%%%%%%%%%%%%%%%%%%%%%%%%%%%%%%%%%%%

\begin{thrm}[Carlson]
	If $f$ is a continuous, $2\pi$-periodic function then
	its Fourier series converges pointwise to $f$ almost
	everywhere.
\end{thrm}

\begin{thrm}[Katznelson and Kahame]
	If $E \subset \real$ is a null set then there is a function
	$f$ such that for all $t \in E$,
		\begin{displaymath}
			\limsup_{n \rightarrow \infty} \modulus{s_n(t)}
				= \infty
		\end{displaymath}
	(where $s_n$ is the $n$th partial sum of the Fourier series
	of $f$).
\end{thrm}

\end{section}