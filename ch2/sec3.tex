\begin{section}{The Fourier Problem}

Recall from section 1 of this chapter that
the heat problem has a solution corresponding
to $f \in C_0[0,\pi]$ if we can
find real constants $b_1, b_2, \ldots, b_n$
such that
	\begin{displaymath}
		f(x) = \sum_{k=1}^n b_k \sin kx
	\end{displaymath}
for $x \in [0,\pi]$.

%%%%%%%%%%%%%%%%%%%%%%%%%%%%%%%%%%%%%%%%%%%
%% Fourier Problem
%%%%%%%%%%%%%%%%%%%%%%%%%%%%%%%%%%%%%%%%%%%

In this section, we ask a slightly different
question, called the \emph{Fourier Problem}:
Given $f:\real \to \real$ is it possible to
find real numbers $a_0, a_1, b_1, a_2, b_2,
\ldots$ such that
	\begin{displaymath}
		f(x) = \frac{a_0}{2} + \sum_{k=1}^\infty
			(a_k \cos kx + b_k \sin kx)
	\end{displaymath}
for all $x \in \real$. If this is possible,
then we must have $f(x + 2\pi) = f(x)$, which
leads to the following definition and results.

%%%%%%%%%%%%%%%%%%%%%%%%%%%%%%%%%%%%%%%%%%%%%
%% Periodic functions
%%%%%%%%%%%%%%%%%%%%%%%%%%%%%%%%%%%%%%%%%%%%%

\begin{defn}
	If $f$ is a complex-valued function on $\real$
	and $p$ is a non-zero real number, we say that
	\emph{f has period p} or \emph{f is p-periodic}
	provided $f(x+p) = f(x)$ for all $x \in \real$.
	We denote the set of all p-periodic functions by
	$P_p$.
\end{defn}

\begin{prop}\label{prop:periodic}
	\begin{enumerate}[i)]
		\item
			If $f \in P_p$ then $f(x+kp) = f(x)$
			for all $k \in \intgr$ and $x \in \real$.
		\item
			Let $T = \{z \in \cmplx: \modulus{z} = 1\}$ and
			suppose $g:T \to \cmplx$ is given. Then $f:
			\real \to \cmplx$ defined by $f(x) = g(e^{ix})$
			is $2\pi$-periodic. Conversely, given $f \in
			P_{2\pi}$, there exists a unique $g:T \to \cmplx$
			such that $f(x) = g(e^{ix})$ for all $x \in \real$.
			The map $f \mapsto g$ is an algebra isomorphism
			between $P_{2\pi}$ and the set of functions from
			$T$ to $\cmplx$.
		\item
			Given $a \in \real$ and $f_0:[a,a+2\pi) \to
			\cmplx$ there exists a unique $2\pi$-periodic
			function $f$ such that $f_0(x) = f(x)$ for all
			$x \in [a,a+2\pi)$.
		\item
			If $f \in P_p$ and we let $g(x) = f(px)$ then
			$g \in P_1$ and the map $f \mapsto g$ is and
			algebra isomorphism between $P_p$ and $P_1$.
		\item
			If $f \in P_{2\pi}$ then $f$ is Riemann Integrable
			on $[a,b]$ whenever $a < b$ iff $f$ is Riemann
			Integrable on $[c,c+2\pi]$ for some $c \in \real$,
			in which case
				\begin{displaymath}
					\int_c^{c+2\pi} f = \int_0^{2\pi} f
				\end{displaymath}
	\end{enumerate}
\end{prop}

\begin{proof}
	\begin{enumerate}[i)]
		\item
			Trivial.
		\item
			Trivial.
		\item
			Trivial.
		\item
			It suffices to note that $g(x+1) = f(p(x+1))
			= f(px + p) = f(px) = g(x)$.
		\item
			The first statement is trivial. To prove the
			equality of the integrals, choose $n \in \intgr$
			such that $2\pi n < c \leq 2\pi (n+1)$. Then
				\begin{IEEEeqnarray*}{rCl}
					\myinta{c}{c+2\pi}{f(x)}{x} & = &
						\myinta{c}{2\pi (n+1)}{f(x)}{x}
						+ \myinta{2\pi (n+1)}{c+2\pi}{f(x)}{x} \\
					& = & \myinta{c-2\pi n}{2\pi}{f(x+2\pi n)}{x}
						+ \myinta{0}{c-2\pi n}{f(x + 2\pi (n+1))}{x} \\
					& = & \myinta{c-2\pi n}{2\pi}{f(x)}{x}
						+ \myinta{0}{c-2\pi n}{f(x)}{x}
						\text{\ \ by part i)} \\
					& = & \myinta{0}{2\pi}{f(x)}{x}
				\end{IEEEeqnarray*}
	\end{enumerate}
\end{proof}

In the remainder of this book we will only consider functions
which are $2\pi$-periodic.

%%%%%%%%%%%%%%%%%%%%%%%%%%%%%%%%%%%%%%%%%%%%%%%%%%%%%%%
%% Derivation of trigonometric fourier coeffs.
%%%%%%%%%%%%%%%%%%%%%%%%%%%%%%%%%%%%%%%%%%%%%%%%%%%%%%%

In order to find real numbers $a_0, a_1, b_1, a_2, b_2,
\ldots$ such that
	\begin{displaymath}
		f(x) = \frac{a_0}{2} + \sum_{k=1}^\infty
			(a_k \cos kx + b_k \sin kx)
	\end{displaymath}
for all $x \in \real$, we need the following lemma.

%% Lemma of trig. identities
\begin{lemma}\label{lemma:trigidentities}
	If $m, n \in \intgr$ then
		\begin{enumerate}[i)]
			\item
				\begin{IEEEeqnarray*}{l}
					\myintb[\pi]{\cos mx}{x}
					 = \myintb[\pi]{\sin mx}{x}
					 = 0
				\end{IEEEeqnarray*}
				provided $m \neq 0$.
			\item
				\begin{IEEEeqnarray*}{l}
					\myintb[\pi]{\cos mx \sin nx}{x}
						= 0
				\end{IEEEeqnarray*}
			\item
				\begin{IEEEeqnarray*}{l}
					\myintb[\pi]{\cos mx \cos nx}{x} \\
						= \myintb[\pi]{\sin mx \sin nx}{x}
						= 0 \text{ if } m \neq \pm n
				\end{IEEEeqnarray*}
			\item
				\begin{IEEEeqnarray*}{l}
					\myintb[\pi]{\cos^2 mx}{x}
						 = \myintb[\pi]{\sin^2 mx}{x}
						 = \pi
				\end{IEEEeqnarray*}
			\item
				\begin{IEEEeqnarray*}{l}
					\myintb[\pi]{e^{imx}e^{-inx}}{x} =
						\begin{cases}
							0 & \text{if } m \neq n \\
							2\pi & \text {if } m = n
						\end{cases}
				\end{IEEEeqnarray*}
		\end{enumerate}
\end{lemma}

\begin{proof}
	\begin{enumerate}[i)]
		\item
			Trivial.
		\item
			It suffices to see that $\cos mx \sin nx$
			is odd.
		\item
			Integrating by parts twice, we obtain
				\begin{IEEEeqnarray*}{l}
					\myintb[\pi]{\cos mx \cos nx}{x} \\
					= \evalat{
						\frac{\sin mx \cos nx}{m}}{x}{-\pi}
						{\pi}
						+ \frac{n}{m}\myintb[\pi]
						{\sin mx \sin nx}{x} \\
					= \evalat{\frac{-n \cos mx \sin nx}{m^2}}{x}
						{-\pi}{\pi}
						+ \frac{n^2}{m^2}\myintb[\pi]
						{\cos mx \cos nx}{x} \\
					= \frac{n^2}{m^2}\myintb[\pi]
						{\cos mx \cos nx}{x}
				\end{IEEEeqnarray*}
			Since $m \neq \pm n$, we must have
				\begin{displaymath}
					\myintb[\pi]{\cos mx \cos nx}{x}
						= 0
				\end{displaymath}
			The other integral is handled similarly.
		\item
			Using the identity $\cos 2x = 2 \cos^2 x - 1$
			we obtain,
				\begin{IEEEeqnarray*}{rCl}
					\myintb[\pi]{\cos^2 mx}{x} & = &
						\myintb[\pi]{\frac{1+\cos 2mx}{2}}{x} \\
					& = & \evalat{\frac{x}{2}}{x}{-\pi}{\pi}
						+ \evalat{\frac{\sin 2mx}{4m}}{x}{-\pi}{\pi} \\
					& = & \pi
				\end{IEEEeqnarray*}
			To obtain the other result, use the identity $\cos 2x
			= 1 - 2 \sin^2 x$.
		\item
			\begin{IEEEeqnarray*}{l}
				\myintb[\pi]{e^{imx}e^{-inx}}{x} =
					\myintb[\pi]{e^{i(m-n)x}}{x} \\
				= \myintb[\pi]{\cos (m-n)x}{x} +
					i \myintb[\pi]{\sin (m-n)x}{x}
			\end{IEEEeqnarray*}
			If $m \neq n$ then by part i) this expression
			is zero. If $m = n$ then $\sin (m-n)x = 0$ and
			$\cos (m-n)x = 1$, so the result follows.
	\end{enumerate}
\end{proof}

\begin{thrm}\label{thrm:trigFcoeffs}
	Suppose for some real numbers $a_0, a_1, b_1, a_2, b_2,
	\ldots$ the series
		\begin{displaymath}
			\frac{a_0}{2} + \sum_{k=1}^\infty
				(a_k \cos kx + b_k \sin kx)
		\end{displaymath}
	converges uniformly to a given function $f:\real \to
	\real$. Then
		\begin{displaymath}
			a_n = \frac{1}{\pi}\myintb[\pi]{f(x)\cos nx}{x}
				\text{ for } 0 \leq n \in \intgr
		\end{displaymath}
	and
		\begin{displaymath}
			b_n = \frac{1}{\pi}\myintb[\pi]{f(x)\sin nx}{x}
				\text{ for } 0 < n \in \intgr
		\end{displaymath}
\end{thrm}

\begin{proof}
	Since the given series converges uniformly, $f$ is
	continuous on $\real$ and hence Riemann Integrable
	on $\real$. Then, integrating $f$ on the interval
	$[-\pi,\pi]$, we obtain
		\begin{IEEEeqnarray*}{rCl}
			\myintb[\pi]{f(x)}{x} & = &
				\myintb[\pi]{\frac{a_0}{2}}{x}
				+ \sum_{k=1}^\infty \left( a_k
				\myintb[\pi]{\cos kx}{x}
				+ b_k \myintb[\pi]{\sin kx}{x}
				\right) \\
			& = & \myintb[\pi]{\frac{a_0}{2}}{x} \\
			& = & a_0 \pi
		\end{IEEEeqnarray*}
	by using Lemma \ref{lemma:trigidentities}. (Note that
	we can integrate the series term by term because of
	uniform convergence.)
	
	Similarly,
		\begin{IEEEeqnarray*}{rCl}
			\myintb[\pi]{f(x)\cos nx}{x} & = &
				\myintb[\pi]{\frac{a_0}{2}\cos nx}{x} \\
			& & + \; \sum_{k=1}^\infty \left( a_k
				\myintb[\pi]{\cos kx \cos nx}{x}
				+ b_k \myintb[\pi]{\sin kx \cos nx}{x}
				\right) \\
			& = & \myintb[\pi]{a_n \cos^2 nx}{x} \\
			& = & a_n \pi
		\end{IEEEeqnarray*}
	again by using Lemma \ref{lemma:trigidentities}. So the
	result is true for the sequence $\{a_n:0 \leq n \in \intgr\}$.
	A similar argument can be used for the sequence $\{b_n:
	0 < n \in \intgr\}$.
\end{proof}

%% Value of trig. Fourier coeffs. when f is even or odd
\begin{cor}\label{cor:EvenOdd}
	If $f:\real \to \real$ is odd then $a_n = 0$ for
	all $n \geq 0$. If $f$ is even then $b_n = 0$ for
	all $n > 0$.
\end{cor}

\begin{proof}
	If $f$ is odd then $f(x)\cos nx$ is odd for all
	$n \geq 0$. So the result follows from the formula
	for $a_n$ given in Theorem \ref{thrm:trigFcoeffs}.
	Similarly, if $f$ is even then $f(x)\sin nx$ is
	odd and so $b_n$ is zero for all $n \in \natrl$.
\end{proof}

%%%%%%%%%%%%%%%%%%%%%%%%%%%%%%%%%%%%%%%%%%%%%%%%%%%%%%
%% Derivation of exponential fourier coeffs.
%%%%%%%%%%%%%%%%%%%%%%%%%%%%%%%%%%%%%%%%%%%%%%%%%%%%%%

We now deal with a reformulation of the
Fourier Problem: Given a $2\pi$-periodic function
$f:\real \to \cmplx$, do there exist complex scalars
$\ldots, c_{-2}, c_{-1}, c_0, c_1, c_2, \ldots$
such that
	\begin{displaymath}
		f(x) = c_0 + \sum_{k=1}^\infty
			(c_k e^{ikx} + c_{-k} e^{-ikx})
	\end{displaymath}
for $x \in \real$?

The next theorem deals with this question.

\begin{thrm}
	Suppose $f:\real \to \cmplx$ is $2\pi$-periodic and
	$f(x) = F(e^{ix})$ for some complex-valued function
	$F$ analytic on the annulus $A = \{z \in \cmplx:r <
	\modulus{z} < R\}$, where $r < 1 < R$. Then the
	reformulation of the Fourier problem, given above,
	has a solution. Moreover, the scalars $\{c_k:k \in
	\intgr\}$ are given by
		\begin{displaymath}
			c_n = \frac{1}{2\pi}
				\myintb[\pi]{f(x)e^{-inx}}{x}
		\end{displaymath}
\end{thrm}

\begin{proof}
	Since $F$ is analytic on $A$, it has a Laurent
	expansion in that region, namely
		\begin{displaymath}
			F(z) = c_0 + \sum_{k=1}^\infty
				(c_k z^k + c_{-k} z^{-k})
		\end{displaymath}
	Since $f(x) = F(e^{ix})$, we have
		\begin{displaymath}
			f(x) = c_0 + \sum_{k=1}^\infty
				(c_k e^{ikx} + c_{-k} e^{-ikx})
		\end{displaymath}
	for $x \in \real$.
	
	Now the series given above converges uniformly on
	the annulus $A$, hence we can integrate term by
	term to obtain
		\begin{IEEEeqnarray*}{rCl}
			\myintb[\pi]{f(x)e^{-inx}}{x} & = &
				\sum_{k=-\infty}^\infty c_k
				\myintb[\pi]{e^{ikx}e^{-inx}}{x} \\
			& = & 2\pi c_n
		\end{IEEEeqnarray*}
	by Lemma \ref{lemma:trigidentities}.		
\end{proof}

%% Definition of exp. and trig. fourier coeffs.
\begin{defn}
	The real scalars $a_0, a_1, b_1, a_2, b_2, \ldots$
	are called the \emph{trigonometric Fourier
	coefficients of f}. The complex scalars
	$\ldots, c_{-2}, c_{-1}, c_0, c_1, c_2,
	\ldots$ are called the \emph{exponential
	Fourier coefficients of f}.
\end{defn}

%% Relationship between trig. and exp. fourier coeffs.
\begin{prop}
	The trigonometric and exponential fourier coefficients
	of a function $f$ are related by
		\begin{IEEEeqnarray*}{l}
			a_0 = 2 c_0 \\
			a_n = c_n + c_{-n} \text{ for } 0 < n \in \intgr \\
			b_n = i(c_n - c_{-n}) \text{ for } 0 < n \in \intgr
		\end{IEEEeqnarray*}
\end{prop}

\begin{proof}
	Clearly $a_0 = 2 c_0$. For $a_n, n > 0$,
		\begin{IEEEeqnarray*}{rCl}
			c_n + c_{-n} & = & \frac{1}{2\pi}
				\myintb[\pi]{f(x)e^{inx}}{x}
				+ \frac{1}{2\pi}\myintb[\pi]{f(x)e^{-inx}}{x} \\
			& = & \frac{1}{2\pi}\left(\myintb[\pi]{f(x)\cos nx}{x}
				+ i \myintb[\pi]{f(x)\sin nx}{x}\right) \\
			& & + \; \frac{1}{2\pi}\left(\myintb[\pi]{f(x)\cos (-nx)}{x}
				+ i \myintb[\pi]{f(x)\sin (-nx)}{x}\right) \\
			& = & \frac{1}{2\pi}\myintb[\pi]{2f(x)\cos nx}{x} \\
			& = & \frac{1}{\pi}\myintb[\pi]{f(x)\cos nx}{x} \\
			& = & a_n
		\end{IEEEeqnarray*}
	The expression for $b_n$ is handled similarly.
\end{proof}

%%%%%%%%%%%%%%%%%%%%%%%%%%%%%%%%%%%%%%%%%%%%%%%%%%%%%%%%
%% When does the Fourier series converge uniformly to f?
%%%%%%%%%%%%%%%%%%%%%%%%%%%%%%%%%%%%%%%%%%%%%%%%%%%%%%%%

\begin{thrm}\label{thrm:coeffbounds}
	Suppose $f$ is a $2\pi$-periodic, $\ctsdiff[p]$ function
	for some $p \in \natrl$. Then there exists $M_p > 0$
	such that
		\begin{enumerate}[i)]
			\item
				$\modulus{a_k} \leq M_p/\modulus{k}^p$
				for all $k \in \natrl$.
			\item
				$\modulus{b_k} \leq M_p/\modulus{k}^p$
				for all $k \in \natrl$.
			\item
				$\modulus{c_k} \leq M_p/\modulus{k}^p$
				for all $0 \neq k \in \intgr$.				
		\end{enumerate}
\end{thrm}

\begin{proof}
	\begin{enumerate}[i)]
		\item
			Using the formula for $a_k$ and integrating by
			parts, we obtain
				\begin{IEEEeqnarray*}{rCl}
					a_k & = & \frac{1}{\pi}\myintb[\pi]{f(x)\cos kx}{x} \\
					& = & \evalat{\frac{f(x)\sin kx}{k\pi}}{x}{-\pi}{\pi}
						+ \frac{1}{k\pi}\myintb[\pi]{f'(x)\sin kx}{x} \\
					& = & \frac{1}{k\pi}\myintb[\pi]{f'(x)\sin kx}{x}
				\end{IEEEeqnarray*}
			Now integrating by parts $p-1$ more times
				\begin{displaymath}
					a_k = \frac{1}{k^p\pi}\myintb[\pi]{f^{(p)}(x)\sin kx}{x}
				\end{displaymath}
			Since $f \in \ctsdiff[p]$, $f^{(p)}\sin kx$ is continuous
			and hence bounded on $[-\pi,\pi]$. Therefore,
				\begin{IEEEeqnarray*}{rCl}
					\modulus{a_k} & = & \frac{1}{\modulus{k}^p\pi}
						\modulus{\myintb[\pi]{f^{(p)}(x)\sin kx}{x}} \\
					& \leq & \frac{1}{\modulus{k}^p\pi}
						\myintb[\pi]{\modulus{f^{(p)}(x)\sin kx}}{x} \\
					& \leq & \frac{1}{\modulus{k}^p\pi}
						\myintb[\pi]{M}{x} \\
					& \leq \frac{M_p}{\modulus{k}^p}
				\end{IEEEeqnarray*}
			for some $M_p > 0$.
		\item
			This is handled similarly to part i).
		\item
			This is handled similarly to part i).
	\end{enumerate}
\end{proof}

\begin{cor}\label{cor:FourierUniformConv}
	If $f \in \ctsdiff[2]$ is $2\pi$-periodic then the Fourier
	series of $f$ converges uniformly on $\real$.
\end{cor}

\begin{proof}
	By Theorem \ref{thrm:coeffbounds} $\modulus{c_k} \leq
	M/\modulus{k}^2$ for some $M > 0$ and all $0 \neq k
	\in \intgr$. Therefore $\modulus{c_k e^{ikx}} \leq
	M/\modulus{k}^2$. Now the series $\sum_{k=1}
	^\infty \frac{M}{k^2}$ and $\sum_{k=-\infty}^{-1}
	\frac{M}{k^2}$ both converge. So by the Weierstrass
	M-test, the Fourier series of $f$
		\begin{displaymath}
			s_n(x) = \sum_{k=-\infty}^\infty c_k e^{ikx}
		\end{displaymath}
	converges uniformly (and absolutely) on $\real$.
\end{proof}

Although Corollary \ref{cor:FourierUniformConv} says the Fourier series of a function $f$ converges uniformly, it doesn't necessarily have to converge to the function $f$ itself. In fact it does,
and this will be shown in Theorem \ref{thrm:DirchConv}.
\end{section}