\begin{section}{Applications of \fejers Theorem}

	Here we revisit the heat problem and also present
	the H. Weyl Equidistribution Theorem, both of which
	depend on \fejers Theorem which was proved
	in the last section.
	
%%%%%%%%%%%%%%%%%%%%%%%%%%%%%%%%%%%%%%%%%%%%%%%%
%% Theorem of H. Weyl
%%%%%%%%%%%%%%%%%%%%%%%%%%%%%%%%%%%%%%%%%%%%%%%%

\begin{thrm}[H. Weyl]\label{thrm:HWeyl}
	Suppose $f$ is a complex-valued, Riemann integrable
	function on $[0,1]$ and $\gamma$ is an irrational real
	number. Let $[\gamma]$ be the largest integer less than
	or equal to $\gamma$ and $<\gamma> = \gamma - [\gamma]$
	(so that $<\gamma>$ is the fractional part of $\gamma$).
	Then
		\begin{displaymath}
			\myinta{0}{1}{f(x)}{x} = \lim_{n \rightarrow \infty}
				\frac{1}{n}\sum_{k=1}^n f(<k \gamma>)
		\end{displaymath}
\end{thrm}

\begin{proof}
	\begin{enumerate}[{Case} 1.]
		
		%% Case 1
		\item
			If $f$ is constant the result is trivially true.
		
		%% Case 2
		\item
			If $f(x) = e^{2\pi mxi}$ for some $0 \neq m \in \intgr$
			then, letting $t = 2\pi x$, we get
				\begin{displaymath}
					\myinta{0}{1}{f(x)}{x} =
						\frac{1}{2\pi}\myinta{0}{2\pi}{e^{mti}}{t} \\
					= 0
				\end{displaymath}
			Also, for $0 < n \in \intgr$,
				\begin{IEEEeqnarray*}{rCl}
					\frac{1}{n}\sum_{k=1}^n f(<k \gamma>) & = &
						\frac{1}{n}\sum_{k=1}^n e^{2\pi m<k \gamma>i} \\
					& = & \frac{1}{n}\sum_{k=1}^n
						e^{(2\pi m<k \gamma>i) + (2\pi m[k \gamma]i)} \\
					& = & \frac{1}{n}\sum_{k=1}^n e^{2\pi mk\gamma i} \\
					& = & \frac{1}{n}\sum_{k=1}^n 
						(e^{2\pi m\gamma i})^k \\
					& = & \frac{e^{2\pi m\gamma i}}{n} \left(
						\frac{1 - e^{2\pi mn\gamma i}}
						{1 - e^{2\pi m\gamma i}} \right)
				\end{IEEEeqnarray*}
			So
				\begin{displaymath}
					\modulus{\frac{1}{n}\sum_{k=1}^n f(<k \gamma>)}
						\leq \frac{2}{n(1-e^{2\pi m\gamma i})}
				\end{displaymath}
			$\rightarrow 0$ as $n \rightarrow \infty$. This proves
			the result.
		
		%% Case 3		
		\item
			Suppose there exist scalars $c_k, -N \leq k \leq N$ such
			that
				\begin{displaymath}
					f(x) = \sum_{k=-N}^N c_k e^{2\pi k\gamma xi}
				\end{displaymath}
			(ie. $f$ is a trigonometric polynomial). Then the result
			follows from cases 1 and 2.
		
		%% Case 4	
		\item
			Suppose $f$ is continuous and $f(0)=f(1)$. By Corollary
			\ref{cor:Fejers4}, given $\epsilon > 0$, there exist
			scalars $c_k, -N \leq k \leq N$ such that $\modulus{
			f(x)-g(x)} < \epsilon/2$, where
				\begin{displaymath}
					g(x) = \sum_{k=-N}^N c_k e^{2\pi kxi}
				\end{displaymath}
			This implies
				\begin{IEEEeqnarray*}{l}
					\modulus{\myinta{0}{1}{f(x)}{x}
						- \myinta{0}{1}{g(x)}{x}} \\
					\leq \myinta{0}{1}{\modulus{f(x)-g(x)}}{x} \\
					\leq \epsilon/2
				\end{IEEEeqnarray*}
			and for all $0 < n \in \intgr$,
				\begin{displaymath}
					\modulus{\frac{1}{n}\sum_{k=1}^n f(<k \gamma>)
						- \frac{1}{n}\sum_{k=1}^n g(<k \gamma>)}
						\leq \frac{1}{n}\sum_{k=1}^n
							\modulus{f(<k \gamma>)-g(<k \gamma>)}
						\leq \epsilon/2
				\end{displaymath}
			Therefore,
				\begin{IEEEeqnarray*}{rCl}
					\modulus{\frac{1}{n}\sum_{k=1}^n f(<k \gamma>)
						- \myinta{0}{1}{f(x)}{x}} & \leq &
						\modulus{\frac{1}{n}\sum_{k=1}^n f(<k \gamma>)
						- \frac{1}{n}\sum_{k=1}^n g(<k \gamma>)} \\
					& & + \modulus{\frac{1}{n}\sum_{k=1}^n g(<k \gamma>)
						- \myinta{0}{1}{g(x)}{x}} \\
					& & + \modulus{\myinta{0}{1}{f(x)}{x} -
						\myinta{0}{1}{g(x)}{x}} \\
					& \leq & \epsilon/2 + \epsilon/2 \\
					& & + \modulus{\frac{1}{n}\sum_{k=1}^n g(<k \gamma>)
						- \myinta{0}{1}{g(x)}{x}} \\
					& = & \epsilon + \modulus{\frac{1}{n}\sum_{k=1}^n
						g(<k \gamma>) - \myinta{0}{1}{g(x)}{x}}
				\end{IEEEeqnarray*}
			But, by case 3,
				\begin{displaymath}
					\myinta{0}{1}{g(x)}{x} = 
						\lim_{n \rightarrow \infty}\frac{1}{n}\sum_{k=1}^n
						g(<k \gamma>)
				\end{displaymath}
			Therefore,
				\begin{displaymath}
					\limsup_{n \rightarrow \infty}
						\modulus{\frac{1}{n}\sum_{k=1}^n f(<k \gamma>)
						- \myinta{0}{1}{f(x)}{x}} \leq 2\epsilon
				\end{displaymath}
			from which the result follows.
		
		%% Case 5	
		\item
			Suppose $f$ is a real-valued, Riemann integrable function
			on $[0,1]$. Then given $\epsilon > 0$
			we can find real-valued functions $f_-$ and $f_+$ in $\cts_1$
			such that $f_-(x) \leq f(x) \leq f_+(x)$ for all $0 \leq x \leq 1$
			and
				\begin{displaymath}
					\myinta{0}{1}{f_+(x)}{x} - \myinta{0}{1}{f_-(x)}{x} < 
						\epsilon \; (\ast)
				\end{displaymath}
			Then
				\begin{displaymath}
					\frac{1}{n}\sum_{k=1}^n f_-(<k \gamma>)
						\leq \frac{1}{n}\sum_{k=1}^n f(<k \gamma>)
						\leq \frac{1}{n}\sum_{k=1}^n f_+(<k \gamma>)
						\; (\dagger)
				\end{displaymath}
			But by case 4,
				\begin{displaymath}
					\myinta{0}{1}{f_-(x)}{x} =
						\lim_{n \rightarrow \infty}
						\frac{1}{n}\sum_{k=1}^n f_-(<k \gamma>)
				\end{displaymath}
			and
				\begin{displaymath}
					\myinta{0}{1}{f_+(x)}{x} =
						\lim_{n \rightarrow \infty}
						\frac{1}{n}\sum_{k=1}^n f_+(<k \gamma>)
				\end{displaymath}
			Substituting into $(\dagger)$ we obtain
				\begin{IEEEeqnarray*}{rCl}
					\myinta{0}{1}{f_-(x)}{x} & \leq & 
						\liminf_{n \rightarrow \infty}
						\frac{1}{n}\sum_{k=1}^n f(<k \gamma>) \\
					& \leq & \limsup_{n \rightarrow \infty}
						\frac{1}{n}\sum_{k=1}^n f(<k \gamma>) \\
					& \leq & \myinta{0}{1}{f_+(x)}{x}
				\end{IEEEeqnarray*}
			Now using $(\ast)$, we see that
				\begin{IEEEeqnarray*}{l}
					\limsup_{n \rightarrow \infty}
						\frac{1}{n}\sum_{k=1}^n f(<k \gamma>)
						- \liminf_{n \rightarrow \infty}
						\frac{1}{n}\sum_{k=1}^n f(<k \gamma>) \\
					\leq \myinta{0}{1}{f_+(x)}{x}
						- \myinta{0}{1}{f_-(x)}{x} \\
					< \epsilon
				\end{IEEEeqnarray*}
			Therefore,
				\begin{displaymath}
					\lim_{n \rightarrow \infty}
						\frac{1}{n}\sum_{k=1}^n f(<k \gamma>)
				\end{displaymath}
			exists and satisfies
				\begin{displaymath}
					\myinta{0}{1}{f_-(x)}{x} \leq
						\lim_{n \rightarrow \infty}
						\frac{1}{n}\sum_{k=1}^n f(<k \gamma>)
						\leq \myinta{0}{1}{f_+(x)}{x}
				\end{displaymath}
			But since $f_-(x) \leq f(x) \leq f_+(x)$, we also have
				\begin{displaymath}
					\myinta{0}{1}{f_-(x)}{x} \leq
						\myinta{0}{1}{f(x)}{x}
						\leq \myinta{0}{1}{f_+(x)}{x}
				\end{displaymath}
			Therefore, by $(\ast)$
				\begin{displaymath}
					\modulus{\myinta{0}{1}{f(x)}{x}
						- \lim_{n \rightarrow \infty}
						\frac{1}{n}\sum_{k=1}^n f(<k \gamma>)}
						< \epsilon
				\end{displaymath}
			which proves the result.
		
		%% Case 6 (Final case)	
		\item
			For the general case, when $f$ is complex-valued,
			simply apply case 5 to the real and imaginary parts
			of $f$.
	
	\end{enumerate}
\end{proof}
	
%%%%%%%%%%%%%%%%%%%%%%%%%%%%%%%%%%%%%%%%%%%%%%%%
%% H. Weyl Equidistribution Theorem
%%%%%%%%%%%%%%%%%%%%%%%%%%%%%%%%%%%%%%%%%%%%%%%%

\begin{thrm}[H. Weyl's Equidistribution Theorem]
	If $\gamma$ is an irrational real number and
	$0 \leq a \leq b \leq 1$ then
		\begin{displaymath}
			\lim_{n \rightarrow \infty}\frac{1}{n}
				\text{card}(\{<k\gamma> \in [a,b]:
				1 \leq k \leq n\}) = b - a
		\end{displaymath}
\end{thrm}

\begin{proof}
	Define $f:[0,1] \rightarrow \real$ by
		\begin{displaymath}
			f(x) =
				\begin{cases}
					1 & a \leq x \leq b \\
					0 & \text{otherwise}
				\end{cases}
		\end{displaymath}
	Then $f$ satisfies the hypotheses of Theorem \ref{thrm:HWeyl}
	and so
		\begin{IEEEeqnarray*}{rCl}
			b-a & = & \myinta{0}{1}{f(x)}{x} \\
			& = & \lim_{n \rightarrow \infty}
				\frac{1}{n}\sum_{k=1}^n f(<k\gamma>) \\
			& = & \lim_{n \rightarrow \infty}\frac{1}{n}
				\text{card}(\{<k\gamma> \in [a,b]:
				1 \leq k \leq n\})
		\end{IEEEeqnarray*}
\end{proof}
	
%%%%%%%%%%%%%%%%%%%%%%%%%%%%%%%%%%%%%%%%%%%%%%%%
%% Application to Heat Problem
%%%%%%%%%%%%%%%%%%%%%%%%%%%%%%%%%%%%%%%%%%%%%%%%

\begin{thrm}
	Given a continuous function $f$ on $[0,\pi]$ which
	satisfies $f(0)=f(\pi)=0$, there exists a unique
	solution to the heat problem.
\end{thrm}

\begin{proof}
	Extend $f$ to an odd, continuous $2\pi$-periodic function
	on $\real$ and, for $\isreal, 0 \leq t \in \real$, make
	the following definitions.
		\begin{enumerate}[i)]
			\item
				\begin{displaymath}
					b_j = \frac{1}{\pi}\myintb[\pi]
						{f(x)\sin jx}{x}
				\end{displaymath}
			\item
				\begin{displaymath}
					s_k(x) = \sum_{j=1}^k b_j \sin jx
				\end{displaymath}
			\item
				\begin{displaymath}
					u_k(x,t) = \sum_{j=1}^k b_j e^{-j^2t}
						\sin jx
				\end{displaymath}
			\item
				\begin{displaymath}
					\sigma_n(x) = \frac{1}{n}(s_1(x)+s_2(x)+\ldots
						+s_n(x))
				\end{displaymath}
			\item
				\begin{displaymath}
					v_n(x,t) = \frac{1}{n}(u_1(x,t)+u_2(x,t)+\ldots
						+u_n(x,t))
				\end{displaymath}
		\end{enumerate}
	By Theorem \ref{thrm:Fejers},
		\begin{displaymath}
			\lim_{n \rightarrow \infty} \sigma_n(x)
				= \lim_{n \rightarrow \infty} s_n(x) = f(x)
		\end{displaymath}
	and, moreover, this convergence is uniform on $\real$.
	Therefore, given $\epsilon > 0$, there exists $N > 0$
	such that
		\begin{displaymath}
			\modulus{\sigma_n(x)-\sigma_m(x)} < \epsilon
		\end{displaymath}
	whenever $m,n > N$. But $v_n(x,t)$ is the solution to
	the heat problem corresponding to $\sigma_n(x)$. Hence,
	by Theorem \ref{thrm:maxmin},
		\begin{displaymath}
			\modulus{v_n(x,t)-v_m(x,t)} < \epsilon
		\end{displaymath}
	whenever $m,n > N$. So $\{v_n(x,t)\}_{n=1}^\infty$
	satisfies the uniform Cauchy criterion and thus 
	converges uniformly to some $u(x,t):\real \times [0,\infty)
	\rightarrow \real$. We claim that $u(x,t)$ is the solution
	to the heat problem corresponding to $f$. We will divide
	our argument into steps.
		\begin{enumerate}[{Step} 1.]
		
			\item
				First, we have,
					\begin{displaymath}
						u(x,0) = \lim_{n \rightarrow \infty}v_n(x,0)
							= \lim_{n \rightarrow \infty}\sigma_n(x)
							= f(x)
					\end{displaymath}
				by Theorem \ref{thrm:Fejers}.
	
			\item
				Again, because each $v_n$ is the solution to the 
				heat problem corresponding the $\sigma_n$,
					\begin{displaymath}
						u(0,t) = \lim_{n \rightarrow \infty}v_n(0,t)
							= 0
					\end{displaymath}
				and
					\begin{displaymath}
						u(\pi,t) = \lim_{n \rightarrow \infty}v_n(\pi,t) 
							= 0
					\end{displaymath}

			\item
				To see that $u$ is continous on its domain, notice that it is the
				uniform limit of the $v_n$'s, each of which are continuous.
	
			\item
				By Theorem \ref{thrm:HeatSolnIsCInfty}, $u$ is $\ctsdiff$ on its
				domain.
				
			\item
				Finally, because of uniform convergence, we have
					\begin{IEEEeqnarray*}{rCl}
						\frac{\partial u}{\partial t}(x,t) & = &
							\frac{\partial}{\partial t}
							\lim_{n \rightarrow \infty}v_n(x,t) \\
						& = & \lim_{n \rightarrow \infty}
							\frac{\partial v_n}{\partial t}(x,t) \\
						& = & \lim_{n \rightarrow \infty}
							\frac{\partial^2 v_n}{\partial x^2}(x,t) \\
						& = & \frac{\partial^2}{\partial x^2}
							\lim_{n \rightarrow \infty}v_n(x,t) \\
						& = & \frac{\partial^2 u}{\partial x^2}(x,t)
					\end{IEEEeqnarray*}
			
		\end{enumerate}
\end{proof}

\end{section}