\begin{section}{Parseval's Theorem}

	In this section we define orthogonal sets,
	orthonormal sets and orthonormal bases. 
	We also present Bessel's Inequality and 
	Parseval's Theorem.
	
%%%%%%%%%%%%%%%%%%%%%%%%%%%%%%%%%%%%%%%%%%%%%%%%%%%%%%%%%%%
%% Defn of orthogonal/orthonormal sets
%%%%%%%%%%%%%%%%%%%%%%%%%%%%%%%%%%%%%%%%%%%%%%%%%%%%%%%%%%%

\begin{defn}
	Let $V$ be an inner product space and $\{v_1,v_2,
	\ldots,v_n\}$ a subset of $V$. We call this set
	of vectors \emph{orthogonal} if $\innrprod{v_j}{v_k}
	= 0$ whenever $j \neq k$. If in addition,
	$\innrprod{v_j}{v_k} = \delta_{jk}$ then we call this
	set of vectors \emph{othornormal}.
\end{defn}

\begin{prop}
	If $\{v_1,v_2,\ldots,v_n\}$ is an orthogonal set of vectors
	in a complex vector space and $\lambda_1,\lambda_2,\ldots,
	\lambda_n$ are complex scalars, then
		\begin{displaymath}
			\norm[]{\sum_{k=1}^n \lambda_k v_k}^2
				= \sum_{k=1}^n \modulus{\lambda_k}^2 \norm[]{v_k}^2
		\end{displaymath}
\end{prop}

\begin{proof}
	Let $v_k$ and $\lambda_k$ be as in the hypothesis. Then
		\begin{IEEEeqnarray*}{rCl}
			\norm[]{\sum_{k=1}^n \lambda_k v_k}^2 & = &
				\innrprod{\sum_{j=1}^n \lambda_j v_j}
				{\sum_{k=1}^n \lambda_k v_k} \\
			& = & \sum_{j=1}^n \sum_{k=1}^n \innrprod
				{\lambda_j v_j}{\lambda_k v_k} \\
			& = & \sum_{j=1}^n \sum_{k=1}^n \lambda_j \conj{\lambda_k}
				\innrprod{v_j}{v_k} \\
			& = & \sum_{k=1}^n \modulus{\lambda_k}^2 \norm[]{v_k}^2
		\end{IEEEeqnarray*}
\end{proof}
	
%%%%%%%%%%%%%%%%%%%%%%%%%%%%%%%%%%%%%%%%%%%%%%%%%%%%%%%%%%%
%% Bessel's Inequality
%%%%%%%%%%%%%%%%%%%%%%%%%%%%%%%%%%%%%%%%%%%%%%%%%%%%%%%%%%%

\begin{thrm}[Bessel's Inequality]\label{thrm:Bessels}
	Suppose that $\{b_1,\ldots,b_n\}$ is an othonormal subset
	of an inner product space $V$ and $x \in V$. Then
		\begin{enumerate}[i)]
			\item
				For all $\lambda_1,\ldots,\lambda_n \in \cmplx$,
					\begin{displaymath}
						\norm[]{x-\sum_{k=1}^n \innrprod{x}{b_k}b_k}
							\leq \norm[]{x-\sum_{k=1}^n \lambda_k b_k}
					\end{displaymath}
				and equality holds iff $\lambda_k = \innrprod{x}{b_k}$
				for all $k$.
			\item
				(Bessel's Inequality)
					\begin{displaymath}
						\sum_{k=1}^n \modulus{\innrprod{x}{b_k}}^2
							\leq \norm[]{x}^2
					\end{displaymath}
		\end{enumerate}
\end{thrm}

\begin{proof}
	Let $x$, $b_k$ and $\lambda_k$ be as in the hypothesis and define
		\begin{displaymath}
			\varphi(\lambda_1,\ldots,\lambda_n) =
				\norm[]{x-\sum_{k=1}^n \lambda_k b_k}^2
		\end{displaymath}
	Then
		\begin{IEEEeqnarray*}{rCl}
			\varphi(\lambda_1,\ldots,\lambda_n) & = &
				\innrprod{x-\sum_{j=1}^n \lambda_j b_j}
				{x-\sum_{k=1}^n \lambda_k b_k} \\
			& = & \norm[]{x}^2 - \innrprod{x}{\sum_{k=1}^n \lambda_k b_k}
				- \innrprod{\sum_{j=1}^n \lambda_j b_j}{x}
				+ \innrprod{\sum_{j=1}^n \lambda_j b_j}
				{\sum_{k=1}^n \lambda_k b_k} \\
			& = & \norm[]{x}^2 - \sum_{k=1}^n \conj{\lambda_k}
				\innrprod{x}{b_k} - \sum_{j=1}^n \lambda_j
				\innrprod{b_j}{x} + \sum_{k=1}^n \modulus{\lambda_k}^2 \\
			& = & \norm[]{x}^2 + \sum_{k=1}^n \left(\modulus{\innrprod{x}{b_k}}^2
				- \conj{\lambda_k}\innrprod{x}{b_k} - \lambda_k
				\conj{\innrprod{x}{b_k}} + \modulus{\lambda_k}^2\right)
				- \sum_{k=1}^n \modulus{\innrprod{x}{b_k}}^2 \\
			& = & \norm[]{x}^2 + \sum_{k=1}^n \modulus{\innrprod{x}{b_k}-\lambda_k}^2
				- \sum_{k=1}^n \modulus{\innrprod{x}{b_k}}^2
		\end{IEEEeqnarray*}
	Obviously, this quantity is minimized by the unique choice
		\begin{displaymath}
			\lambda_k = \innrprod{x}{b_k}
		\end{displaymath}
	This proves the first assertion. For the second, simply note that
		\begin{displaymath}
			0 \leq \varphi(\innrprod{x}{b_1},\ldots,\innrprod{x}{b_n})
				= \norm[]{x}^2 - \sum_{k=1}^n \modulus{\innrprod{x}{b_k}}^2
		\end{displaymath}
\end{proof}
	
%%%%%%%%%%%%%%%%%%%%%%%%%%%%%%%%%%%%%%%%%%%%%%%%%%%%%%%%%%%
%% Defn of e_k and preliminary version of Fourier Transform
%%%%%%%%%%%%%%%%%%%%%%%%%%%%%%%%%%%%%%%%%%%%%%%%%%%%%%%%%%%

\begin{prop}\label{prop:DefnOfEk}
	In the inner product space $\riemmint$ (with the inner
	product defined in the previous section), let
		\begin{displaymath}
			e_k(t) = \frac{e^{ikt}}{\sqrt{2\pi}}
		\end{displaymath}
	for $\isintgr[k]$. Also, for $f \in \riemmint$ define
		\begin{displaymath}
			\f(k) = \innerprod{f}{e_k}
		\end{displaymath}
	(this is sometimes called the \emph{Fourier Transform of $f$}
	but we will reserve this terminology for a different function).
	Then
		\begin{enumerate}[i)]
			\item
				The set $\{e_k\}_{k=-\infty}^\infty$ is orthonormal.
			\item
				$\f(k) = \sqrt{2\pi}c_k$ where $c_k, \isintgr[k]$ are
				the exponential Fourier coefficients of $f$.
			\item
				If $s_n, n \geq 1$ is the $n$th partial sum of the
				Fourier series of $f$, then
					\begin{displaymath}
						s_n = \sum_{k=-n}^n \f(k)e_k
					\end{displaymath}
			\item
				The series $\sum_{k=0}^\infty \modulus{\f(k)}^2$ and
				$\sum_{k=1}^\infty \modulus{\f(-k)}^2$ both converge.
				Also, if we define
					\begin{displaymath}
						\sum_{k=-\infty}^\infty \modulus{\f(k)}^2
							= \sum_{k=0}^\infty \modulus{\f(k)}^2
							+ \sum_{k=1}^\infty \modulus{\f(-k)}^2
					\end{displaymath}
				then this series converges as well.
			\item
				If $f \in \riemmint$ then
					\begin{displaymath}
						\sum_{k=1}^\infty \modulus{\frac{\f(k)}{k}}
					\end{displaymath}
				and
					\begin{displaymath}
						\sum_{k=1}^\infty \modulus{\frac{\f(-k)}{k}}
					\end{displaymath}
				both converge.
		\end{enumerate}
\end{prop}

\begin{proof}
	Let $f$, $\f$ and the $e_k$ be as in the hypothesis.
		\begin{enumerate}[i)]
		
			\item
				By Lemma \ref{lemma:trigidentities} we have
					\begin{IEEEeqnarray*}{rCl}
						\innerprod{e_j(t)}{e_k(t)} & = & \myintb[\pi]
							{\frac{e^{ijt}}{\sqrt{2\pi}}
							\conj{\frac{e^{ikt}}{\sqrt{2\pi}}}}{t} \\
						& = & \frac{1}{2\pi}\myintb[\pi]{e^{i(j-k)t}}{t} \\
						& = & \delta_{jk}
					\end{IEEEeqnarray*}
			
			\item
				\begin{displaymath}
					\f(k) = \innerprod{f}{e_k} = \frac{1}{\sqrt{2\pi}}
						\myintb[\pi]{f(t)e^{-ikt}}{t} = \sqrt{2\pi}c_k
				\end{displaymath}
				
			\item
				For $\isnatrl$ and $\isreal$,
					\begin{IEEEeqnarray*}{rCl}
						s_n(x) & = & \sum_{k=-n}^n \left( \frac{1}{2\pi}
							\myintb[\pi]{f(t)e^{-ikt}}{t} \right)
							e^{ikx} \\
						& = & \sum_{k=-n}^n \left( \myintb[\pi]{f(t)\frac
							{e^{-ikt}}{\sqrt{2\pi}}}{t} \right) \frac
							{e^{ikx}}{\sqrt{2\pi}} \\
						& = & \sum_{k=-n}^n \innerprod{f}{e_k}e_k(x) \\
						& = & \sum_{k=-n}^n \f(k)e_k(x)
					\end{IEEEeqnarray*}
					
			\item
				By Theorem \ref{thrm:Bessels}, for all $\isnatrl$,
					\begin{displaymath}
						\sum_{k=-n}^n \modulus{\f(k)}^2
							= \sum_{k=-n}^n \modulus{\innerprod{f}{e_k}}^2
							\leq \norm[]{f}^2 = \myintb[\pi]{\modulus{f(t)}^2}{t}
							< \infty
					\end{displaymath}
				since $f \in \riemmint$. Therefore $\sum_{k=0}^\infty \modulus{\f(k)}^2$
				and $\sum_{k=1}^\infty \modulus{\f(-k)}^2$ are both convergent and
					\begin{displaymath}
						\sum_{k=-\infty}^\infty \modulus{\f(k)}^2 = 
							\sum_{k=0}^\infty \modulus{\f(k)}^2 +
							\sum_{k=1}^\infty \modulus{\f(-k)}^2
							\leq \myintb[\pi]{\modulus{f(t)}^2}{t}
							< \infty
					\end{displaymath}
			
			\item
				Let
					\begin{displaymath}
						S = \{\isnatrl[k] \; | \; \modulus{\f(k)} \leq \frac{1}{k} \}
					\end{displaymath}
				and
					\begin{displaymath}
						T = \{\isnatrl[k] \; | \; \modulus{\f(k)} > \frac{1}{k} \}
					\end{displaymath}
				Then
					\begin{IEEEeqnarray*}{rCl}
						\sum_{k=1}^\infty \modulus{\frac{\f(k)}{k}} & = &
							\sum_{{\substack {k \in S}}} \modulus{\frac{\f(k)}{k}}
							+ \sum_{{\substack {k \in T}}} \modulus{\frac{\f(k)}{k}} \\
						& < & \sum_{{\substack {k \in S}}} \frac{1}{k^2}
							+ \sum_{{\substack {k \in T}}} \modulus{\f(k)}^2 \\
						& < & \sum_{k=1}^\infty \frac{1}{k^2} + \sum_{k=-\infty}^\infty
							\modulus{\innerprod{f}{e_k}}^2
					\end{IEEEeqnarray*}
				Now the first sum in the above expression converges and by Theorem
				\ref{thrm:Bessels},
					\begin{displaymath}
						\sum_{k=-\infty}^\infty \modulus{\innerprod{f}{e_k}}^2
							\leq \norm[]{f}^2 < \infty
					\end{displaymath}
				since $f \in \riemmint$. Therefore
					\begin{displaymath}
						\sum_{k=1}^\infty \modulus{\frac{\f(k)}{k}} < \infty
					\end{displaymath}
				A similar argument shows that
					\begin{displaymath}
						\sum_{k=1}^\infty \modulus{\frac{\f(-k)}{k}} < \infty
					\end{displaymath}
				as well.
				
		\end{enumerate}
\end{proof}

%%%%%%%%%%%%%%%%%%%%%%%%%%%%%%%%%%%%%%%%%%%%%%%%%%%%%%%%%%%
%% Parseval's Theorem
%%%%%%%%%%%%%%%%%%%%%%%%%%%%%%%%%%%%%%%%%%%%%%%%%%%%%%%%%%%

\begin{thrm}[Parseval's Theorem]\label{thrm:Parsevals}
	If $f \in \riemmint$ then
		\begin{displaymath}
			\lim_{n \rightarrow \infty} \norm[]
				{f-\sum_{k=-n}^n \f(k)e_k} = 0
		\end{displaymath}
	and
		\begin{displaymath}
			\sum_{k=-\infty}^\infty \modulus{\f(k)}^2
				= \norm[]{f}^2 = \myintb[\pi]
				{\modulus{f(t)}^2}{t}
		\end{displaymath}
\end{thrm}

\begin{proof}
	\begin{enumerate}[{Case} i)]
	
		%% Case 1: f is cts
		\item
			First suppose $f \in \cts_{2\pi}$. For $n \geq 0$,
			let $s_n$ be the $n$th partial sum of the Fourier
			series of $f$ and define
				\begin{displaymath}
					\sigma_n = \frac{1}{n+1}\sum_{k=0}^n s_k
				\end{displaymath}
			By Theorem \ref{thrm:Fejers}, given $\epsilon > 0$
			there exists $N > 0$ such that for $n > N$,
			$\modulus{f(t)-\sigma_n(t)} < \sqrt{\epsilon/{2\pi}}$.
			Hence for this choice of $n$,
				\begin{displaymath}
					\norm[]{f-\sigma_n}^2 = \myintb[\pi]
						{\modulus{f(t)-\sigma_n(t)}^2}{t}
						< \myintb[\pi]{\frac{\epsilon}{2\pi}}{t}
						= \epsilon
				\end{displaymath}
			But by Proposition \ref{prop:DefnOfEk},
				\begin{displaymath}
					s_n = \sum_{k=-n}^n \f(k)e_k
				\end{displaymath}
			which means $\sigma_n$ is a linear combination of
			the $e_k$'s. Therefore, by Theorem \ref{thrm:Bessels},
				\begin{IEEEeqnarray*}{rCl}
					\norm[]{f-s_n}^2 & = &
						\norm[]{f-\sum_{k=-n}^n \innerprod{f}{e_k}e_k}^2 \\
					& \leq & \norm[]{f-\sigma_n}^2 \\
					& < & \epsilon
				\end{IEEEeqnarray*}
			if $n > N$. Thus $\lim_{n \rightarrow \infty} \norm[]{f-s_n}=0$.
			The second assertion follows by noting that $\modulus
			{\norm[]{f}-\norm[]{s_n}} \leq \norm[]{f-s_n} \rightarrow 0$
			as $n \rightarrow \infty$.
			
		%% Case 2: f is in R2pi
		\item
			Suppose $f \in \riemmint$ and $\epsilon > 0$. Choose $g \in
			\cts_{2\pi}$ such that $\norm[]{f-g} < \epsilon/2$. Also,
			by the first case, we can choose $N > 0$ such that
				\begin{displaymath}
					\norm[]{g-\sum_{k=-N}^N \f[g](k)e_k}
						< \epsilon/2
				\end{displaymath}
			Then if $n > N$, by Theorem \ref{thrm:Bessels}, we have
				\begin{IEEEeqnarray*}{rCl}
					\norm[]{f-\sum_{k=-n}^n \f(k)e_k}
						& \leq & \norm[]{f-\sum_{k=-N}^N
						\f[g](k)e_k} \\
					& \leq & \norm[]{f-g}
						+ \norm[]{g-\sum_{k=-N}^N \f[g](k)e_k} \\
					& < & \epsilon
				\end{IEEEeqnarray*}
			Thus
				\begin{displaymath}
					\lim_{n \rightarrow \infty} \norm[]
						{f-\sum_{k=-n}^n \f(k)e_k} = 0
				\end{displaymath}
			The second assertion follows as in Case i).
	
	\end{enumerate}
\end{proof}

%%%%%%%%%%%%%%%%%%%%%%%%%%%%%%%%%%%%%%%%%%%%%%%%%%%%%%%%%%%
%% Defn of orthonormal basis
%%%%%%%%%%%%%%%%%%%%%%%%%%%%%%%%%%%%%%%%%%%%%%%%%%%%%%%%%%%

\begin{defn}
	An orthonormal sequence $\{b_k\}_{k=1}^\infty$ in an
	inner product space $V$ is said to be an \emph{orthonormal
	basis for $V$} provided that for all $x \in V$,
		\begin{displaymath}
			x = \lim_{n \rightarrow \infty} \sum_{k=1}^n
				\innrprod{x}{b_k}b_k
		\end{displaymath}
	In other words,
		\begin{displaymath}
			\lim_{n \rightarrow \infty} \norm[]
				{x-\sum_{k=1}^n \innrprod{x}{b_k}b_k}
				= 0
		\end{displaymath}
\end{defn}

\begin{ex}
	By Theorem \ref{thrm:Parsevals}, $\{e_0,e_{-1},e_1,\ldots\}$
	is an orthonormal basis for $\riemmint$.
\end{ex}

%%%%%%%%%%%%%%%%%%%%%%%%%%%%%%%%%%%%%%%%%%%%%%%%%%%%%%%%%%%
%% Theorem of orthonormal basis
%%%%%%%%%%%%%%%%%%%%%%%%%%%%%%%%%%%%%%%%%%%%%%%%%%%%%%%%%%%

\begin{thrm}\label{thrm:OrthoBasis}
	If $\{b_k\}_{k=1}^\infty$ is an orthonormal basis for
	an inner product space $V$, then for all $x,y \in V$
	and $\iscmplx[\lambda]$
		\begin{enumerate}[i)]
			\item
				\begin{displaymath}
					\lambda x + y = \sum_{k=1}^\infty \left(
						\lambda \innrprod{x}{b_k} + \innrprod{y}{b_k}
						\right) b_k
				\end{displaymath}
			\item
				\begin{displaymath}
					\innrprod{x}{y} = \sum_{k=1}^\infty \innrprod{x}{b_k}
						\conj{\innrprod{y}{b_k}}
				\end{displaymath}
			\item
				\begin{displaymath}
					\norm[]{x}^2 = \sum_{k=1}^\infty \modulus{\innrprod{x}{b_k}}^2
				\end{displaymath}
		\end{enumerate}
\end{thrm}

\begin{proof}
	\begin{enumerate}[i)]
	
		\item
			Trivial.
			
		\item
			Given $\epsilon > 0$ choose $N > 0$ such that if $n > N$
			then
				\begin{displaymath}
					\norm[]{x-\sum_{k=1}^n \innrprod{x}{b_k}b_k}
						< \frac{\epsilon}{2(\norm[]{y}+1)}
				\end{displaymath}
			and also
				\begin{displaymath}
					\norm[]{y-\sum_{k=1}^n \innrprod{y}{b_k}b_k}
						< \frac{\epsilon}{2(\norm[]{x}+1)}
				\end{displaymath}
			(This can be done because $\{b_k\}_{k=1}^\infty$ is a basis
			for $V$.) Then for this choice of $n$,
				\begin{IEEEeqnarray*}{rCl}
					\modulus{\innrprod{x}{y}-\sum_{k=1}^n \innrprod{x}{b_k}
						\conj{\innrprod{y}{b_k}}} & = & \modulus{\innrprod{x}{y}
						- \innrprod{\sum_{k=1}^n \innrprod{x}{b_k}b_k}
						{\sum_{k=1}^n \innrprod{y}{b_k}b_k}} \\
%%					& = & \modulus{\innrprod{x}{y} - 
%%						\innrprod{\sum_{k=1}^n \innrprod{x}{b_k}b_k}{y} +
%%						\innrprod{\sum_{k=1}^n \innrprod{x}{b_k}b_k}{y} \\
%%					& & \; - \innrprod{\sum_{k=1}^n \innrprod{x}{b_k}b_k}
%%						{\sum_{k=1}^n \innrprod{y}{b_k}b_k}} \\
					& \leq & \modulus{\innrprod{x}{y} - 
						\innrprod{\sum_{k=1}^n \innrprod{x}{b_k}b_k}{y}} \\
					& &	+ \; \modulus{\innrprod{\sum_{k=1}^n \innrprod{x}{b_k}b_k}{y}
						- \innrprod{\sum_{k=1}^n \innrprod{x}{b_k}b_k}
						{\sum_{k=1}^n \innrprod{y}{b_k}b_k}} \\
					& = & \modulus{\innrprod{x-\sum_{k=1}^n \innrprod{x}{b_k}b_k}{y}} \\
					& & + \; \modulus{\innrprod{\sum_{k=1}^n \innrprod{x}{b_k}b_k}
						{y-\sum_{k=1}^n \innrprod{y}{b_k}b_k}} \\
					& \leq & \norm[]{x-\sum_{k=1}^n \innrprod{x}{b_k}b_k}\norm[]{y} \\
					& & + \; \norm[]{\sum_{k=1}^n \innrprod{x}{b_k}b_k}\norm[]
						{y-\sum_{k=1}^n \innrprod{y}{b_k}b_k} \; \text{by Proposition } \ref{prop:CauchySchwarz} \\
					& < & \frac{\epsilon}{2(\norm[]{y}+1)}\norm[]{y}
						+ \norm[]{x}\frac{\epsilon}{2(\norm[]{x}+1)} 
						\; \text{by Theorem } \ref{thrm:Bessels} \\
					& < & \frac{\epsilon}{2} + \frac{\epsilon}{2} \\
					& = & \epsilon
				\end{IEEEeqnarray*}
			which proves the result.
		
		\item
			This follows immediately from part ii).
	
	\end{enumerate}
\end{proof}

\begin{cor}\label{cor:OrthoBasis}
	If $f,g \in \riemmint$ and $\iscmplx[\lambda]$ then
		\begin{enumerate}[i)]
			\item
				$\F[(\lambda f + g)](k) = \lambda (\f(k)+\f[g](k))
				= (\lambda \f + \f[g])(k)$ for all $\isintgr[k]$.
			\item
				\begin{displaymath}
					\myintb[\pi]{f(t)\conj{g(t)}}{t}
						= \sum_{k=-\infty}^\infty \f(k)\conj{\f[g](k)}
				\end{displaymath}
			\item
				\begin{displaymath}
					\myintb[\pi]{\modulus{f(t)}^2}{t}
						= \sum_{k=-\infty}^\infty \modulus{\f(k)}^2
				\end{displaymath}
		\end{enumerate}
\end{cor}

\begin{proof}
	All parts of this corollary follow from Theorem \ref{thrm:OrthoBasis}.
\end{proof}

\end{section}