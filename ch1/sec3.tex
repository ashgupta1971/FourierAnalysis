\begin{section}{Riemann-Lebesgue Lemma}

The Riemann-Lebesgue Lemma is used extensively in this
book and will be proven in this section.

\begin{thrm}[Riemann-Lebesgue Lemma]\label{thrm:RLL}
	Suppose $f:[a,b] \to \cmplx$ is Riemann Integrable on its domain.
	Then
	\begin{displaymath}
	\lim_{\substack{\lambda \rightarrow \infty \\ \lambda \in \real}}
	\myinta{a}{b}{f(x)\sin(\lambda x + \alpha)}{x}
	= 0
	\end{displaymath}
\end{thrm}
	
\begin{proof}
	\begin{enumerate}[{Case} 1.]
%%%%%%%%%%%%%%%%%%%%%%%%%%%%%%%%%%%%%%%%%
%% f is constant
%%%%%%%%%%%%%%%%%%%%%%%%%%%%%%%%%%%%%%%%%
	\item
	$f = c$ on $[a,b]$ for some $c \in \real$. Then for all $\lambda \in \real$,
	$\lambda \neq 0$,
	\begin{IEEEeqnarray*}{l}
	\myinta{a}{b}{f(x)\sin(\lambda x + \alpha)}{x} \\
	= c \myinta{a}{b}{\sin(\lambda x + \alpha)}{x} \\
	= \evalat{\frac{-c \cos(\lambda x + \alpha)}{\lambda}}{x}{a}{b} \\
	= \frac{c}{\lambda} \left(\cos(\lambda a + \alpha) - \cos(\lambda b + \alpha) \right)
	\end{IEEEeqnarray*}
	
	So,
	\begin{displaymath}
	\modulus{\myinta{a}{b}{f(x)\sin(\lambda x + \alpha)}{x}} \leq
	\frac{2\modulus{c}}{\modulus{\lambda}} \rightarrow 0
	\end{displaymath}
	as $\lambda \rightarrow \infty$.
%%%%%%%%%%%%%%%%%%%%%%%%%%%%%%%%%%%%%%%%%
%% f is a step function
%%%%%%%%%%%%%%%%%%%%%%%%%%%%%%%%%%%%%%%%%	
	\item
	$f$ is a step function. Then there exist $x_0, x_1, \ldots, x_n$ and
	$c_1, c_2, \ldots, c_n$ such that $a = x_0 < x_1 < \ldots < x_n = b$ and
	$f(x) = c_k$ for $x_{k-1} < c_k \leq x_k$, $1 \leq k \leq n$. In this case
	we have
	\begin{displaymath}
	\myinta{a}{b}{f(x)\sin(\lambda x + \alpha)}{x}
	= \sum_{k=1}^n \myinta{x_{k-1}}{x_k}{c_k\sin(\lambda x + \alpha)}{x}
	\end{displaymath}
	$\rightarrow 0$ as $\lambda \rightarrow \infty$ by case 1.
%%%%%%%%%%%%%%%%%%%%%%%%%%%%%%%%%%%%%%%%%
%% f is real-valued
%%%%%%%%%%%%%%%%%%%%%%%%%%%%%%%%%%%%%%%%%	
	\item
	$f$ is real-valued. Fix $\epsilon > 0$. Since $f$ is Riemann Integrable,
	there exists a step function $g:[a,b] \to \real$ such that
	\begin{displaymath}
	\myinta{a}{b}{\modulus{f(x) - g(x)}}{x} < \epsilon/2
	\end{displaymath}
	Then
	\begin{IEEEeqnarray*}{l}
	\modulus{\myinta{a}{b}{f(x)\sin(\lambda x + \alpha)}{x}
	- \myinta{a}{b}{g(x)\sin(\lambda x + \alpha)}{x}} \\
	= \modulus{\myinta{a}{b}{(f(x)-g(x))\sin(\lambda x + \alpha)}{x}} \\
	\leq \myinta{a}{b}{\modulus{f(x)-g(x)}}{x} \\
	< \epsilon/2
	\end{IEEEeqnarray*}
	
	Therefore,
	\begin{IEEEeqnarray*}{l}
	\limsup_{\lambda \rightarrow \infty}
	\modulus{\myinta{a}{b}{f(x)\sin(\lambda x + \alpha)}{x}} \\
	\leq \limsup_{\lambda \rightarrow \infty}
	\modulus{\myinta{a}{b}{(f(x)- g(x))\sin(\lambda x + \alpha)}{x}} \\
	+ \; \limsup_{\lambda \rightarrow \infty}
	\modulus{\myinta{a}{b}{g(x)\sin(\lambda x + \alpha)}{x}} \\
	< \epsilon/2 + \limsup_{\lambda \rightarrow \infty}
	\modulus{\myinta{a}{b}{g(x)\sin(\lambda x + \alpha)}{x}} \\
	< \epsilon
	\end{IEEEeqnarray*}
	(by Case 2 and the fact that $g$ is a step function).
	
	Since this is true for all $\epsilon > 0$,
	\begin{displaymath}
	\limsup_{\lambda \rightarrow \infty}
	\modulus{\myinta{a}{b}{f(x)\sin(\lambda x + \alpha)}{x}} = 0
	\end{displaymath}
	and the result follows.
%%%%%%%%%%%%%%%%%%%%%%%%%%%%%%%%%%%%%%%%%
%% General case of RL Lemma
%%%%%%%%%%%%%%%%%%%%%%%%%%%%%%%%%%%%%%%%%	
	\item
	For the general case in which $f$ is complex-valued, simply apply
	the previous case to the real and imaginary parts of $f$.
	\end{enumerate}
\end{proof}

%%%%%%%%%%%%%%%%%%%%%%%%%%%%%%%%%%%%%%%%%%%%%%%%%%%%%%%%%%%%
%% Corollary to Riemann-Lebesgue Lemma
%%%%%%%%%%%%%%%%%%%%%%%%%%%%%%%%%%%%%%%%%%%%%%%%%%%%%%%%%%%%

\begin{cor}\label{cor:RLL}
	Let $f:[a,b] \to \cmplx$ be Riemann Integrable on its domain. Then
	\begin{enumerate}[i)]
	\item
	\begin{IEEEeqnarray*}{l}
	\myinta{a}{b}{f(x)\sin \lambda x}{x} \rightarrow 0
	\end{IEEEeqnarray*}
	as $\lambda \rightarrow \infty$ or $\lambda \rightarrow -\infty$.
	
	\item
	\begin{IEEEeqnarray*}{l}
	\myinta{a}{b}{f(x)\cos \lambda x}{x} \rightarrow 0
	\end{IEEEeqnarray*}
	as $\lambda \rightarrow \infty$ or $\lambda \rightarrow -\infty$.
	
	\item
	\begin{IEEEeqnarray*}{l}
	\myinta{a}{b}{f(x)e^{i\lambda x}}{x} \rightarrow 0
	\end{IEEEeqnarray*}
	as $\lambda \rightarrow \infty$ or $\lambda \rightarrow -\infty$.
	\end{enumerate}
\end{cor}

\begin{proof}
	\begin{enumerate}[i)]
	\item
	Apply the Riemann-Lebesgue Lemma with $\alpha = 0$.
	
	\item
	Apply the Riemann-Lebesgue Lemma with $\alpha = \pi/2$.
	
	\item
	This is a simple consequence of parts i) and ii) and
	the fact that $e^{i\lambda x} = \cos \lambda x + i \sin \lambda x$
	for all $\lambda, x \in \real$.
	\end{enumerate}
\end{proof} 
\end{section}
