\begin{section}{The Schwarz Space}

	Here we present the Schwarz Space and prove
	several relevant results.
	
%%%%%%%%%%%%%%%%%%%%%%%%%%%%%%%%%%%%%%%%%%%%%%%%%%%%%
%% Defn of Schwarz Space
%%%%%%%%%%%%%%%%%%%%%%%%%%%%%%%%%%%%%%%%%%%%%%%%%%%%%

\begin{defn}
	Let $\schwarz$, the \emph{Schwarz Space}, be the set of all
	$\ctsdiff$ functions from $\real$ to $\cmplx$ having the
	property that for every $k,m \in \intgr, k \geq 0,
	m \geq 0$,
		\begin{displaymath}
			\lim_{\modulus{x} \rightarrow \infty}
				\modulus{x^m f^{(k)}(x)} = 0
		\end{displaymath}
\end{defn}
	
%%%%%%%%%%%%%%%%%%%%%%%%%%%%%%%%%%%%%%%%%%%%%%%%%%%%%
%% First simple propostion
%%%%%%%%%%%%%%%%%%%%%%%%%%%%%%%%%%%%%%%%%%%%%%%%%%%%%

\begin{prop}
	\begin{enumerate}[i)]
		\item
			If $G$ is the Gaussian Distribution then $G \in \schwarz$.
		\item
			$\schwarz \subset \ctsdiff(\real) \cap \ctszeror$.
		\item
			$\schwarz$ is a complex, commutative algebra (lacking
			identity) with respect to pointwise addition and
			multiplication.
		\item
			If, for $f \in \schwarz$, we define $f^-:\real \rightarrow 
			\cmplx$ by $f^-(x) = f(-x)$ then $f^- \in \schwarz$.
		\item
			If, for $f \in \schwarz$, we define $\conj{f}:\real \rightarrow
			\cmplx$ by $\conj{f}(x)=\conj{f(x)}$ then $\conj{f} \in \schwarz$.
	\end{enumerate}
\end{prop}

\begin{proof}
	The proof of all parts is trivial.
\end{proof}

%%%%%%%%%%%%%%%%%%%%%%%%%%%%%%%%%%%%%%%%%%%%%%%%%%%%%
%% First lemma
%%%%%%%%%%%%%%%%%%%%%%%%%%%%%%%%%%%%%%%%%%%%%%%%%%%%%

\begin{lemma}\label{lemma:S1}
	If $f \in \schwarz$ then
		\begin{enumerate}[i)]
			\item
				$f' \in \schwarz$.
			\item
				$Pf \in \schwarz$ where $P:\real \rightarrow \cmplx$
				is any polynomial.
			\item
				$f \in \absint$.
		\end{enumerate}
\end{lemma}

\begin{proof}
	The proof of i) and ii) is trivial. For iii), if $f \in \schwarz$
	then $\modulus{(1+x^2)f(x)} \leq \modulus{f(x)}+\modulus{x^2 f(x)}$
	for all $\isreal$. Hence
		\begin{displaymath}
			\lim_{\modulus{x} \rightarrow \infty} \modulus{(1+x^2)f(x)}=0
		\end{displaymath}
	Choose $C>0$ such that $\modulus{(1+x^2)f(x)}\leq C$ for all $\isreal$.
	Then $\modulus{f(x)} \leq C/(1+x^2)$ which implies $f \in \absint$.
\end{proof}

%%%%%%%%%%%%%%%%%%%%%%%%%%%%%%%%%%%%%%%%%%%%%%%%%%%%%
%% First theorem
%%%%%%%%%%%%%%%%%%%%%%%%%%%%%%%%%%%%%%%%%%%%%%%%%%%%%

\begin{thrm}\label{thrm:S1}
	If $f \in \schwarz$ then so is $\f$.
\end{thrm}

\begin{proof}
	Let $f \in \schwarz$. Since $f \in \absint$ by Lemma \ref{lemma:S1},
	$\f \in \ctszeror$ and by definition,
		\begin{displaymath}
			\f(x)=\myintb{f(t)e^{-ixt}}{t}
		\end{displaymath}
	for all $\isreal$. By Theorem \ref{thrm:Leibniz2},
		\begin{displaymath}
			\f'(x)= -i \myintb{tf(t)e^{-ixt}}{t}
		\end{displaymath}
	for all $\isreal$. Note that the map $t \mapsto tf(t)$ belongs
	to $\schwarz$ and hence to $\absint \cap \ctsdiff(\real)$. By
	induction, $\f \in \ctsdiff(\real)$ and
		\begin{displaymath}
			\f^{(k)}(x)= (-i)^k \myintb{t^k f(t)e^{-ixt}}{t}
		\end{displaymath}
	for all $\isreal, \isnatrl[k]$. Now suppose $\isnatrl[k]$. Then
	for each $\isreal$,
		\begin{IEEEeqnarray*}{rCl}
			x\f^{(k)}(x) & = & x (-i)^k \myintb{t^k f(t)e^{-ixt}}{t} \\
			& = & (-i)^{k-1} \myintb{t^k f(t) \frac{d}{dt}(e^{-ixt})}{t} \\
			& = & (-i)^{k-1}\evalat{t^k f(t)e^{-ixt}}{t}{-\infty}{\infty}
				- (-i)^{k-1}\myintb{(kt^{k-1}f(t)+t^k f'(t))e^{-ixt}}{t} \\
			& = & \f[g](x)
		\end{IEEEeqnarray*}
	where $g$ is that member of $\schwarz$ defined by
		\begin{displaymath}
			g(t) = -(-i)^{k-1}(kt^{k-1}f(t)+t^k f'(t))
		\end{displaymath}
	We know that $\f[g] \in \ctszeror$ so 
		\begin{displaymath}
			\lim_{\modulus{x} \rightarrow \infty} x \f^{(k)}(x)
				= \lim_{\modulus{x} \rightarrow \infty} \f[g] = 0
		\end{displaymath}
	By induction, for all $m,k \geq 0$,
		\begin{displaymath}
			\lim_{\modulus{x} \rightarrow \infty} 
				\modulus{x^m\f^{(k)}(x)}=0
		\end{displaymath}
	so that $\f \in \schwarz$.
\end{proof}

%%%%%%%%%%%%%%%%%%%%%%%%%%%%%%%%%%%%%%%%%%%%%%%%%%%%%
%% Defn of M, D and F
%%%%%%%%%%%%%%%%%%%%%%%%%%%%%%%%%%%%%%%%%%%%%%%%%%%%%

\begin{defn}
	For $f \in \schwarz$ define $Mf(x)=-ixf(x)$ for $\isreal$.
	Note that $M$ is a linear operator on $\schwarz$ and for all
	$\isnatrl[k]$, $M^k f(x)=(-ix)^k f(x)$. Also, define $Df=f'$
	and again note that this is a linear operator on $\schwarz$.
	Finally, let
		\begin{displaymath}
			Ff = \frac{1}{\sqrt{2\pi}}\f
		\end{displaymath}
	(which is also a linear operator on $\schwarz$ and is called
	the Fourier Transform of $f$ in some texts). (Note that by
	Example \ref{ex:GaussianDist}, $FG=G$	where $G$ is the Gaussian
	Distribution.)
\end{defn}

%%%%%%%%%%%%%%%%%%%%%%%%%%%%%%%%%%%%%%%%%%%%%%%%%%%%%
%% Second lemma -- involving M,D,F
%%%%%%%%%%%%%%%%%%%%%%%%%%%%%%%%%%%%%%%%%%%%%%%%%%%%%

\begin{lemma}\label{lemma:S2}
	If $f \in \schwarz$ then
		\begin{enumerate}[i)]
			\item
				$DFf = FMf$
			\item
				$FD = -MFf$
		\end{enumerate}
\end{lemma}

\begin{proof}
	i) follows from the proof of Theorem \ref{thrm:S1}. For ii),
		\begin{IEEEeqnarray*}{rCl}
			FDf(x) & = & \frac{1}{\sqrt{2\pi}}\myintb{f'(t)e^{-ixt}}{t} \\
			& = & \evalat{\frac{1}{\sqrt{2\pi}}f(t)e^{-ixt}}{t}{-\infty}
				{\infty} - (-ix)\frac{1}{\sqrt{2\pi}}\myintb{f(t)e^{-ixt}}{t} \\
			& = & -(-ix)\frac{1}{\sqrt{2\pi}}\f(x) \\
			& = & -MFf(x)
		\end{IEEEeqnarray*}
\end{proof}

%%%%%%%%%%%%%%%%%%%%%%%%%%%%%%%%%%%%%%%%%%%%%%%%%%%%%
%% Corollary to Second lemma -- involving M,D,F
%%%%%%%%%%%%%%%%%%%%%%%%%%%%%%%%%%%%%%%%%%%%%%%%%%%%%

\begin{cor}\label{cor:S2}
	If $f \in \schwarz$ and $\isnatrl[k]$ then
		\begin{enumerate}[i)]
			\item
				$D^k F f = F M^k f$
			\item
				$F D^k f = (-1)^k M^k F f$
		\end{enumerate}
\end{cor}

\begin{proof}
	The proof of both assertions follows from Lemma \ref{lemma:S2}
	and induction.
\end{proof}

%%%%%%%%%%%%%%%%%%%%%%%%%%%%%%%%%%%%%%%%%%%%%%%%%%%%%%%%
%% Theorem 3 - F is an automorphism on S
%%%%%%%%%%%%%%%%%%%%%%%%%%%%%%%%%%%%%%%%%%%%%%%%%%%%%%%%

\begin{thrm}\label{thrm:FisAutomorphism}
	The operator $F$ is a vector space automorphism of $\schwarz$
	onto itself and $F^{-1}f=Ff^-$ for all $f \in \schwarz$.
\end{thrm}

\begin{proof}
	This follows from Corollary \ref{cor:FIT2}.
\end{proof}

%%%%%%%%%%%%%%%%%%%%%%%%%%%%%%%%%%%%%%%%%%%%%%%%%%%%%%%%
%% Prop. involving Schwarz Space and Convolutions
%%%%%%%%%%%%%%%%%%%%%%%%%%%%%%%%%%%%%%%%%%%%%%%%%%%%%%%%

\begin{prop}
	If $f,g \in \schwarz$ then $\conva{f}{g} \in \schwarz$
	and
		\begin{displaymath}
			\convb{f}{g}{x} = \myintb{f(s)}{s}\myintb{g(t)}{t}
		\end{displaymath}
\end{prop}

\begin{proof}
	It follows from Theorem \ref{thrm:Fubini3} that $\conva{f}{g}
	\in \ctsdiff(\real)$ and from Theorem \ref{thrm:Leibniz2} that
		\begin{displaymath}
			D^k (\conva{f}{g}) = \conva{f}{(D^k g)}
		\end{displaymath}
	for all $\isnatrl[k]$. Now suppose $\isnatrl[m]$. Then for all
	$\isreal, \isreal[t]$,
		\begin{displaymath}
			\modulus{x}^m \leq (\modulus{x-t}+\modulus{t})^m
				= \sum_{k=0}^m {m \choose k}\modulus{x-t}^{m-k}
				\modulus{t}^k
		\end{displaymath}
	Therefore,
		\begin{IEEEeqnarray*}{rCl}
			\modulus{x^m \convb{f}{g}{x}} & \leq &
				\sum_{k=0}^m {m \choose k} \myintb
				{\modulus{t}^k \modulus{f(t)}
				\modulus{x-t}^{m-k} \modulus{g(x-t)}}{t} \\
			& \leq & \sum_{k=0}^m {m \choose k} \myintb
				{\modulus{t}^k \modulus{f(t)} \supnorm{M^{m-k} g}}{t} \\
			& = & C_m
		\end{IEEEeqnarray*}
	So we have
		\begin{displaymath}
			\modulus{x^{m-1} \convb{f}{g}{x}} \leq \frac{C_m}{\modulus{x}}
		\end{displaymath}
	for $0 \neq x \in \real$ and hence,
		\begin{displaymath}
			\lim_{\modulus{x} \rightarrow \infty} \modulus{x^{m-1}
			\convb{f}{g}{x}} = 0
		\end{displaymath}
	Thus for all $0 \leq m,k \in \natrl$,
		\begin{displaymath}
			\lim_{\modulus{x} \rightarrow \infty} \modulus{x^m
			(\convb{f}{g}{x})^{(k)}} = \lim_{\modulus{x} \rightarrow \infty}
			\modulus{x^m \convb{f}{g^{(k)}}{x}} = 0
		\end{displaymath}
	Finally, the formula for $\convb{f}{g}{x}$ was derived in the proof of
	Theorem \ref{thrm:ConvForFourierTransform}.
\end{proof}

%%%%%%%%%%%%%%%%%%%%%%%%%%%%%%%%%%%%%%%%%%%%%%%%%%%%%%%%
%% Defn of "circled asterisk"
%%%%%%%%%%%%%%%%%%%%%%%%%%%%%%%%%%%%%%%%%%%%%%%%%%%%%%%%

\begin{defn}
	For the sake of elegance in certain formulae, we define
		\begin{displaymath}
			\convc{f}{g} = \frac{1}{\sqrt{2\pi}}(\conva{f}{g})
		\end{displaymath}
\end{defn}

%%%%%%%%%%%%%%%%%%%%%%%%%%%%%%%%%%%%%%%%%%%%%%%%%%%%%%%%
%% New version of conv. theorem for the Fourier Trans.
%%%%%%%%%%%%%%%%%%%%%%%%%%%%%%%%%%%%%%%%%%%%%%%%%%%%%%%%

\begin{thrm}\label{thrm:ConvForFourierTransform2}
	If $f,g \in \schwarz$ then $F(\convc{f}{g})
	= (Ff)(Fg)$.
\end{thrm}

\begin{proof}
	This follows immediately from Theorem \ref{thrm:ConvForFourierTransform}.
\end{proof}

%%%%%%%%%%%%%%%%%%%%%%%%%%%%%%%%%%%%%%%%%%%%%%%%%%%%%%%%
%% Properties of circled asterisk
%%%%%%%%%%%%%%%%%%%%%%%%%%%%%%%%%%%%%%%%%%%%%%%%%%%%%%%%

\begin{prop}
	\begin{enumerate}[i)]
		\item
			The operation $\circledast$ is associative on $\schwarz$.
		\item
			$\schwarz$ is isomorphic as an algebra with respect to
			$\circledast$ to itself as an algebra with respect to
			pointwise multiplication. $F$ is such an isomorphism.
	\end{enumerate}
\end{prop}

\begin{proof}
	\begin{enumerate}[i)]
	
		%% circled asterisk is associative
		\item
			Suppose $f,g,h \in \schwarz$. Then by Theorem
			\ref{thrm:ConvForFourierTransform2},
				\begin{IEEEeqnarray*}{rCl}
					F(\convc{(\convc{f}{g})}{h}) & = &
						F(\convc{f}{g}) F(h) \\
					& = & (Ff)(Fg)(Fh) \\
					& = & (Ff) F(\convc{g}{h}) \\
					& = & F(\convc{f}{(\convc{g}{h})})
				\end{IEEEeqnarray*}
			But by Theorem \ref{thrm:FisAutomorphism} $F$
			is a bijection. Hence
				\begin{displaymath}
					\convc{(\convc{f}{g})}{h} =
						\convc{f}{(\convc{g}{h})}
				\end{displaymath}
		
		%% F is an isomorphism
		\item
			This follows immmediately from Theorem 
			\ref{thrm:ConvForFourierTransform2}.
	
	\end{enumerate}
\end{proof}

%%%%%%%%%%%%%%%%%%%%%%%%%%%%%%%%%%%%%%%%%%%%%%%%%%%%%%%%
%% Another general propostion
%%%%%%%%%%%%%%%%%%%%%%%%%%%%%%%%%%%%%%%%%%%%%%%%%%%%%%%%

\begin{prop}\label{prop:PropertiesOfF}
	Suppose $f,g \in \schwarz$. Then
		\begin{enumerate}[i)]
			\item
				$FFf = f^-$
			\item
				$F^4 f = f$
			\item
				$\myintb{\f(x)g(x)}{x} = \myintb{f(x)\f[g](x)}{x}$
			\item
				$\conj{Ff} = F(\conj{f^-})$.
		\end{enumerate}
\end{prop}

\begin{proof}
	\begin{enumerate}[i)]
		
		\item
			This follows from Theorem \ref{thrm:FisAutomorphism}.
			
		\item
			This follows from i).
			
		\item
			Since $f,g \in \schwarz$ they belong to $\absint$.
			Hence by Theorem \ref{thrm:Fubini3},
				\begin{IEEEeqnarray*}{rCl}
					\myintb{\f(x)g(x)}{x} & = &
						\myintb{\left(\myintb{f(t)e^{-ixt}}{t}
						\right)g(x)}{x} \\
					& = & \myintb{\left(\myintb{f(t)g(x)e^{-ixt}}{t}
						\right)}{x} \\
					& = & \myintb{f(t)\left(\myintb{g(x)e^{-ixt}}{x}
						\right)}{t} \\
					& = & \myintb{f(t)\f[g](t)}{t}
				\end{IEEEeqnarray*}
			
		\item
			By definition,
				\begin{IEEEeqnarray*}{rCl}
					\conj{Ff(x)} & = & \frac{1}{\sqrt{2\pi}} \conj{
						\myintb{f(t)e^{-ixt}}{t}} \\
					& = & \frac{1}{\sqrt{2\pi}}\myintb{\conj{f(t)}
						e^{ixt}}{t} \\
					& = & \frac{1}{\sqrt{2\pi}}\myintb{\conj{f(-s)}
						e^{-ixs}}{s} \; \; (s=-t) \\
					& = & F(\conj{f^-})
				\end{IEEEeqnarray*}
				
	\end{enumerate}
\end{proof}

%%%%%%%%%%%%%%%%%%%%%%%%%%%%%%%%%%%%%%%%%%%%%%%%%%%%%%%%
%% Defn of inner product and norm on Schwarz Space
%%%%%%%%%%%%%%%%%%%%%%%%%%%%%%%%%%%%%%%%%%%%%%%%%%%%%%%%

\begin{defn}
	For $f,g \in \schwarz$ define
		\begin{displaymath}
			\innerprod{f}{g} = \frac{1}{\sqrt{2\pi}}
				\myintb{f(x)\conj{g(x)}}{x}
		\end{displaymath}
	Then $\innerprod{\cdot}{\cdot}$ is an inner product on $\schwarz$
	and the norm induced from this inner product is obviously
	$\norm[2]{f} = \innerprod{f}{f}^{1/2}$.
\end{defn}

%%%%%%%%%%%%%%%%%%%%%%%%%%%%%%%%%%%%%%%%%%%%%%%%%%%%%%%%
%% F is unitary
%%%%%%%%%%%%%%%%%%%%%%%%%%%%%%%%%%%%%%%%%%%%%%%%%%%%%%%%

\begin{thrm}
	$F$ is a unitary operator on $\schwarz$. That is,
	for all $f,g \in \schwarz$,
		\begin{enumerate}[i)]
			\item
				$\innerprod{Ff}{Fg} = \innerprod{f}{g}$.
			\item
				$\norm[2]{Ff} = \norm[2]{f}$.
		\end{enumerate}
\end{thrm}

\begin{proof}
	\begin{enumerate}[i)]
	
		\item
			By Proposition \ref{prop:PropertiesOfF},
				\begin{IEEEeqnarray*}{rCl}
					\innerprod{Ff}{Fg} & = & \frac{1}{\sqrt{2\pi}}
						\myintb{Ff(x)\conj{Fg(x)}}{x} \\
					& = & \frac{1}{\sqrt{2\pi}}\myintb
						{f(x)F(\conj{Fg(x)})}{x} \\
					& = & \frac{1}{\sqrt{2\pi}}\myintb
						{f(x)(FF(\conj{g^-(x)}))}{x} \\
					& = & \frac{1}{\sqrt{2\pi}}\myintb
						{f(x)\conj{g(x)}}{x} \\
					& = & \innerprod{f}{g}
				\end{IEEEeqnarray*}
			
		\item
			This follows immediately from part i).
			
	\end{enumerate}
\end{proof}

%%%%%%%%%%%%%%%%%%%%%%%%%%%%%%%%%%%%%%%%%%%%%%%%%%%%%%%%
%% Metric completion of Schwarz Space
%%%%%%%%%%%%%%%%%%%%%%%%%%%%%%%%%%%%%%%%%%%%%%%%%%%%%%%%

	The following theorem will be stated without proof.
	
\begin{thrm}
	Let $L^2$ denote the metric completion of $\schwarz$ with respect
	to $\norm[2]{\cdot}$. Then $L^2$ is a Hilbert space (a complete
	inner product space), F has a unique continuous extension, say
	$\tilde{F}$, to a map from $L^2$ onto itself and this $\tilde{F}$
	is a Hilbert space automorphism of $L^2$. Moreover, $\schwarz$ is
	a dense subspace of $L^2$.
\end{thrm}

\end{section}