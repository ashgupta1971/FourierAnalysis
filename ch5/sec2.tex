\begin{section}{Theorem of Approximation by Convolution}

	In this section we present the Theorem of Approximation by
	Convolution and two of its consequences. One of which 
	involves the heat problem and another
	providing an alternate proof of the Weierstrass Approximation
	Theorem. But before presenting any of these results, we prove
	the following.
	
%%%%%%%%%%%%%%%%%%%%%%%%%%%%%%%%%%%%%%%%%%%%%%%%%%%%%%%%%%%%%%%%%%%%%
%% Utility Grade of the Convolution Theorem for the Fourier Transform
%%%%%%%%%%%%%%%%%%%%%%%%%%%%%%%%%%%%%%%%%%%%%%%%%%%%%%%%%%%%%%%%%%%%%

\begin{thrm}[Convolution Theorem for the Fourier Transform]
\label{thrm:ConvForFourierTransform}
	Suppose $f,g \in \absint \cap \ctszeror$. Then $\conva{f}{g}
	\in \absint \cap \ctszeror$, $\F[\conva{f}{g}] = \f\f[g]$
	and $\norm[1]{\conva{f}{g}} \leq \norm[1]{f}\norm[1]{g}$.
\end{thrm}

\begin{proof}
	We will divide the proof into steps.
		\begin{enumerate}[i)]
		
			%% f*g is in ctszeror
			\item
				For $\isnatrl[k]$ choose $\gamma_k \in \ctszeror$ such
				that $0 \leq \gamma_k(t) \leq 1$, $\gamma_k(t) = 1$ if
				$-k \leq t \leq k$ and $\gamma_k(t) = 0$ if $\modulus{t}
				\geq k+1$. Now let $g_k = \gamma_k g$ for $\isnatrl[k]$.
				Then for $\isreal$,
					\begin{IEEEeqnarray*}{rCl}
						\modulus{\convb{f}{g}{x}-\convb{f}{g_k}{x}} & = &
							\modulus{\convb{f}{(g-g_k)}{x}} \\
						& \leq & \supnorm{f}\norm[1]{g-g_k} \\
						& & \text{(by Step 2 of the proof of Lemma \ref{lemma:Conv1})} \\
						& \leq & \supnorm{f} \left(2\myinta{-\infty}{-k}
							{\modulus{g(t)}}{t} + 2\myinta{k}{\infty}{\modulus{g(t)}}{t}
							\right)
					\end{IEEEeqnarray*}
				It follows that $\{\conva{f}{g_k}\}_{k=1}^\infty$ converges to
				$\conva{f}{g}$ uniformly on $\real$ and hence that $\conva{f}{g}$
				lies in $\ctszeror$.
				
			%% f*g is in absint
			\item
				For any $A > 0$,
					\begin{IEEEeqnarray*}{rCl}
						\myintb[A]{\modulus{\convb{f}{g}{x}}}{x} & \leq &
							\myintb[A]{\left(\myintb{\modulus{f(t)}
							\modulus{g(x-t)}}{t}\right)}{x} \\
						& = & \myintb{\left(\myintb[A]{\modulus{f(t)}
							\modulus{g(x-t)}}{x}\right)}{t} 
							\; \; \text{(by Theorem \ref{thrm:Fubini2})} \\
						& = & \myintb{\modulus{f(t)}\left(\myintb[A]
							{\modulus{g(x-t)}}{x}\right)}{t} \\
						& \leq & \myintb{\modulus{f(t)}\norm[1]{g}}{t} \\
						& = & \norm[1]{f}\norm[1]{g}
					\end{IEEEeqnarray*}
				Hence $\conva{f}{g} \in \absint$ and $\norm[1]{\conva{f}{g}} \leq
				\norm[1]{f}\norm[1]{g}$. A similar argument shows that
					\begin{displaymath}
						\myintb{\convb{f}{g}{x}}{x} = \myintb{f(t)}{t}
							\myintb{g(t)}{t}
					\end{displaymath}
				
			%% Final step -- equality of transforms
			\item
				For any $\isreal[y]$,
					\begin{IEEEeqnarray*}{rCl}
						\F[(\conva{f}{g})](y) & = & \myintb{\left(\myintb
							{f(t)g(x-t)}{t}\right)e^{-iyx}}{x} \\
						& = & \myintb{\left(\myintb{f(t)e^{-iyt}
							g(x-t)e^{-iy(x-t)}}{t}\right)}{x} \\
						& = & \myintb{\convb{f_y}{g_y}{x}}{x}
					\end{IEEEeqnarray*}
				where $f_y(t) = f(t)e^{-iyt}$ and $g_y(t) = g(t)e^{-iyt}$
				for all $\isreal[t]$. By the remarks in Step 1, we get
					\begin{IEEEeqnarray*}{rCl}
						\F[(\conva{f}{g})](y) & = & \myintb{\convb
							{f_y}{g_y}{x}}{x} \\
						& = & \myintb{f(t)e^{-iyt}}{t}\myintb{g(s)e^{-iys}}{s} \\
						& = & \f(y)\f[g](y)
					\end{IEEEeqnarray*}
					
		\end{enumerate}
\end{proof}

%%%%%%%%%%%%%%%%%%%%%%%%%%%%%%%%%%%%%%%%%%%%%%%%%%%%%%%%%%%%%%%%%%%%%
%% Theorem involving Fourier transform of derivative of a fn
%%%%%%%%%%%%%%%%%%%%%%%%%%%%%%%%%%%%%%%%%%%%%%%%%%%%%%%%%%%%%%%%%%%%%

\begin{thrm}
	Suppose $f \in \absint \cap \ctszeror \cap \ctsdiff[1](\real)$ and
	$f' \in \absint$. Then $\F[f'](x)=ix\f(x)$.
\end{thrm}

\begin{proof}
	For all $\isreal$,
		\begin{IEEEeqnarray*}{rCl}
			\F[f'](x) & = & \myintb{f'(t)e^{-ixt}}{t} \\
			& = & \evalat{f(t)e^{-ixt}}{t}{-\infty}{\infty}
				+ ix \myintb{f(t)e^{-ixt}}{t} \\
			& = & ix \f(x)
		\end{IEEEeqnarray*}
\end{proof}

%%%%%%%%%%%%%%%%%%%%%%%%%%%%%%%%%%%%%%%%%%%%%%%%%%%%%%%%%%%%%%%%%%%%%
%% Theorem of Approximation by Convolution
%%%%%%%%%%%%%%%%%%%%%%%%%%%%%%%%%%%%%%%%%%%%%%%%%%%%%%%%%%%%%%%%%%%%%

\begin{thrm}\label{thrm:ApproxByConv}
	Suppose $g \in \absint$, $0 \leq g(t) \in \real$ for all $\isreal[t]$
	and
		\begin{displaymath}
			\myintb{g(t)}{t} = 1
		\end{displaymath}
	For $\lambda > 0$ let $H_\lambda g(t)=\lambda g(\lambda t)$ for all
	$\isreal[t]$. If $f:\real \rightarrow \cmplx$ and $f$ is bounded and
	uniformly continuous on $\real$ then
		\begin{displaymath}
			\lim_{\lambda \rightarrow \infty}\convb{f}{H_\lambda g}{x}
				= f(x) \; \; \text{(uniformly)}
		\end{displaymath}
\end{thrm}

\begin{proof}
	Choose $M > 0$ such that $\modulus{f(t)}<M$ for all $\isreal[t]$. Let
	$\epsilon>0$. Choose $\delta>0$ such that $\modulus{f(s)-f(t)}<\epsilon/2$
	whenever $\modulus{s-t} \leq \delta$. Choose $\lambda_0>0$ such that
		\begin{displaymath}
			\myintb{g(s)}{s}-\myintb[\lambda \delta]{g(s)}{s} <
				\frac{\epsilon}{4M}
		\end{displaymath}
	whenever $0 < \lambda_0 < \lambda \in \real$. Note that
		\begin{displaymath}
			\myintb{H_\lambda g(t)}{t} = \myintb{g(s)}{s}
		\end{displaymath}
	for all $\lambda>0$. Then for all $\isreal$ and $\lambda>\lambda_0$,
		\begin{IEEEeqnarray*}{rCl}
			\modulus{\convb{f}{H_\lambda g}{x}-f(x)} & = &
				\modulus{\myintb{f(x-t)\lambda g(\lambda t)}{t}
				- f(x)\myintb{\lambda g(\lambda t)}{t}} \\
			& = & \modulus{\myintb{[f(x-t)-f(x)]\lambda g(\lambda t)}{t}} \\
			& \leq & \myinta{-\infty}{-\delta}{2M\lambda g(\lambda t)}{t}
				+ \myintb[\delta]{\frac{\epsilon}{2}\lambda g(\lambda t)}{t}
				+ \myinta{\delta}{\infty}{2M\lambda g(\lambda t)}{t} \\
			& = & 2M\left(\myinta{-\infty}{-\delta}{\lambda g(\lambda t)}{t}
				+ \myinta{\delta}{\infty}{\lambda g(\lambda t)}{t}\right)
				+ \frac{\epsilon}{2}\myintb{\lambda g(\lambda t)}{t} \\
			& = & 2M\left(\myintb{\lambda g(\lambda t)}{t}
				- \myintb[\lambda \delta]{\lambda g(\lambda t)}{t}\right)
				+ \frac{\epsilon}{2}\myintb{\lambda g(\lambda t)}{t} \\
			& < & 2M\left(\frac{\epsilon}{4M}\right)+\frac{\epsilon}{2} \\
			& = & \epsilon
		\end{IEEEeqnarray*}
\end{proof}

%%%%%%%%%%%%%%%%%%%%%%%%%%%%%%%%%%%%%%%%%%%%%%%%%%%%%%%%%%%%%%%%%%%%%
%% Heat Problem Revisited
%%%%%%%%%%%%%%%%%%%%%%%%%%%%%%%%%%%%%%%%%%%%%%%%%%%%%%%%%%%%%%%%%%%%%

\begin{thrm}[Heat Problem. Revisited]
	Suppose $f:\real \rightarrow \cmplx$ is bounded and uniformly
	continuous on $\real$. Define $u(x,t):\real \times [0,\infty)
	\rightarrow \cmplx$ by
		\begin{displaymath}
			u(x,t) = 
				\begin{cases}
					\displaystyle{\frac{1}{\sqrt{2\pi t}}
						\myintb
						{f(y)\exp\left(-\frac{(x-y)^2}{2t}\right)}{y}}
						& \isreal, t>0 \\
					f(x) & \isreal, t=0
				\end{cases}
		\end{displaymath}
	Then
		\begin{enumerate}[i)]
			\item
				$u$ is continuous on its domain.
			\item
				$u \in \ctsdiff$ on $\real \times (0,\infty)$.
			\item
				$\displaystyle{\frac{\partial u}{\partial t}(x,t) =
				\frac{\partial^2 u}{\partial x^2}(x,t)}$ for all $\isreal,
				t>0$.
			\item
				$u(x,0)=f(x)$.
		\end{enumerate}
\end{thrm}

\begin{proof}
	Statement iv) is obvious and statements ii) and iii) can be easily
	proven by using Theorem \ref{thrm:Leibniz2}. For statement i), note
	that
		\begin{displaymath}
			u(x,t) = \convb{f}{H_{1/\sqrt{t}}G}{x}
		\end{displaymath}
	for $\isreal, t>0$, where $G$ is the Gaussian distribution and is
	defined by
		\begin{displaymath}
			G(t) = \frac{1}{\sqrt{2\pi}}e^{-t^2/2}
		\end{displaymath}
	By Theorem \ref{thrm:ApproxByConv},
		\begin{displaymath}
			\lim_{t \rightarrow 0}u(x,t)
				= \lim_{t \rightarrow 0}\convb{f}{H_{1/\sqrt{t}}G}{x}
				= f(x)
		\end{displaymath}
	uniformly for all $\isreal$. Hence $u$ is continuous on its entire
	domain.
\end{proof}

%%%%%%%%%%%%%%%%%%%%%%%%%%%%%%%%%%%%%%%%%%%%%%%%%%%%%%%%%%%%%%%%%%%%%
%% Weierstrass Approximation Theorem Revisited
%%%%%%%%%%%%%%%%%%%%%%%%%%%%%%%%%%%%%%%%%%%%%%%%%%%%%%%%%%%%%%%%%%%%%

\begin{thrm}[Weierstrass Approximation Theorem Revisited]
	Suppose $[a,b] \subset \real, \phi:[a,b] \rightarrow \real$ and
	$\phi$ is continuous. Then for all $\epsilon>0$ there exists a
	polynomial function $P:\real \rightarrow \real$ such that
	$\modulus{\phi(x)-P(x)}<\epsilon$ for all $x \in [a,b]$.
\end{thrm}

\begin{proof}
	Choose $A>0$ such that $[a,b] \subset (-A,A)$ and choose a
	continuous function $f:\real \rightarrow \real$ which agrees
	with $\phi$ on $[a,b]$ and vanishes outside $(-A,A)$. Fix
	$\epsilon>0$. By Theorem \ref{thrm:ApproxByConv}, we can choose
	$t>0$ such that
		\begin{displaymath}
			\modulus{f(x)-\convb{f}{H_{1/\sqrt{t}}G}{x}} < \epsilon/2
		\end{displaymath}
	for all $\isreal$. Observe that
		\begin{displaymath}
			\convb{f}{H_{1/\sqrt{t}}G}{x} = \frac{1}{\sqrt{2\pi t}}
				\myintb{f(y)\exp\left(-\frac{(x-y)^2}{2t}\right)}{y}
		\end{displaymath}
	for all $\isreal$. Let
		\begin{displaymath}
			F(z) = \frac{1}{\sqrt{2\pi t}}
				\myintb{f(y)\exp\left(-\frac{(z-y)^2}{2t}\right)}{y}
		\end{displaymath}
	for $\iscmplx$. It follows from Theorem \ref{thrm:Fubini1} that
	$F$ is continuous. Now if $\gamma:[\alpha,\beta] \rightarrow \cmplx$
	is a closed, $\ctsdiff[1]$ curve then
		\begin{IEEEeqnarray*}{rCl}
			\myintc{F(z)}{z} & = & \myinta{\alpha}{\beta}
				{F(\gamma(s))\gamma'(s)}{s} \\
			& = & \myinta{\alpha}{\beta}{\left(\frac{1}{\sqrt{2\pi t}}
				\myintb[A]{f(y)\exp\left(-\frac{(\gamma(s)-y)^2}{2t}\right)}{y}
				\right)\gamma'(s)}{s} \\
			& = & \frac{1}{\sqrt{2\pi t}}\myintb[A]{\left(
				\myinta{\alpha}{\beta}
				{\exp\left(-\frac{(\gamma(s)-y)^2}{2t}\right)\gamma'(s)}{s}\right)
				f(y)}{y} \\
			& = & \frac{1}{\sqrt{2\pi t}}\myintb[A]{\left(\myintc
				{\exp\left(-\frac{(z-y)^2}{2t}\right)}{z}\right)f(y)}{y} \\
			& = & 0
		\end{IEEEeqnarray*}
	by Cauchy's Theorem. Thus
		\begin{displaymath}
			\myintc{F(z)}{z} = 0
		\end{displaymath}
	for every closed, $\ctsdiff[1]$ curve $\gamma$ in $\cmplx$. By
	Morera's Theorem, $F$ is entire and so its Taylor series converges
	to it uniformly on the set $\{\iscmplx \; : \; \modulus{z} \leq A\}$.
	It follows that there exists a polynomial function $P_1:\cmplx \rightarrow
	\cmplx$ such that $\modulus{F(z)-P_1(z)}<\epsilon/2$ for all $\iscmplx$,
	$\modulus{z} \leq A$. Since $F$ is real valued on the real line, its
	derivatives at zero are real and so $P_1$ is real valued on the reals
	as well. Let $P = P_1 \vert_\real$. Then
		\begin{IEEEeqnarray*}{rCl}
			\modulus{f(x)-P(x)} & \leq & \modulus{f(x)-F(x)}
				+ \modulus{F(x)-P(x)} \\
			& < & \frac{\epsilon}{2} + \frac{\epsilon}{2} \\
			& = & \epsilon
		\end{IEEEeqnarray*}
	for all $x \in [-A,A]$.
\end{proof}

\end{section}