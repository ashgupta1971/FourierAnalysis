\begin{section}{The Fourier Transform}

	In this section we present the Fourier Transform
	of a function and its properties.
	
%%%%%%%%%%%%%%%%%%%%%%%%%%%%%%%%%%%%%%%%%%%%%%%%%%%%%%%%%%%%
%% Defn of Fourier Transform
%%%%%%%%%%%%%%%%%%%%%%%%%%%%%%%%%%%%%%%%%%%%%%%%%%%%%%%%%%%%

\begin{defn}
	Suppose $f \in \absint$. Then since $\modulus{f(t)e^{-ixt}}
	= \modulus{f}$ for all $\isreal$, the map $t \mapsto f(t)e^{-ixt}$
	belongs to $\absint$. We define
		\begin{displaymath}
			\f(x)=\myintb{f(t)e^{-ixt}}{t}
		\end{displaymath}
	for all $\isreal$ and call $\f$ the \emph{Fourier Transform} of $f$.
\end{defn}

%%%%%%%%%%%%%%%%%%%%%%%%%%%%%%%%%%%%%%%%%%%%%%%%%%%%%%%%%%%%
%% Properties of Fourier Transform
%%%%%%%%%%%%%%%%%%%%%%%%%%%%%%%%%%%%%%%%%%%%%%%%%%%%%%%%%%%%

\begin{thrm}
	If $f \in \absint$ then
		\begin{enumerate}[i)]
			\item
				$\lim_{\modulus{x} \rightarrow \infty} \f(x)=0$
			\item
				$\f$ is continuous.
		\end{enumerate}
\end{thrm}

\begin{proof}
	\begin{enumerate}[i)]
	
		\item
			Given $\epsilon > 0$ choose $A > 0$ such that
				\begin{displaymath}
					\myinta{-\infty}{-A}{\modulus{f(t)}}{t}
						+ \myinta{A}{\infty}{\modulus{f(t)}}{t}
						< \epsilon/2
				\end{displaymath}
			Then for all $\isreal$,
				\begin{IEEEeqnarray*}{rCl}
					\modulus{\myintb{f(t)e^{-ixt}}{t}} & \leq &
						\myinta{-\infty}{-A}{\modulus{f(t)e^{-ixt}}}{t}
						+ \modulus{\myintb[A]{f(t)e^{-ixt}}{t}}
						+ \myinta{A}{\infty}{\modulus{f(t)e^{-ixt}}}{t} \\
					& = & \myinta{-\infty}{-A}{\modulus{f(t)}}{t}
						+ \modulus{\myintb[A]{f(t)e^{-ixt}}{t}}
						+ \myinta{A}{\infty}{\modulus{f(t)}}{t} \\
					& < & \epsilon/2 + \modulus{\myintb[A]{f(t)e^{-ixt}}{t}}
						\; (\ast)
				\end{IEEEeqnarray*}
			By Theorem \ref{thrm:RLL},
				\begin{displaymath}
					\lim_{\modulus{x} \rightarrow \infty}
						\myintb[A]{f(t)e^{-ixt}}{t} = 0
				\end{displaymath}
			Therefore, from $(\ast)$ we get
				\begin{displaymath}
					\limsup_{\modulus{x} \rightarrow \infty}
						\modulus{\f(x)} < \epsilon
				\end{displaymath}
			Since this is true for all $\epsilon > 0$ we have
				\begin{displaymath}
					\lim_{\modulus{x} \rightarrow \infty}
						\f(x) = 0
				\end{displaymath}
				
		\item
			Fix $\isreal$ and choose a real sequence $h_k$ converging
			to zero as $k \rightarrow \infty$. Let $\epsilon > 0$ and
			choose $A > 0$ such that
				\begin{displaymath}
					\myinta{-\infty}{-A}{\modulus{f(t)}}{t}
						+ \myinta{A}{\infty}{\modulus{f(t)}}{t}
						< \epsilon/2
				\end{displaymath}
			Then for all $k \geq 1$,
				\begin{IEEEeqnarray*}{rCl}
					\modulus{\f(x+h_k)-\f(x)} & = & \modulus{
						\myintb{f(t)[e^{-i(x+h_k)t}-e^{-ixt}]}{t}} \\
					& \leq & \myinta{-\infty}{-A}{2\modulus{f(t)}}{t}
						+ \modulus{\myintb[A]
							{f(t)e^{-ixt}(e^{-i h_k t}-1)}{t}} \\
					& & + \; \myinta{A}{\infty}{2\modulus{f(t)}}{t} \\
					& < & \epsilon + \myintb[A]{\modulus{f(t)}
						\modulus{e^{-i h_k t}-1}}{t}
				\end{IEEEeqnarray*}
			But $\modulus{e^{-i h_k t}-1} \rightarrow 0$ uniformly on $[-A,A]$ as $k \rightarrow
			\infty$.
			So
				\begin{displaymath}
					\lim_{k \rightarrow \infty}\modulus{\f(x+h_k)-\f(x)}
						\leq 2\epsilon
				\end{displaymath}
			from which it follows that
				\begin{displaymath}
					\lim_{k \rightarrow \infty}\f(x+h_k)=f(x)
				\end{displaymath}
			Thus $\f$ is continuous at $x$.
			
	\end{enumerate}
\end{proof}

%%%%%%%%%%%%%%%%%%%%%%%%%%%%%%%%%%%%%%%%%%%%%%%%%%%%%%%%%%%%
%% Some useless definitions
%%%%%%%%%%%%%%%%%%%%%%%%%%%%%%%%%%%%%%%%%%%%%%%%%%%%%%%%%%%%

\begin{defn}
	Let
		\begin{enumerate}[i)]
			\item
				$\ctsr = \{f:\real \to \cmplx \; | \; 
					f \text{ is continuous} \}$
			\item
				$\ctszeror = \{g \in \ctsr \; | \;
					\lim_{\modulus{x} \rightarrow \infty} g(x) = 0 \}$
			\item
				$\ctscompact = \{g \in \ctsr \; | \; \text{there exists }
					A > 0 \text{ such that } g(x)=0 \text{ if }
					\modulus{x} > A \}$
		\end{enumerate}
\end{defn}

	The following Proposition will be stated without proof.
	
\begin{prop}
	\begin{enumerate}[i)]
		\item
			$\ctscompact \subset \ctszeror \subset \ctsr$
		\item
			$\ctscompact \subset \absint$
		\item
			$\ctszeror \not \subset \absint$
		\item
			$\absint \cap \ctsr \not \subset \ctszeror$
		\item
			$\ctszeror$ is a Banach space with respect to the
			sup norm.
		\item
			$\ctscompact$ forms a dense subspace of $\ctszeror$.
		\item
			If $f \in \absint$ then $\f \in \ctszeror$.
	\end{enumerate}
\end{prop}

%%%%%%%%%%%%%%%%%%%%%%%%%%%%%%%%%%%%%%%%%%%%%%%%%%%%%%%%%%%%
%% Second proposition on Fourier Transforms
%%%%%%%%%%%%%%%%%%%%%%%%%%%%%%%%%%%%%%%%%%%%%%%%%%%%%%%%%%%%

\begin{defn}
	Given $f \in \absint$ let $\ftransform=\f$. Then
	$\ftransform[]$ is a linear transformation from $\absint$
	to $\ctszeror$.
\end{defn}

\begin{prop}\label{prop:FTransform}
	If $f \in \absint$ then $\supnorm{\ftransform}
	\leq \norm[1]{f}$ and $\ftransform[]$ is uniformly
	continuous from $\absint$ to $\ctszeror$.
\end{prop}

\begin{proof}
	If $f \in \absint$ and $\isreal$ then
		\begin{displaymath}
			\modulus{\f(x)} = \modulus{
				\myintb{f(t)e^{-ixt}}{t}}
				\leq \myintb{\modulus{f(t)}}{t}
				= \norm[1]{f}
		\end{displaymath}
	which means $\supnorm{\f} \leq \norm[1]{f}$.
	This also implies for all $f,g \in \absint$
		\begin{displaymath}
			\supnorm{\ftransform-\ftransform{g}}
			= \supnorm{\ftransform[(f-g)]}
			\leq \norm[1]{f-g}
		\end{displaymath}
	so that $\ftransform[]$ is uniformly continuous from
	$\absint$ to $\ctszeror$.
\end{proof}

%%%%%%%%%%%%%%%%%%%%%%%%%%%%%%%%%%%%%%%%%%%%%%%%%%%%%%%%%%%%
%% Example of Fourier transform bound
%%%%%%%%%%%%%%%%%%%%%%%%%%%%%%%%%%%%%%%%%%%%%%%%%%%%%%%%%%%%

\begin{ex}
	Let $f(t) = e^{-\modulus{t}}$ for $\isreal[t]$. Show
	that $\supnorm{\ftransform} \leq \norm[1]{f}$.
\end{ex}

\begin{soln}
	For any $A > 0$ we have
		\begin{displaymath}
			\myinta{0}{A}{\modulus{f(t)}}{t}
				= \myinta{0}{A}{e^{-t}}{t}
				= 1 - e^{-A}
		\end{displaymath}
	$\rightarrow 1$ as $A \rightarrow \infty$. Similarly
		\begin{displaymath}
			\myinta{-A}{0}{\modulus{f(t)}}{t}
		\end{displaymath}
	$\rightarrow 1$ as $A \rightarrow \infty$. Thus
	$f \in \absint$ and
		\begin{displaymath}
			\norm[1]{f} = \myintb{\modulus{f(t)}}{t}
				= 2
		\end{displaymath}
	Now let $\isreal$. Then for $A > 0$
		\begin{IEEEeqnarray*}{rCl}
			\myinta{0}{A}{f(t)e^{-ixt}}{t} & = &
				\myinta{0}{A}{e^{-(1+ix)t}}{t} \\
			& = & \evalat{-\frac{e^{-(1+ix)t}}{1+ix}}
				{t}{0}{A} \\
			& = & \frac{1}{1+ix}\left(1-e^{-A}e^{-ixA}
				\right) \\
			& \rightarrow & \frac{1}{1+ix}
		\end{IEEEeqnarray*}
	as $A \rightarrow \infty$. Similarly,
		\begin{displaymath}
			\myinta{-\infty}{0}{f(t)e^{-ixt}}{t}
				= \frac{1}{1-ix}
		\end{displaymath}
	Thus
		\begin{displaymath}
			\ftransform = \frac{1}{1+ix} + \frac{1}{1-ix}
				= \frac{2}{1+x^2}
				\leq 2
				= \norm[1]{f}
		\end{displaymath}
	as required.
\end{soln}

%%%%%%%%%%%%%%%%%%%%%%%%%%%%%%%%%%%%%%%%%%%%%%%%%%%%%%%%%%%%
%% First example of Fourier transform of a fn
%%%%%%%%%%%%%%%%%%%%%%%%%%%%%%%%%%%%%%%%%%%%%%%%%%%%%%%%%%%%

\begin{ex}\label{ex:GaussianDist}
	Let
		\begin{displaymath}
			G(t) = \frac{e^{-t^2/2}}{\sqrt{2\pi}}
		\end{displaymath}
	for $\isreal[t]$. (This is called the \emph{Gaussian Distribution}.)
	Find $\f[G]$.
\end{ex}

\begin{soln}
	It is easily checked that $G \in \absint \cap \ctszero^\infty(\real)$.
	Therefore, by Theorem \ref{thrm:Leibniz2} and using Integration by
	Parts, we get
		\begin{IEEEeqnarray*}{rCl}
			\f[G]'(x) & = & \frac{d}{dx}\myintb{G(t)e^{-ixt}}{t} \\
			& = & \myintb{\frac{\partial}{\partial x}\left(
				\frac{e^{-t^2/2}}{\sqrt{2\pi}}e^{-ixt}\right)}{t} \\
			& = & \frac{-i}{\sqrt{2\pi}}\myintb{te^{-t^2/2}e^{-ixt}}{t} \\
			& = & \evalat{\frac{i}{\sqrt{2\pi}}e^{-t^2/2}e^{-ixt}}{t}
				{-\infty}{\infty} - x\myintb{\frac{e^{-t^2/2}}{\sqrt{2\pi}}
				e^{-ixt}}{t} \\
			& = & -x \f[G](x)
		\end{IEEEeqnarray*}
	Thus
		\begin{displaymath}
			\f[G](x) = C e^{-x^2/2}
		\end{displaymath}
	But
		\begin{displaymath}
			\f[G](0) = \myintb{\frac{e^{-t^2/2}}{\sqrt{2\pi}}}{t}
				= 1
		\end{displaymath}
	So $C = 1$ and $\f[G](x)=\sqrt{2\pi}G(x), \isreal$.
\end{soln}

%%%%%%%%%%%%%%%%%%%%%%%%%%%%%%%%%%%%%%%%%%%%%%%%%%%%%%%%%%%%
%% A trivial proposition
%%%%%%%%%%%%%%%%%%%%%%%%%%%%%%%%%%%%%%%%%%%%%%%%%%%%%%%%%%%%

\begin{prop}
	Suppose $f \in \absint$ and $0 < \lambda \in \real$. Define
	$f_\lambda(t)=\lambda f(\lambda t), \isreal[t]$. Then
	$f_\lambda \in \absint$ and $\f_\lambda(x)=\f(\frac{x}{\lambda}),
	\isreal$.
\end{prop}

\begin{proof}
	For the first part of the proof, note that
		\begin{IEEEeqnarray*}{rCl}
			\myintb{\modulus{f_\lambda(t)}}{t} & = & \lim_{A \rightarrow
				\infty}\myintb[A]{\lambda \modulus{f(\lambda t)}}{t} \\
			& = & \lim_{A \rightarrow \infty}\myintb[A/\lambda]
				{\modulus{f(s)}}{s} \; \; (s = \lambda t) \\
			& = & \myintb{\modulus{f(s)}}{s} \\
			& < & \infty
		\end{IEEEeqnarray*}
	So $f_\lambda \in \absint$. Next,
		\begin{IEEEeqnarray*}{rCl}
			\f_\lambda(x) & = & \myintb{f_\lambda(t)e^{-ixt}}{t} \\
			& = & \lim_{A \rightarrow \infty}\myintb[A]
				{\lambda f(\lambda t)e^{-ixt}}{t} \\
			& = & \lim_{A \rightarrow \infty}\myintb[A/\lambda]
				{f(s)e^{-ixs/\lambda}}{s} \; \; (s = \lambda t) \\
			& = & \myintb{f(s)e^{-ixs/\lambda}}{s} \\
			& = & \f(x/\lambda)
		\end{IEEEeqnarray*}
	This completes the proof.
\end{proof}

%%%%%%%%%%%%%%%%%%%%%%%%%%%%%%%%%%%%%%%%%%%%%%%%%%%%%%%%%%%%
%% Another trivial proposition
%%%%%%%%%%%%%%%%%%%%%%%%%%%%%%%%%%%%%%%%%%%%%%%%%%%%%%%%%%%%

\begin{prop}
	Suppose $f:\real \rightarrow \real$ is even. Then $\f$ is
	real valued and
		\begin{displaymath}
			\f(x) = \myintb{f(t)\cos xt}{t}
		\end{displaymath}
	for $\isreal$.
\end{prop}

\begin{proof}
	By definition,
		\begin{IEEEeqnarray*}{rCl}
			\f(x) & = & \myintb{f(t)e^{-ixt}}{t} \\
			& = & \myintb{f(t)\cos(-xt)}{t}
				+ i\myintb{f(t)\sin(-xt)}{t} \\
			& = & \myintb{f(t)\cos(xt)}{t}
		\end{IEEEeqnarray*}
	(since $f$ is even).
\end{proof}

%%%%%%%%%%%%%%%%%%%%%%%%%%%%%%%%%%%%%%%%%%%%%%%%%%%%%%%%%%%%
%% Another example calculating F Transform of a fn
%%%%%%%%%%%%%%%%%%%%%%%%%%%%%%%%%%%%%%%%%%%%%%%%%%%%%%%%%%%%

\begin{ex}
	Let
		\begin{displaymath}
			f(t) =
				\begin{cases}
					1/2 & -1 \leq t \leq 1 \\
					0 & \text{otherwise}
				\end{cases}
		\end{displaymath}
	Find $\f$ and show that it does not belong to $\absint$.
\end{ex}

\begin{soln}
	If $x=0$ then clearly $\f(x)=1$. Otherwise,
		\begin{IEEEeqnarray*}{rCl}
			\f(x) & = & \myintb{f(t)e^{-ixt}}{t} \\
			& = & \frac{1}{2}\myintb[1]{e^{-ixt}}{t} \\
			& = & \evalat{\frac{1}{-2ix}e^{-ixt}}{t}{-1}{1} \\
			& = & \frac{1}{2ix}\left(e^{ix}-e^{-ix}\right) \\
			& = & \frac{1}{2ix}\left(2i\sin x\right) \\
			& = & \frac{\sin x}{x}
		\end{IEEEeqnarray*}
	This function is not in $\absint$ as was shown in Example
	\ref{ex:SinTOverT}.
\end{soln}

%%%%%%%%%%%%%%%%%%%%%%%%%%%%%%%%%%%%%%%%%%%%%%%%%%%%%%%%%%%%
%% Second example of Fourier transform of a fn
%%%%%%%%%%%%%%%%%%%%%%%%%%%%%%%%%%%%%%%%%%%%%%%%%%%%%%%%%%%%

\begin{ex}
	Let
		\begin{displaymath}
			f(t) = t e^{-\modulus{t}}
		\end{displaymath}
	for $\isreal[t]$. Show that $f \in \absint$ and find $\f$.
\end{ex}

\begin{soln}
	We will divide the solution into steps.
		\begin{enumerate}[{Step} 1]
		
			%% f is in absint
			\item
				To show that $f \in \absint$ proceed as follows. First, if
				$t > 0$ then
					\begin{displaymath}
						f'(t) = \frac{e^t - t e^t}{e^{2t}}
					\end{displaymath}
				So $f$ is decreasing along the positive real axis when $t > 1$.
				Now let $a_n = n e^{-n}, \isnatrl$. Then
					\begin{displaymath}
						\lim_{n \rightarrow \infty}\frac{\modulus{a_{n+1}}}
							{\modulus{a_n}} = \frac{1}{e} < 1
					\end{displaymath}
				Hence, by the ratio test,
					\begin{displaymath}
						\sum_{n=1}^\infty n e^{-n}
					\end{displaymath}
				converges. But since $f(t)$ is decreasing when $t > 1$, this implies
				by the integral test,
					\begin{displaymath}
						\myinta{1}{\infty}{t e^{-t}}{t} < \infty
					\end{displaymath}
				Also,
					\begin{IEEEeqnarray*}{rCl}
						\myinta{-\infty}{-1}{\modulus{f(t)}}{t} & = &
							\myinta{-\infty}{-1}{(-t)e^t}{t} \\
						& = & \myinta{1}{\infty}{s e^{-s}}{s} \; \; (s = -t) \\
						& < & \infty
					\end{IEEEeqnarray*}
				Therefore $f \in \absint$.
	
			%% Calculation of f transform
			\item
				Let
					\begin{displaymath}
						f_1(x) = \myinta{0}{\infty}{f(t)e^{-ixt}}{t}
					\end{displaymath}
				Then using integration by parts, we get
					\begin{IEEEeqnarray*}{rCl}
						f_1(x) & = & \myinta{0}{\infty}{t e^{-t}e^{-ixt}}{t} \\
						& = & \myinta{0}{\infty}{t e^{(-1-ix)t}}{t} \\
						& = & \evalat{-t \frac{e^{-(1-ix)t}}{1+ix}}{t}{0}{\infty}
							+ \myinta{0}{\infty}{e^{(-1-ix)t}}{t} \\
						& = & \evalat{- \frac{e^{(-1-ix)t}}{1+ix}}{t}{0}{\infty} \\
						& = & \frac{1}{1+ix}
					\end{IEEEeqnarray*}
				Similarly, if
					\begin{displaymath}
						f_2(x) = \myinta{-\infty}{0}{f(t)e^{-ixt}}{t}
					\end{displaymath}
				then
					\begin{displaymath}
						f_2(x) = - \frac{1}{1-ix}
					\end{displaymath}
				Therefore,
					\begin{IEEEeqnarray*}{rCl}
						\f(x) & = & f_1(x) + f_2(x) \\
						& = & \frac{1}{1+ix} - \frac{1}{1-ix} \\
						& = & - \frac{2ix}{1+x^2}
					\end{IEEEeqnarray*}
				
		\end{enumerate}
\end{soln}

%%%%%%%%%%%%%%%%%%%%%%%%%%%%%%%%%%%%%%%%%%%%%%%%%%%%%%%%%%%%
%% Third example of Fourier transform of a fn
%%%%%%%%%%%%%%%%%%%%%%%%%%%%%%%%%%%%%%%%%%%%%%%%%%%%%%%%%%%%

\begin{ex}
	For $\isnatrl[k]$ let $G_k(t) = t^k G(t), \isreal[t]$ (where
	$G(t) = \exp(-t^2/2)/\sqrt{2\pi}$). Show that $G_k \in \absint
	\cap \ctsdiff(\real)$ and find $\f[G]_k$ for each $\isnatrl[k]$.
\end{ex}

\begin{soln}
	\begin{enumerate}[{Step} 1]
	
		%% Show that G_k is in absint
		\item
			For $t > 0$ it is easily checked that
				\begin{displaymath}
					G_k'(t) = \sqrt{2\pi}
						\frac{k t^{k-1}e^{t^2/2}-t^{k+1}e^{t^2/2}}
						{e^{t^2}}
				\end{displaymath}
			Hence, along the positive real axis, $G_k$ is decreasing when
			$t > \sqrt{k}$. Now for $\isnatrl[k]$ let $a_n = n^k G(n)$.
			Then
				\begin{IEEEeqnarray*}{rCl}
					\lim_{n \rightarrow \infty}\frac{\modulus{a_{n+1}}}
						{\modulus{a_n}} & = &
						\lim_{n \rightarrow \infty}\frac{(n+1)^k \exp(n^2/2)}
						{n^k \exp((n+1)^2/2)} \\
					& = & \lim_{n \rightarrow \infty}\frac{1}{\exp((2n+1)/2)} \\
					& = & e^{-1/2} \\
					& < & 1
				\end{IEEEeqnarray*}
			Therefore, by the ratio test,
				\begin{displaymath}
					\sum_{n=\ceil{\sqrt{k}}}^\infty a_n
						= \sum_{n=\ceil{\sqrt{k}}}^\infty n^k e^{-n^2/2}/\sqrt{2\pi}
				\end{displaymath}
			converges. But then, by the integral test (since $G_k(t)$ is decreasing when
			$t > \sqrt{k}$),
				\begin{displaymath}
					\myinta{\sqrt{k}}{\infty}{\modulus{G_k(t)}}{t}
						= \myinta{\sqrt{k}}{\infty}{\frac{t^k e^{-t^2/2}}{\sqrt{2\pi}}}{t}
				\end{displaymath}
			converges as well. Thus, if $k$ is even,
				\begin{IEEEeqnarray*}{rCl}
					\myinta{-\infty}{-\sqrt{k}}{\modulus{G_k(t)}}{t}
						& = & \myinta{-\infty}{-\sqrt{k}}{\frac{t^k e^{-t^2/2}}{\sqrt{2\pi}}}{t} \\
						& = & \myinta{\sqrt{k}}{\infty}{\frac{s^k e^{-s^2/2}}{\sqrt{2\pi}}}{s} \; \; (s=-t) \\
						& < & \infty
				\end{IEEEeqnarray*}
			If $k$ is odd,	
				\begin{IEEEeqnarray*}{rCl}
					\myinta{-\infty}{-\sqrt{k}}{\modulus{G_k(t)}}{t}
						& = & \myinta{-\infty}{-\sqrt{k}}{\frac{-t^k e^{-t^2/2}}{\sqrt{2\pi}}}{t} \\
						& = & \myinta{\sqrt{k}}{\infty}{\frac{s^k e^{-s^2/2}}{\sqrt{2\pi}}}{s} \; \; (s=-t) \\
						& < & \infty
				\end{IEEEeqnarray*}
			as well. Therefore, in all cases, we have $G_k \in \absint$. Clearly, $G_k$ belongs
			to $\ctsdiff(\real)$. So we have proved the first assertion.
			
		%% Calculation of the F transform of G_k
		\item
	
	\end{enumerate}
\end{soln}

\end{section}