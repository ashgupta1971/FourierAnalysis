\begin{section}{Metric Spaces}

	In this brief section we define metric spaces and
	Banach spaces and prove some related results.
	
%%%%%%%%%%%%%%%%%%%%%%%%%%%%%%%%%%%%%%%%%%%%%%%%%%%%%%%%%%%%
%% Defn of metric and Banach space
%%%%%%%%%%%%%%%%%%%%%%%%%%%%%%%%%%%%%%%%%%%%%%%%%%%%%%%%%%%%

\begin{defn}
	A set $X$ together with a function $d(x,y):X \rightarrow
	[0,\infty), x,y \in X$
	is called a \emph{metric space} provided the function $d$
	satisfies
		\begin{enumerate}[i)]
			\item
				$d(x,x) = 0$
			\item
				$d(x,y) = d(y,x)$
			\item
				$d(x,y)+d(y,z) \geq d(x,z)$
		\end{enumerate}
	This function $d$ is called a \emph{metric} on $X$. If,
	in addition, $X$ is complete with respect to $d$, then
	$X$ is called a \emph{Banach space}.
\end{defn}

%%%%%%%%%%%%%%%%%%%%%%%%%%%%%%%%%%%%%%%%%%%%%%%%%%%%%%%%%%%%
%% Metric/Banach space example
%%%%%%%%%%%%%%%%%%%%%%%%%%%%%%%%%%%%%%%%%%%%%%%%%%%%%%%%%%%%

\begin{ex}\label{ex:MetricInducedFromNorm}
	Let $X$ be a normed space and define $d(x,y) = \norm[]{x-y}$
	for $x,y \in X$. Show that $d$ is a metric on $X$ (it is called
	the \emph{metric induced by the norm $\norm[]{\cdot}$}).
	Furthermore, if $X$ is complete with respect to $d$ (ie. it is
	a Banach space), and $\{x_k\}_{k=1}^\infty$ is a sequence in $X$
	for which $\sum_{k=1}^\infty \norm[]{x_k}$ converges, then
	$\sum_{k=1}^\infty x_k$ also converges.
\end{ex}

\begin{soln}
	If $x \in X$ then $d(x,x) = \norm[]{x-x} = \norm[]{0} = 0$. Also,
	if $x,y \in X$, then $d(x,y)=\norm[]{x-y}=\norm[]{y-x}=d(y,x)$.
	Finally, by the Triangle Inequality, if $x,y,z \in X$,
		\begin{IEEEeqnarray*}{rCl}
			d(x,y)+d(y,z) & = & \norm[]{x-y}+\norm[]{y-z} \\
			& \geq & \norm[]{(x-y)+(y-z)} \\
			& = & \norm[]{x-z} \\
			& = & d(x,z)
		\end{IEEEeqnarray*}
	Thus $d$ is indeed a metric. For the last part of the solution,
	let
		\begin{displaymath}
			S_n = \sum_{k=1}^n x_k
		\end{displaymath}
	for $n \geq 1$. Now, given $\epsilon > 0$ choose $N > 0$ 
	such that if $q > p > N$ we	have,
		\begin{displaymath}
			\sum_{k=p+1}^q \norm[]{x_k} < \epsilon
		\end{displaymath}
	This can be done since $\sum_{k=1}^\infty \norm[]{x_k}$ converges
	and hence is Cauchy. Then, by the Triangle Inequality, 
		\begin{displaymath}
			d(S_q,S_p) = \norm[]{S_q-S_p}
				= \norm[]{\sum_{k=p+1}^q x_k}
				\leq \sum_{k=p+1}^q \norm[]{x_k} 
				< \epsilon
		\end{displaymath}
	So the series $\sum_{k=1}^\infty x_k$ is Cauchy in $X$. Since $X$ is
	complete, this implies this series converges.
\end{soln}

%%%%%%%%%%%%%%%%%%%%%%%%%%%%%%%%%%%%%%%%%%%%%%%%%%%%%%%%%%%%
%% Defn of bounded metric
%%%%%%%%%%%%%%%%%%%%%%%%%%%%%%%%%%%%%%%%%%%%%%%%%%%%%%%%%%%%

\begin{defn}
	A metric $d$ on a set $X$ is called a 
	\emph{bounded metric} provided there exists $M > 0$ 
	such that $d(x,y) < M$ for all $x,y \in X$.
\end{defn}

%%%%%%%%%%%%%%%%%%%%%%%%%%%%%%%%%%%%%%%%%%%%%%%%%%%%%%%%%%%%
%% Basic result of bounded metrics
%%%%%%%%%%%%%%%%%%%%%%%%%%%%%%%%%%%%%%%%%%%%%%%%%%%%%%%%%%%%

\begin{prop}\label{prop:BoundedMetric}
	Let $g$ be a metric on a set $X$ and define
		\begin{displaymath}
			d(x,y) = \frac{g(x,y)}{1+g(x,y)}
		\end{displaymath}
	for all $x,y \in X$. Then $d$ is a bounded metric on $X$.
\end{prop}

\begin{proof}
	First we will show that $d$ is indeed a metric.
		\begin{enumerate}[i)]
			\item
				For all $x \in X$,
					\begin{displaymath}
						d(x,x) = \frac{g(x,x)}{1+g(x,x)} = 0
					\end{displaymath}
			\item
				For all $x,y \in X$,
					\begin{displaymath}
						d(x,y)=\frac{g(x,y)}{1+g(x,y)}
							=\frac{g(y,x)}{1+g(y,x)}
							=d(y,x)
					\end{displaymath}
			\item
				For all $x,y,z \in X$,
					\begin{IEEEeqnarray*}{rCl}
						d(x,z) & = & \frac{g(x,z)}{1+g(x,z)} \\
						& \leq & \frac{g(x,y)+g(y,z)}{1+g(x,y)+g(y,z)} \\
						& = & \frac{g(x,y)}{1+g(x,y)+g(y,z)}
							+ \frac{g(y,z)}{1+g(x,y)+g(y,z)} \\
						& \leq & \frac{g(x,y)}{1+g(x,y)}
							+ \frac{g(y,z)}{1+g(y,z)} \\
						& = & d(x,y)+d(y,z)
					\end{IEEEeqnarray*}
		\end{enumerate}
	Finally it is easily seen that $d$ is bounded above by 1.
\end{proof}

%%%%%%%%%%%%%%%%%%%%%%%%%%%%%%%%%%%%%%%%%%%%%%%%%%%%%%%%%%%%
%% Defn of equivalent metrics
%%%%%%%%%%%%%%%%%%%%%%%%%%%%%%%%%%%%%%%%%%%%%%%%%%%%%%%%%%%%

\begin{defn}
	Two metrics $d$ and $g$ on a set $X$ are called
	\emph{equivalent} provided there exist positive constants
	$C_1$ and $C_2$ such that $C_1 g(x,y) \leq d(x,y) \leq
	C_2 g(x,y)$ for all $x,y \in X$.
\end{defn}

	Readers familiar with general topology will know that
	equivalent metrics generate identical topologies on $X$.

%%%%%%%%%%%%%%%%%%%%%%%%%%%%%%%%%%%%%%%%%%%%%%%%%%%%%%%%%%%%
%% Basic theorem on equivalent metrics
%%%%%%%%%%%%%%%%%%%%%%%%%%%%%%%%%%%%%%%%%%%%%%%%%%%%%%%%%%%%
	
\begin{prop}
	Let $g$ be a metric on a set $X$ and define
		\begin{displaymath}
			d(x,y) = \frac{g(x,y)}{1+g(x,y)}
		\end{displaymath}
	for all $x,y \in X$. Then $d$ and $g$ are equivalent
	metrics on $X$.
\end{prop}

\begin{proof}
	It is clear that $d(x,y) \leq g(x,y)$ for all $x,y \in X$.
	We will show that $g(x,y) \leq 2d(x,y)$ to complete the
	proof. This will be done in steps.
		\begin{enumerate}[{Step} 1]
			
			\item
				For $\isreal, x \geq 0$, let
					\begin{displaymath}
						f(x)=\frac{x}{1+x}-2x
					\end{displaymath}
				Then
					\begin{displaymath}
						f'(x)=\frac{1}{(1+x)^2}-2
					\end{displaymath}
				and
					\begin{displaymath}
						f''(x)=-\frac{2}{(1-x)^3}
					\end{displaymath}
				Hence 
					and
					\begin{displaymath}
						f'''(x)=\frac{6}{(1-x)^4}
					\end{displaymath}
		\end{enumerate}
\end{proof}

%%%%%%%%%%%%%%%%%%%%%%%%%%%%%%%%%%%%%%%%%%%%%%%%%%%%%%%%%%%%
%% Final theorem on bounded metrics
%%%%%%%%%%%%%%%%%%%%%%%%%%%%%%%%%%%%%%%%%%%%%%%%%%%%%%%%%%%%

\begin{thrm}\label{thrm:SumOfMetrics}
	Let $X$ be a non-empty set and $\{d_k\}_{k=1}^\infty$ a
	sequence of metrics on $X$. Define
		\begin{displaymath}
			d(x,y)=\sum_{k=1}^\infty \frac{1}{2^k} \frac
				{d_k(x,y)}{1+d_k(x,y)}
		\end{displaymath}
	for all $x,y \in X$. Then
		\begin{enumerate}[i)]
			\item
				$d$ is a metric on $X$.
			\item
				A sequence $\{x_n\}_{n=1}^\infty$ in $X$ converges
				to some $x_0 \in X$ with respect to $d$ iff that 
				sequence also converges to $x_0$ with respect to each $d_k$.
			\item
				$X$ is complete with respect to $d$ iff it is complete
				with respect to each $d_k$.
		\end{enumerate}
\end{thrm}

\begin{proof}
	\begin{enumerate}[i)]
		
		\item
			All the properties of a metric can be proven for $d$ in
			a similar fashion as they were in Proposition
			\ref{prop:BoundedMetric}.
		
		\item
			Suppose $\{x_n\}_{n=1}^\infty$ does not converge to $x_0$
			with respect to some $d_K$. This implies for some $\epsilon > 0$
			there is a subsequence $\{x_{n_j}\}_{j=1}^\infty$ such that
			$d_K(x_{n_j},x_0) > \epsilon$ for all $j \geq 1$. Hence
				\begin{displaymath}
					d(x_{n_j},x_0)=\sum_{k=1}^\infty \frac{1}{2^k} \frac
						{d_k(x_{n_j},x_0)}{1+d_k(x_{n_j},x_0)}
						\geq \frac{1}{2^K} d_K(x_{n_j},x_0)
						> \frac{\epsilon}{2^K}
				\end{displaymath}
			which means $\{x_n\}_{n=1}^\infty$ doesn't converge to $x_0$ with
			respect to $d$.
			
			Conversely, suppose $\{x_n\}_{n=1}^\infty$ converges to $x_0$ with
			respect to each $d_k$. Given $\epsilon > 0$ choose $N' > 0$ such
			that
				\begin{displaymath}
					\sum_{k=N'+1}^\infty \frac{1}{2^k} < \epsilon/2
				\end{displaymath}
			Next choose $N > 0$ such that for all $1 \leq k \leq N'$,
				\begin{displaymath}
					d_k(x_n,x_0) < \frac{\epsilon 2^{k-1}}{N'}
				\end{displaymath}
			if $n > N$. This can be done because $\{x_n\}_{n=1}^\infty$ converges
			to $x_0$ with respect to each $d_k$. Then if $n > N$,
				\begin{IEEEeqnarray*}{rCl}
					d(x_n,x_0) & = & \sum_{k=1}^\infty \frac{1}{2^k} \frac
						{d_k(x_n,x_0)}{1+d_k(x_n,x_0)} \\
					& = & \sum_{k=1}^{N'} \frac{1}{2^k} \frac{d_k(x_n,x_0)}
						{1+d_k(x_n,x_0)}
						+ \sum_{k=N'+1}^\infty \frac{1}{2^k} \frac
						{d_k(x_n,x_0)}{1+d_k(x_n,x_0)} \\
					& < & \sum_{k=1}^{N'} \frac{1}{2^k} \frac{\epsilon 2^{k-1}}{N'}
						+ \sum_{k=N'+1}^\infty \frac{1}{2^k} \\
					& < & \frac{\epsilon}{2}+\frac{\epsilon}{2} \\
					& = & \epsilon
				\end{IEEEeqnarray*}
			Hence $\{x_n\}_{n=1}^\infty$ converges to $x_0$ with respect to $x_0$.
			
		\item
			An argument similar to that in ii) can be used to show that sequences
			are Cauchy with respect to $d$ iff they are Cauchy with respect to each
			$d_k$. Therefore if $d$ is complete and $\{x_n\}_{n=1}^\infty$ is
			Cauchy with respect to $d$ then this sequence converges to some $x_0 \in
			X$ with respect to $d$. But by part ii), this implies it converges to
			$x_0$ with respect to each $d_k$.
	%%%%%%%%%%%%%%%%%%%%%%%%%% INCOMPLETE!!!
	
	\end{enumerate}
\end{proof}

\end{section}