\begin{section}{Fubini's Theorem}

The main result of this section is Fubini's Theorem
which deals with the order of integration of an integral.
One the consequences of Fubini's Theorem is a result due
to Leibniz, which will be proven in the next section.

%%%%%%%%%%%%%%%%%%%%%%%%%%%%%%%%%%%%%%%%%%%%%%%%%%
%% Version 1 of Fubini's Theorem
%%%%%%%%%%%%%%%%%%%%%%%%%%%%%%%%%%%%%%%%%%%%%%%%%%

\begin{thrm}[Fubini. Version 1]\label{thrm:Fubini1}
Suppose $F:[a,b] \times [c,d] \to \cmplx$ is continuous and
define
	\begin{displaymath}
	\varphi(x) = \myinta{c}{d}{F(x,y)}{y} , \; a \leq x \leq b
	\end{displaymath}
and
	\begin{displaymath}
	\psi(y) = \myinta{a}{b}{F(x,y)}{x} , \; c \leq y \leq d
	\end{displaymath}
Then,
	\begin{enumerate}[i)]
	\item
	$\varphi$ and $\psi$ are both continuous on their
	domain.
	
	\item
	\begin{displaymath}
	\myinta{a}{b}{\varphi(x)}{x} = \myinta{c}{d}{\psi(y)}{y}
	\end{displaymath}
	\end{enumerate}
\end{thrm}

\begin{proof}
	\begin{enumerate}[i)]
	%%%%%%%% Continuity
	\item
	Since $F$ is continuous on the compact set $[a,b] \times [c,d]$, it
	is uniformly continuous. So given $\epsilon > 0$ we can choose
	$\delta > 0$ such that $\modulus{F(x,y) - F(s,t)} <
	\epsilon/(d-c)$ whenever $\norm[]{(x,y) - (s,t)} < \delta$. Thus,
	if $x,s \in [a,b]$ and $\modulus{x - s} < \delta$, we have
		\begin{IEEEeqnarray*}{rCl}
		\modulus{\varphi(x) - \varphi(s)}
		& = & \modulus{\myinta{c}{d}{F(x,y)}{y}
		- \myinta{c}{d}{F(s,y)}{y}} \\
		& \leq & \myinta{c}{d}{\modulus{F(x,y) - F(s,y)}}{y} \\
		& \leq & \myinta{c}{d}{\epsilon/(d-c)}{y} \text{ since }
		\norm[]{(x,y) - (s,y)} < \delta \\
		& = & \epsilon
		\end{IEEEeqnarray*}
	Therefore, $\varphi$ is continuous on $[a,b]$. A similar argument
	shows that $\psi$ is continuous on $[c,d]$.
	
	%%%%%%%%%%%%% Equality of iterated integrals
	\item
	Without loss of generality, we may assume $F$ is real-valued. The
	general case can be handled by applying this special case to the
	real and imaginary parts of $F$.
	
	Choose $\delta > 0$ such that $\modulus{F(x,y) - F(s,t)} < \epsilon/(b-a)(d-c)$
	whenever $\norm[]{(x,y) - (s,t)} < \delta$. Choose partitions
	$a = x_0 < x_1 < \ldots < x_m = b$ of $[a,b]$ and $c = y_0 < y_1 <
	\ldots < y_n = d$ of $[c,d]$ such that $\modulus{x_j - x_{j-1}} <
	\delta/\sqrt{2}$ for $1 \leq j \leq m$ and $\modulus{y_k - y_{k-1}} <
	\delta/\sqrt{2}$ for $1 \leq k \leq n$. Then,
		\begin{IEEEeqnarray*}{rCl}
		\myinta{a}{b}{\varphi(x)}{x}
		& = & \sum_{j=1}^m \myinta{x_{j-1}}{x_j}{\varphi(x)}{x} \\
		& = & \sum_{j=1}^m \myinta{x_{j-1}}{x_j}{\myinta{c}{d}{F(x,y)}{y}}{x} \\
		& = & \sum_{j=1}^m \myinta{x_{j-1}}{x_j}{\sum_{k=1}^n
		\myinta{y_{k-1}}{y_k}{F(x,y)}{y}}{x} \\
		& = & \sum_{j=1}^m \sum_{k=1}^n \myinta{x_{j-1}}{x_j}
		{\myinta{y_{k-1}}{y_k}{F(x,y)}{y}}{x} \\
		& \leq & \sum_{j=1}^m \sum_{k=1}^n M_{jk}(x_j - x_{j-1})(y_k - y_{k-1})
		\end{IEEEeqnarray*}
	Where, $M_{jk} = \max \{F(x,y):(x,y) \in [x_{j-1},x_j] \times [y_{k-1},y_k]\}$.
	Denote this latter sum by $U$.
		
	Similarly,
		\begin{displaymath}
		\myinta{a}{b}{\varphi(x)}{x}
		\geq \sum_{j=1}^m \sum_{k=1}^n m_{jk}(x_j - x_{j-1})(y_k - y_{k-1})
		= L
		\end{displaymath}
	where $m_{jk} = \min \{F(x,y):(x,y) \in [x_{j-1},x_j] \times [y_{k-1},y_k]\}$.
	
	A similar argument shows that	
		\begin{displaymath}
		L \leq \myinta{c}{d}{\psi(y)}{y} \leq U
		\end{displaymath}

	But, by the choice of the $x_j$'s and $y_k$'s the diameter of $[x_{j-1},x_j]
	\times [y_{k-1},y_k]$ is less than $\delta$. So $M_{jk} - m_{jk}
	\leq \epsilon/(b-a)(d-c)$.
			
	So we have,
		\begin{IEEEeqnarray*}{rCl}
		\modulus{\myinta{a}{b}{\varphi(x)}{x}
		- \myinta{c}{d}{\psi(y)}{y}}
		& \leq & U - L \\
		& = & \sum_{j=1}^m \sum_{k=1}^n 
		(M_{jk} - m_{jk})(x_j - x_{j-1})(y_k - y_{k-1}) \\
		& \leq & \sum_{j=1}^m \sum_{k=1}^n 
		\frac{\epsilon}{(b-a)(d-c)} (x_j - x_{j-1})(y_k - y_{k-1}) \\
		& = & \epsilon
		\end{IEEEeqnarray*}
	
	Since this is true for all $\epsilon > 0$, 
		\begin{displaymath}
		\myinta{a}{b}{\varphi(x)}{x} =
		\myinta{c}{d}{\psi(y)}{y}
		\end{displaymath}
	\end{enumerate}
\end{proof}

%%%%%%%%%%%%%%%%%%%%%%%%%%%%%%%%%%%%%%%%%%%%%%%%%%
%% Version 2 of Fubini's Theorem
%%%%%%%%%%%%%%%%%%%%%%%%%%%%%%%%%%%%%%%%%%%%%%%%%%

A natural question to ask is whether Theorem
\ref{thrm:Fubini1} holds when one or both of the
intervals $[a,b]$ and $[c,d]$ are replaced with
unbounded intervals. The next two versions of
Fubini's theorem deal with that question and generalize
it further.

\begin{thrm}[Fubini. Version 2]\label{thrm:Fubini2}
	Suppose that $X \subset \real^m$ and $X$ is either
	compact or open. Also suppose that $F:X \times \real \to
	\cmplx$ is continuous and there exists $h \in \absint$
	such that $\modulus{F(x,t)} \leq \modulus{h(t)}$ for all
	$(x,t) \in X \times \real$.
	
	\begin{enumerate}[i)]
	\item
	The function $\varphi:X \to \cmplx$ defined by
		\begin{displaymath}
		\varphi(x) = \myintb{F(x,t)}{t}
		\end{displaymath}
	is continuous on its domain.
	
	\item
	If $m = 1$ and $[a,b] \subset X$ and we define $\psi:
	\real \to \cmplx$ by
		\begin{displaymath}
		\psi(t) = \myinta{a}{b}{F(x,t)}{x}
		\end{displaymath}
	then $\psi$ is continuous and $\psi \in \absint$.
	
	\item
	The functions $\varphi$ and $\psi$ satisfy
		\begin{displaymath}
		\myinta{a}{b}{\varphi(x)}{x} =
		\myintb{\psi(t)}{t}
		\end{displaymath}
	\end{enumerate}
\end{thrm}

\begin{proof}
		\begin{enumerate}[i)]
		%%%%%%%%%%%%% Continuity
		\item
		Let
			\begin{displaymath}
			\varphi_n(x) = \myintb[n]{F(x,t)}{t}
			\end{displaymath}
		for $x \in X$ and $n \in \natrl$.
		
		To show that $\varphi_n$ is continuous, proceed in
		a manner similar to that in Theorem \ref{thrm:Fubini1}.
		Choose a compact set $A \subset X$. Then $A \times
		[-n,n]$ is compact. Therefore, $F$ is uniformly continuous
		on $A \times [-n,n]$. So given $\epsilon > 0$ there exists
		$\delta > 0$ such that $\modulus{F(x,s) - F(y,t)} <
		\epsilon/2n$ whenever $(x,s), (y,t) \in A \times [-n,n]$ and
		$\norm[]{(x,s) - (y,t)} < \delta$. Then,
			
			\begin{IEEEeqnarray*}{rCl}
				\modulus{\varphi_n(x) - \varphi_n(y)} & = &
					\modulus{
					\myintb[n]{F(x,t)}{t} - \myintb[n]{F(y,t)}{t}} \\
				& \leq & \myintb[n]{\modulus{
					F(x,t) - F(y,t)}}{t} \\
				& \leq & \myintb[n]{\epsilon/2n}{t} \\
				= \epsilon
			\end{IEEEeqnarray*}
		whenever $x, y \in A$ and $\norm[]{x - y} < \delta$. So
		$\varphi_n$ is continuous on $A$ and hence on its entire
		domain.
		
		To show that $\varphi$ is continuous, 
			\begin{IEEEeqnarray*}{rCl}
				\modulus{\varphi(x) - \varphi_n(x)} & = &
				\modulus{\myintb{F(x,t)}{t}
				- \myintb[n]{F(x,t)}{t}} \\
				& = & \modulus{\myinta{-\infty}{-n}{F(x,t)}{t}
				+ \myinta{n}{\infty}{F(x,t)}{t}} \\
				& \leq & \myinta{-\infty}{-n}{\modulus{F(x,t)}}{t}
				+ \myinta{n}{\infty}{\modulus{F(x,t)}}{t} \\
				& \leq & \myinta{-\infty}{-n}{\modulus{h(t)}}{t}
				+ \myinta{n}{\infty}{\modulus{h(t)}}{t} \\
				& = & \myintb{\modulus{h(t)}}{t}
				- \myintb[n]{\modulus{h(t)}}{t}
			\end{IEEEeqnarray*}
		But
			\begin{displaymath}
				\myintb{\modulus{h(t)}}{t}
				= \lim_{n \rightarrow \infty}
				\myintb[n]{\modulus{h(t)}}{t}
			\end{displaymath}
		It follows that $\{\varphi_n\}$ converges uniformly to
		$\varphi$ on $X$. Therefore $\varphi$ is continuous since
		each $\varphi_n$ is.
		
		%%%%%%%%%%%%% Continuity, absolute integrability
		\item
		By Theorem \ref{thrm:Fubini1}, $\psi$ is continuous
		on bounded intervals. It follows that $\psi$ is 
		continuous on its entire domain.
		
		To show that $\psi \in \absint$, we have
			\begin{IEEEeqnarray*}{rCl}
				\myintb[n]{\modulus{\psi(t)}}{t}
				& = & \myintb[n]{\modulus{\myinta{a}{b}{F(x,t)}{x}}}{t} \\
				& \leq & \myintb[n]{\left( \myinta{a}{b}{\modulus{F(x,t)}}{x}
				\right)}{t} \\
				& = & \myinta{a}{b}{\left( \myintb[n]{\modulus{F(x,t)}}{t}
				\right)}{x} \\
				& \leq & \myinta{a}{b}{\left( \myintb[n]{\modulus{h(t)}}{t}
				\right)}{x} \\
				& \leq & (b-a) \myintb{\modulus{h(t)}}{t} \\
				& < & \infty
			\end{IEEEeqnarray*}
		since $h \in \absint$.
		
		%%%%%%%%%%%%% Equality of iterated integrals
		\item
		To show the equality of
			\begin{displaymath}
				\myinta{a}{b}{\varphi(x)}{x} = 
				\myintb{\psi(t)}{t}
			\end{displaymath}
			
		first note that
			\begin{IEEEeqnarray*}{rCl}
				\myintb{\psi(t)}{t}
				& = & \lim_{n \rightarrow +\infty}
				\myintb[n]{\psi(t)}{t} \\
				& = & \lim_{n \rightarrow +\infty}
				\myintb[n]{\left( \myinta{a}{b}{F(x,t)}{x} \right)}{t} \\
				& = & \lim_{n \rightarrow +\infty}
				\myinta{a}{b}{\left( \myintb[n]{F(x,t)}{t} \right)}{x}
			\end{IEEEeqnarray*}
			
		Therefore,
			\begin{IEEEeqnarray*}{l}
				\modulus{
					\myinta{a}{b}{\varphi(x)}{x} - 
					\myintb[n]{\psi(t)}{t}} \\
				= \modulus{
					\myinta{a}{b}{\varphi(x)}{x} - 
					\myinta{a}{b}{\left( \myintb[n]{F(x,t)}{t} \right)}{x}} \\
				= \modulus{
					\myinta{a}{b}{\left( \myintb{F(x,t)}{t} \right)}{x}
					- \myinta{a}{b}{\left( \myintb[n]{F(x,t)}{t} \right)}{x}} \\
				= \modulus{
					\myinta{a}{b}{\left( \myintb{F(x,t)}{t} -
					\myintb[n]{F(x,t)}{t} \right)}{x}} \\
				\leq \myinta{a}{b}{\modulus{\myintb{F(x,t)}{t} -
					\myintb[n]{F(x,t)}{t}}}{x} \\
				= \myinta{a}{b}{\modulus{ \myinta{-\infty}{-n}{F(x,t)}{t} +
					\myinta{n}{\infty}{F(x,t)}{t}}}{x} \\
				\leq \myinta{a}{b}{\left( \myinta{-\infty}{-n}{\modulus{F(x,t)}}{t} +
					\myinta{n}{\infty}{\modulus{F(x,t)}}{t} \right)}{x} \\
				\leq \myinta{a}{b}{\left( \myinta{-\infty}{-n}{\modulus{h(t)}}{t} +
					\myinta{n}{\infty}{\modulus{h(t)}}{t} \right)}{x} \\
				= \myinta{a}{b}{\left( \myintb{\modulus{h(t)}}{t} -
					\myintb[n]{\modulus{h(t)}}{t} \right)}{x} \\
				= (b-a) \left( \myintb{\modulus{h(t)}}{t} -
					\myintb[n]{\modulus{h(t)}}{t} \right)
			\end{IEEEeqnarray*}
		$\rightarrow 0$ as $n \rightarrow +\infty$.
			
		Thus,
			\begin{displaymath}
				\myinta{a}{b}{\varphi(x)}{x} = 
				\myintb{\psi(t)}{t}
			\end{displaymath}
		\end{enumerate}
\end{proof}

%%%%%%%%%%%%%%%%%%%%%%%%%%%%%%%%%%%%%%%%%%%%%%%%%%
%% Version 3 of Fubini's Theorem
%%%%%%%%%%%%%%%%%%%%%%%%%%%%%%%%%%%%%%%%%%%%%%%%%%

\begin{thrm}[Fubini. Version 3]\label{thrm:Fubini3}
	Suppose $F:\real^2 \to \cmplx$ is continuous and there
	exist $h, k \in \absint$ such that $\modulus{F(x,t)} \leq
	h(x)k(t)$ for all $x,t \in \real$. Define
		
		\begin{displaymath}
			\varphi(x) = \myintb{F(x,t)}{t} , \; \isreal
		\end{displaymath}
	and	
		\begin{displaymath}
			\psi(t) = \myintb{F(x,t)}{x} , \; \isreal[t]
		\end{displaymath}
		
	Then,
	
	\begin{enumerate}[i)]
	
	\item
	$\varphi$ and $\psi$ are well-defined.
	
	\item
	$\varphi$ and $\psi$ are absolutely integrable on $\real$.

	\item
	$\varphi$ and $\psi$ are continuous.

	\item
		\begin{displaymath}
			\myintb{\varphi(x)}{x}
			= \myintb{\psi(t)}{t}
		\end{displaymath}
	
	\end{enumerate}
\end{thrm}

\begin{proof}
	
	\begin{enumerate}[i)]
	%%%%%%%%%%% Well-defined
	\item
	To see that $\varphi$ is well-defined,
		\begin{IEEEeqnarray*}{rCl}
			\modulus{\varphi(x)} & = &
				\modulus{\myintb{F(x,t)}{t}} \\
			& \leq & \myintb{\modulus{F(x,t)}}{t} \\
			& \leq & \myintb{\modulus{
				h(x)}\modulus{k(t)}}{t} \\
			& = & \modulus{h(x)} \myintb{\modulus{k(t)}}{t} \\
			& = & \modulus{h(x)} \norm[1]{k}
		\end{IEEEeqnarray*}
	Similarly, $\modulus{\psi(t)} \leq \modulus{k(t)}
	\norm[1]{h}$.
	
	%%%%%%%%%%%%%% Absolute integrability
	\item
	The above remarks also show that $\varphi$ and $\psi$
	are absolutely integrable on $\real$ because,
		\begin{displaymath}
			\modulus{\myintb{\varphi(x)}{x}}
				\leq \myintb{\modulus{\varphi(x)}}{x}
				\leq \myintb{\modulus{h(x)}
				\norm[1]{k}}{x}
				= \norm[1]{h} \norm[1]{k}
		\end{displaymath}
	and similarly,
		\begin{displaymath}
			\modulus{\myintb{\psi(t)}{t}}
				\leq \myintb{\modulus{\psi(t)}}{t}
				\leq \myintb{\modulus{k(t)}
				\norm[1]{h}}{t}
				= \norm[1]{k} \norm[1]{h}
		\end{displaymath}
	
	%%%%%%%%%%%%%%% Continuity
	\item
	To show that $\varphi$ is continuous, let
		\begin{displaymath}
			\varphi_n(x) = \myintb[n]{F(x,t)}{t}
		\end{displaymath}
	Then, $\varphi_n$ is continuous by Theorem
	\ref{thrm:Fubini2}.
	
	So we have,
		\begin{IEEEeqnarray*}{rCl}
			\modulus{\varphi(x) - \varphi_n(x)} & = &
				\modulus{\myintb{F(x,t)}{t}
				- \myintb[n]{F(x,t)}{t}} \\
			& = & \modulus{\myinta{-\infty}{-n}{F(x,t)}{t}
				+ \myinta{n}{\infty}{F(x,t)}{t}} \\
			& \leq & \myinta{-\infty}{-n}{\modulus{F(x,t)}}{t}
				+ \myinta{n}{\infty}{\modulus{F(x,t)}}{t} \\
			& \leq & \myinta{-\infty}{-n}{\modulus{h(x)}
				\modulus{k(t)}}{t}
				+ \myinta{n}{\infty}{\modulus{h(x)}
				\modulus{k(t)}}{t} \\
			& = & \modulus{h(x)} \left( \myintb{\modulus{k(t)}}{t}
				- \myintb[n]{\modulus{k(t)}}{t} \right)
		\end{IEEEeqnarray*}
	$\rightarrow 0$ as $n \rightarrow +\infty$ uniformly on bounded
	intervals. This is because $k \in \absint$ and $h$ is bounded
	on bounded intervals (because $h \in \absint$). Thus, the sequence of
	continuous functions $\{\varphi_n\}$ converges uniformly to 
	$\varphi$ on each bounded interval. Hence $\varphi$ is continuous.
	
	A similar argument shows that $\psi$ is continuous.
	
	%%%%%%%%%%%%%%%%%%%%% Equality of iterated integrals
	\item
	\begin{IEEEeqnarray*}{l}
		\modulus{
			\myintb{\varphi(x)}{x}
			- \myintb[n]{\left(
			\myintb[n]{F(x,t)}{t} \right)}{x}} \\
		\leq \modulus{
			\myintb{\varphi(x)}{x}
			- \myintb[n]{\left(
			\myintb{F(x,t)}{t} \right)}{x}} \\
		+ \; \modulus{
			\myintb[n]{\left(
			\myintb{F(x,t)}{t} \right)}{x}
			- \myintb[n]{\left(
			\myintb[n]{F(x,t)}{t} \right)}{x}}
	\end{IEEEeqnarray*}
	
	Now,
		\begin{IEEEeqnarray*}{l}
			\modulus{\myintb{\varphi(x)}{x}
				- \myintb[n]{\left(
				\myintb{F(x,t)}{t} \right)}{x}} \\
			= \modulus{\myintb{\varphi(x)}{x}
				- \myintb[n]{\varphi(x)}{x}}
		\end{IEEEeqnarray*}
	$\rightarrow 0$ as $n \rightarrow +\infty$.
	
	Also,
		\begin{IEEEeqnarray*}{l}
			\modulus{
				\myintb[n]{\left(
				\myintb{F(x,t)}{t} \right)}{x}
				- \myintb[n]{\left(
				\myintb[n]{F(x,t)}{t} \right)}{x}} \\
			\leq \myintb[n]{\modulus{
				\myinta{-\infty}{-n}{F(x,t)}{t}
				+ \myinta{n}{\infty}{F(x,t)}{t}}}{x} \\
			\leq \myintb[n]{\left( \myinta{-\infty}{-n}
				{\modulus{F(x,t)}}{t}
				+ \myinta{n}{\infty}{\modulus{F(x,t)}}{t}
				\right)}{x} \\
			\leq \myintb[n]{\left( \myinta{-\infty}{-n}
				{\modulus{h(x)}\modulus{k(t)}}{t}
				+ \myinta{n}{\infty}{\modulus{h(x)}\modulus{k(t)}}{t}
				\right)}{x} \\
			= \left( \myintb[n]{\modulus{h(x)}}{x} \right)
				\left(
				\myintb{\modulus{k(t)}}{t}
				- \myintb[n]{\modulus{k(t)}}{t} \right) \\
			\leq \norm[1]{h} \left(
				\myintb{\modulus{k(t)}}{t}
				- \myintb[n]{\modulus{k(t)}}{t} \right)
		\end{IEEEeqnarray*}
	$\rightarrow 0$ as $n \rightarrow +\infty$.
	
	Thus, 
		\begin{displaymath}
			\myintb{\varphi(x)}{x}
				= \lim_{n \rightarrow +\infty}
				\myintb[n]{\left(
				\myintb[n]{F(x,t)}{t} \right)}{x}
		\end{displaymath}
	and similarly,
		\begin{displaymath}
			\myintb{\psi(t)}{t}
				= \lim_{n \rightarrow +\infty}
				\myintb[n]{\left(
				\myintb[n]{F(x,t)}{x} \right)}{t}
		\end{displaymath}
	But,
		\begin{displaymath}
			\myintb[n]{\left(
				\myintb[n]{F(x,t)}{t} \right)}{x}
			= \myintb[n]{\left(
				\myintb[n]{F(x,t)}{x} \right)}{t}
		\end{displaymath}
	so the result follows.
	\end{enumerate}
\end{proof}

\end{section}
