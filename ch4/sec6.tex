\begin{section}{The Fourier Integral Theorem}

	In this section we present two versions of
	the Fourier Integral Theorem.
	
%%%%%%%%%%%%%%%%%%%%%%%%%%%%%%%%%%%%%%%%%%%%%%%%%%%%%%%%
%% Preliminary Lemma
%%%%%%%%%%%%%%%%%%%%%%%%%%%%%%%%%%%%%%%%%%%%%%%%%%%%%%%%

\begin{lemma}\label{lemma:FIT}
	Suppose $f \in \absint$, $\isreal[a]$, $g(t)=f(t+a)$
	and $h(t)=f(-t)$ for $\isreal[t]$. Then $g,h \in
	\absint$,
		\begin{displaymath}
			\myintb{g(t)}{t}=\myintb{h(t)}{t}=\myintb{f(t)}{t}
		\end{displaymath}
	and $\norm[1]{g}=\norm[1]{h}=\norm[1]{f}$.
\end{lemma}

\begin{proof}
	If $A > 0$ then
		\begin{IEEEeqnarray*}{rCl}
			\myintb[A]{\modulus{g(t)}}{t} & = &
				\myintb[A]{\modulus{f(t+a)}}{t} \\
			& = & \myinta{a-A}{a+A}{\modulus{f(s)}}{s}
				\; \;(s=t+a) \\
			& \rightarrow & \myintb{\modulus{f(t)}}{t} \\
			& = & \norm[1]{f}
		\end{IEEEeqnarray*}
	Hence $g \in \absint$ and $\norm[1]{g}=\norm[1]{f}$. A
	similar argument shows that
		\begin{displaymath}
			\myintb[A]{g(t)}{t} \rightarrow
				\myintb{f(t)}{t}
		\end{displaymath}
	as $A \rightarrow \infty$ and also that $h \in \absint$,
	$\norm[1]{h}=\norm[1]{f}$ and
		\begin{displaymath}
			\myintb{h(t)}{t} = \myintb{f(t)}{t}
		\end{displaymath}
\end{proof}

%%%%%%%%%%%%%%%%%%%%%%%%%%%%%%%%%%%%%%%%%%%%%%%%%%%%%%%%
%% Fourier Integral Thrm (Preliminary Version)
%%%%%%%%%%%%%%%%%%%%%%%%%%%%%%%%%%%%%%%%%%%%%%%%%%%%%%%%

\begin{thrm}[Fourier Integral Theorem. Preliminary Version]
\label{thrm:FIT1}
	Suppose $f \in \absint$ and for some $\isreal$ and $\delta
	> 0$ the following all exist
		\begin{enumerate}[i)]
			\item
				$f(x+)=\lim_{y \rightarrow x^+}f(y)$
			\item
				$f(x-)=\lim_{y \rightarrow x^-}f(y)$
			\item
				\begin{displaymath}
					\myinta{0}{\delta}{\frac{f(x+t)-f(x+)}{t}}{t}
				\end{displaymath}
			\item
				\begin{displaymath}
					\myinta{0}{\delta}{\frac{f(x-t)-f(x-)}{t}}{t}
				\end{displaymath}
		\end{enumerate}
	(These conditions can be met, for example, if $f$ has left and
	right derivatives at $x$.) Then
		\begin{displaymath}
			\frac{f(x+)+f(x-)}{2}=\lim_{\lambda \rightarrow \infty}
				\frac{1}{\pi}\myintb{f(u)\left(
				\myinta{0}{\lambda}{\cos v(u-x)}{v}\right)}{u}
		\end{displaymath}
\end{thrm}

\begin{proof}
	By Theorem \ref{thrm:RLL},
		\begin{displaymath}
			\lim_{\lambda \rightarrow \infty}\myinta{-\infty}{-\delta}
				{f(x+t)\frac{\sin \lambda t}{t}}{t}
				= 0
		\end{displaymath}
	and
		\begin{displaymath}
			\lim_{\lambda \rightarrow \infty}\myinta{\delta}{\infty}
				{f(x+t)\frac{\sin \lambda t}{t}}{t}
				= 0
		\end{displaymath}
	Also, by Corollary \ref{cor:Dini3},
		\begin{displaymath}
			\lim_{\lambda \rightarrow \infty}\myinta{0}{\delta}
				{f(x+t)\frac{\sin \lambda t}{t}}{t}
				= \frac{\pi}{2}f(x+)
		\end{displaymath}
	and
		\begin{displaymath}
			\lim_{\lambda \rightarrow \infty}\myinta{-\delta}{0}
				{f(x+t)\frac{\sin \lambda t}{t}}{t}
				= \frac{\pi}{2}f(x-)
		\end{displaymath}
	Hence
		\begin{IEEEeqnarray*}{rCl}
			\lim_{\lambda \rightarrow \infty}\frac{1}{\pi}
				\myintb{f(x+t)\frac{\sin \lambda t}{t}}{t}
				& = & \lim_{\lambda \rightarrow \infty}\frac{1}{\pi}
				\myinta{-\infty}{-\delta}{f(x+t)\frac{\sin \lambda t}{t}}{t} \\
			& & + \; \lim_{\lambda \rightarrow \infty}\frac{1}{\pi}
				\myinta{-\delta}{0}{f(x+t)\frac{\sin \lambda t}{t}}{t} \\
			& & + \; \lim_{\lambda \rightarrow \infty}\frac{1}{\pi}
				\myinta{0}{\delta}{f(x+t)\frac{\sin \lambda t}{t}}{t} \\
			& & + \; \lim_{\lambda \rightarrow \infty}\frac{1}{\pi}
				\myinta{\delta}{\infty}{f(x+t)\frac{\sin \lambda t}{t}}{t} \\
			& = & \frac{f(x+)+f(x-)}{2}
		\end{IEEEeqnarray*}
	But using Lemma \ref{lemma:FIT}
		\begin{IEEEeqnarray*}{rCl}
			\myintb{f(x+t)\frac{\sin \lambda t}{t}}{t} & = &
				\myintb{f(u)\frac{\sin \lambda(u-x)}{u-x}}{u}
				\; \; (u=x+t) \\
			& = & \myintb{f(u)\left(\myinta{0}{\lambda}{\cos v(u-x)}{v}
				\right)}{u}
		\end{IEEEeqnarray*}
	and so
		\begin{displaymath}
			\frac{f(x+)+f(x-)}{2} = \lim_{\lambda \rightarrow \infty}
				\frac{1}{\pi}\myintb{f(u)\left(
				\myinta{0}{\lambda}{\cos v(u-x)}{v}\right)}{u}
		\end{displaymath}
	as required.
\end{proof}

%%%%%%%%%%%%%%%%%%%%%%%%%%%%%%%%%%%%%%%%%%%%%%%%%%%%%%%%
%% Fourier Integral Thrm (Utility Grade)
%%%%%%%%%%%%%%%%%%%%%%%%%%%%%%%%%%%%%%%%%%%%%%%%%%%%%%%%

\begin{thrm}[Fourier Integral Theorem]\label{thrm:FIT2}
	Suppose $f \in \absint$, $f$ is continuous, $\isreal$ and
	for some $\delta > 0$ the following integrals exist.
		\begin{displaymath}
			\myinta{0}{\delta}{\frac{f(x+t)-f(x)}{t}}{t}
		\end{displaymath}
	and
		\begin{displaymath}
			\myinta{0}{\delta}{\frac{f(x-t)-f(x)}{t}}{t}
		\end{displaymath}
	(Again, these integrals exist if, for example, $f$ has left and
	right derivatives at $x$.) Then
		\begin{IEEEeqnarray*}{rCl}
			f(x) & = & \lim_{\lambda \rightarrow \infty}
				\myintb[\lambda]{\left(\myintb{f(u)e^{ivu}}{u}
				\right)e^{-ixv}}{v} \\
			& = & \lim_{\lambda \rightarrow \infty}
				\myintb[\lambda]{\f(-v)e^{-ixv}}{v}
		\end{IEEEeqnarray*}
\end{thrm}

\begin{proof}
	Fix $\lambda > 0$. By Theorem \ref{thrm:Fubini2},
		\begin{IEEEeqnarray*}{rCl}
			\frac{1}{\pi}\myintb{f(u)\left(
				\myinta{0}{\lambda}{\cos v(u-x)}{v}
				\right)}{u} & = &
				\frac{1}{\pi}\myinta{0}{\lambda}{\left(
				\myintb{f(u)\cos v(u-x)}{u}\right)}{v} \\
			& = & \frac{1}{2\pi}\myinta{0}{\lambda}{\left(
				\myintb{f(u)(e^{iv(u-x)}+e^{-iv(u-x)})}{u}
				\right)}{v} \\
			& = & \frac{1}{2\pi}\myinta{0}{\lambda}{\left(
				\myintb{f(u)e^{ivu}}{u}\right)e^{-ivx}}{v} \\
			& & + \; \frac{1}{2\pi}\myinta{0}{\lambda}{\left(
				\myintb{f(u)e^{-ivu}}{u}\right)e^{ivx}}{v} \\
			& = & \frac{1}{2\pi}\myinta{0}{\lambda}{\f(-v)e^{-ivx}}{v} \\
			& & + \; \frac{-1}{2\pi}\myinta{0}{-\lambda}{\left(
				\myintb{f(u)e^{iwu}}{u}\right)e^{-iwx}}{w} \; \;
				(w=-v) \\
			& = & \frac{1}{2\pi}\myinta{0}{\lambda}{\f(-v)e^{-ivx}}{v} \\
			& & + \; \frac{1}{2\pi}\myinta{-\lambda}{0}{\f(-w)e^{-iwx}}{w} \\
			& = & \frac{1}{2\pi}\myintb[\lambda]{\f(-v)e^{-ivx}}{v}
		\end{IEEEeqnarray*}
	But from Theorem \ref{thrm:FIT1}
		\begin{displaymath}
			f(x) = \lim_{\lambda \rightarrow \infty}\frac{1}{\pi}
				\myintb{f(u)\left(\myinta{0}{\lambda}{\cos v(u-x)}{v}
				\right)}{u}
		\end{displaymath}
	and so
		\begin{displaymath}
			f(x) = \lim_{\lambda \rightarrow \infty}
				\frac{1}{2\pi}\myintb[\lambda]{\f(-v)e^{-ivx}}{v}
		\end{displaymath}
\end{proof}


%%%%%%%%%%%%%%%%%%%%%%%%%%%%%%%%%%%%%%%%%%%%%%%%%%%%%%%%
%% Defn and Cor. to Fourier Integral Thrm (Utility Grade)
%%%%%%%%%%%%%%%%%%%%%%%%%%%%%%%%%%%%%%%%%%%%%%%%%%%%%%%%

\begin{defn}
	For $f \in \absint$ let $\check{f}(x)=\f(-x)$ for
	all $\isreal$ and define $\fmtransform[]:\absint
	\rightarrow \ctszeror$ by $\fmtransform = \check{f}$.
\end{defn}

\begin{cor}\label{cor:FIT2}
	\begin{enumerate}[i)]
		\item
			$\fmtransform[]$ is a linear map from $\absint$ to
			$\ctszeror$ and $\supnorm{\fmtransform} \leq 
			\norm[1]{f}$ for all $f \in \absint$.
		\item
			If $f \in \absint \cap \ctszeror$ and $\f \in
			\absint$ then
				\begin{displaymath}
					f = \frac{1}{2\pi}\ftransform[(\fmtransform)]
						= \frac{1}{2\pi}\fmtransform[(\ftransform)]
				\end{displaymath}
	\end{enumerate}
\end{cor}

\begin{proof}
	\begin{enumerate}[i)]
		\item
			Clearly $\fmtransform[]$ is linear and the second
			assertion follows from Proposition \ref{prop:FTransform}.
		\item
			This result follows from Theorem \ref{thrm:FIT2}.
	\end{enumerate}
\end{proof}

\end{section}