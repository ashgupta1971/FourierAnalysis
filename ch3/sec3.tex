\begin{section}{\fejers Kernel}

	In this section we will derive an expression
	called \fejers Kernel and then deduce some of
	its properties.
	
%%%%%%%%%%%%%%%%%%%%%%%%%%%%%%%%%%%%%%%%%%%%%%%%%%%%%%%
%% Derivation of Fejer's Kernel
%%%%%%%%%%%%%%%%%%%%%%%%%%%%%%%%%%%%%%%%%%%%%%%%%%%%%%%

	Suppose $f$ is a $2\pi$-periodic, Riemann integrable
	function and let $s_k, k \geq 0$ be the $k$th partial 
	sum of its Fourier series. Define $\sigma_n(x): \real
	\rightarrow \cmplx$ by
		\begin{displaymath}
			\sigma_n(x) = \frac{1}{n+1}\sum_{k=0}^n s_k(x)
		\end{displaymath}
	Then, we have
		\begin{IEEEeqnarray*}{rCl}
			\sigma_n(x) & = & \frac{1}{n+1}
				\sum_{k=0}^n \frac{1}{2\pi}
				\myintb[\pi]{f(x-t) D_k(t)}{t} \\
			& = & \frac{1}{2\pi}\myintb[\pi]{f(x-t)\left(
				\sum_{k=0}^n \frac{1}{n+1}D_k(t) \right)}{t} \\
			& = & \frac{1}{2\pi}\myintb[\pi]{f(x-t) F_n(t)}{t}
		\end{IEEEeqnarray*}
	where
		\begin{displaymath}
			F_n(t) = \frac{1}{n+1}\sum_{k=0}^n D_k(t)
		\end{displaymath}
	for $\isreal[t]$. We call $\{F_n\}_{n=0}^\infty$ the \emph{\fejers
	Kernel}.
	
%%%%%%%%%%%%%%%%%%%%%%%%%%%%%%%%%%%%%%%%%%%%%%%%%%%%%%%
%% Properties of the Fejer's Kernel
%%%%%%%%%%%%%%%%%%%%%%%%%%%%%%%%%%%%%%%%%%%%%%%%%%%%%%%

\begin{thrm}\label{thrm:FejersKernel}
	\begin{enumerate}[i)]
	
		\item
			For all $n \geq 0$, \fejers Kernel is a real-valued, even,
			trigonometric polynomial of "degree n".
			
		\item
			For all $n \geq 0$,
				\begin{displaymath}
					F_n(x) =
						\begin{cases}
							\displaystyle{
								\frac{1}{n+1}
								\frac{\sin^2(\frac{n+1}{2}x)}
								{\sin^2 x/2}} & \text{ if }
								\isreal, x \neq 2\pi k, \isintgr[k] \\
							n+1  & \text{ if } x = 2\pi k, \isintgr[k]
						\end{cases}
				\end{displaymath}
		
		\item
			For all $n \geq 0$ and all $\isreal$, $0 \leq F_n(x)
			\leq F_n(0) = n+1$.
			
		\item
			For all $n \geq 0$,
				\begin{displaymath}
					\myintb[\pi]{F_n(t)}{t} = 2\pi
				\end{displaymath}
				
		\item
			For $0 < \delta < \pi$ and $\isnatrl$,
				\begin{displaymath}
					0 \leq F_n(x) \leq \frac{2}{n+1}\left(
						\frac{1}{1-\cos \delta}\right)
				\end{displaymath}
			if $\isreal$ and $\delta \leq x \leq \pi$.
			
		\item
			For $0 < \delta < \pi$,
				\begin{displaymath}
					\lim_{n \rightarrow \infty}
						\myinta{\delta}{\pi}{F_n(t)}{t} = 0
						= \lim_{n \rightarrow \infty}
						\myinta{-\pi}{-\delta}{F_n(t)}{t}
				\end{displaymath}
				
		\item
			For $0 < \delta < \pi$,
				\begin{displaymath}
					\lim_{n \rightarrow \infty}
						\myintb[\delta]{F_n(t)}{t} = 2\pi
				\end{displaymath}
	
	\end{enumerate}
\end{thrm}

\begin{proof}
	\begin{enumerate}[i)]
		
		\item
			This follows from the properties of the Dirichlet kernel.
		
		\item
			For all $n \geq 0$, we have
				\begin{IEEEeqnarray*}{rCl}
					(\sin^2 x/2) F_n(x) & = & \frac{\sin^2 x/2}{2} \sum_{k=0}^n D_k(x) \\
					& = & \frac{\sin^2 x/2}{2} \sum_{k=0}^n \frac{\sin(k+1/2)x}{\sin x/2} \\
					& = & \frac{1}{n+1}\sum_{k=0}^n (\sin(k+1/2)x)(\sin x/2) \, (\ast)
				\end{IEEEeqnarray*}
			Recall that
				\begin{IEEEeqnarray*}{rCl}
					\cos(a-b)-\cos(a+b) & = & [\cos a \cos (-b) - \sin a \sin (-b)] \\
						& & - [\cos a \cos b - \sin a \sin b] \\
						& = & [\cos a \cos b + \sin a \sin b] \\
						& & - [\cos a \cos b - \sin a \sin b] \\
						& = & 2 \sin a \sin b
				\end{IEEEeqnarray*}
			Substituting into $(\ast)$, we obtain
				\begin{IEEEeqnarray*}{rCl}
					(\sin^2 x/2) F_n(x) & = & \frac{1}{n+1}\sum_{k=0}^n
						\frac{\cos kx - \cos (k+1)x}{2} \\
					& = & \frac{1 - \cos (n+1)x}{2(n+1)} \, (\dagger)
				\end{IEEEeqnarray*}
			Substituting the following identity into $(\dagger)$,
				\begin{displaymath}
					1 - \cos 2x = 2\sin^2 x
				\end{displaymath}
			we obtain
				\begin{displaymath}
					(\sin^2 x/2) F_n(x) = \frac{\sin^2 \frac{n+1}{2}x}{n+1}
				\end{displaymath}
			which proves the assertion when $x \neq 2\pi k, \isnatrl[k]$.
			When $x$ is a multiple of $2\pi$, $D_k(x) = 2k+1$, hence
				\begin{IEEEeqnarray*}{rCl}
					F_n(x) & = & \frac{1}{n+1}\sum_{k=0}^n D_k(x) \\
					& = & \frac{1}{n+1}\sum_{k=0}^n (2k+1) \\
					& = & \frac{1}{n+1}\left(2\frac{n(n+1)}{2}+(n+1)\right) \\
					& = & n+1
				\end{IEEEeqnarray*}
			as required.
			
		\item
			By part ii), we clearly have $0 \leq F_n(x)$ for all $\isreal$. Also, since $\modulus{D_k(x)} \leq 2k+1$, we have
				\begin{IEEEeqnarray*}{rCl}
					F_n(x) & = & \frac{1}{n+1}\sum_{k=0}^n D_k(x) \\
					& \leq & \frac{1}{n+1}\sum_{k=0}^n (2k+1) \\
					& = & n+1 \\
					& = & F_n(0)
				\end{IEEEeqnarray*}
			again by part ii).
			
		\item
			This follows from the fact that
				\begin{displaymath}
					\myintb[\pi]{D_n(t)}{t} = 2\pi
				\end{displaymath}
				
		\item
			By part ii) above, we have
				\begin{IEEEeqnarray*}{rCl}
					0 \leq F_n(x) & = & \frac{1}{n+1}
						\frac{\sin^2 \frac{n+1}{2}x}{\sin^2 x/2} \\
					& \leq & \frac{1}{n+1}\frac{1}{\sin^2 \delta/2} \\
					& = & \frac{1}{n+1}\frac{1}{1 - \cos \delta}
				\end{IEEEeqnarray*}
			provided $0 < \delta \leq x \leq \pi$.
			
		\item
			If $0 < \delta < \pi$, then
				\begin{IEEEeqnarray*}{rCl}
					0 \leq \myinta{\delta}{\pi}{F_n(t)}{t}
						& \leq & \myinta{\delta}{\pi}
						{\frac{1}{n+1}\frac{1}{1-\cos \delta}}{t} \\
					& = & \frac{1}{n+1}\frac{\pi-\delta}{1-\cos \delta}
				\end{IEEEeqnarray*}
			$\rightarrow 0$ as $n \rightarrow \infty$. Since $F_n(x)$
			is even,
				\begin{displaymath}
					\lim_{n \rightarrow \infty}\myinta{-\pi}{-\delta}
						{F_n(t)}{t} =
						\lim_{n \rightarrow \infty}\myinta{\delta}{\pi}{F_n(t)}{t} = 0
				\end{displaymath}

		\item
			This follows from parts iv) and vi).
			
	\end{enumerate}
\end{proof}

\end{section}