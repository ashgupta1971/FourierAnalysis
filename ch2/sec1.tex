\begin{section}{The Heat Problem}

The one dimensional heat problem is the following:
Given a continuous function $f:[0,\pi] \to \real$
which satisfies $f(0) = f(\pi) = 0$ and represents
the temperature at time zero of a wire of length
$\pi$, is it possible to find a
continuous function $u:[0,\pi] \times [0,\infty)
\to \real$ which represents the temperature of the
wire at time $t \geq 0$?

It can be shown (although we will not do it here) using
methods from physics, that $u$ must have partial derivatives
on $(0,\pi) \times (0,\infty)$ and furthermore, it must
satisfy the following conditions

%% Conditions for the heat equation
	\begin{IEEEeqnarray}{l}
		u(x,0) = f(x) \text{ for all } x \in [0,\pi] \\ \label{heat1}
		u(0,t) = u(\pi,t) = 0 \text{ for all } t \geq 0 \label{heat2} \\
		\frac{\partial u}{\partial t}(x,t) = c^2 \,
			\frac{\partial^2 u}{\partial x^2}(x,t)
			\text{ for all } (x,t) \in 
			(0,\pi) \times (0,\infty) \label{heat3}
	\end{IEEEeqnarray}
where $c$ is a real constant.

To find a possible solution for $u$, let us make two
simplying assumptions, namely, that $c = 1$ and $u(x,t)
= X(x)T(t)$ for some functions $X:[0,\pi] \to \real$ and
$T:[0,\infty) \to \real$.

By condition \ref{heat3}, we must have $X(x)T'(t) =
X''(x)T(t)$. If we omit the trivial case where $X =
T = 0$, then there exist $0 < x_0 < \pi$ and $t_0 > 0$
such that $X(x_0) \neq 0$ and $T(t_0) \neq 0$.

Let
	\begin{displaymath}
		\lambda = \frac{T'(t_0)}{T(t_0)}
			= \frac{X''(x_0)}{X(x_0)}
	\end{displaymath}
	
Then $X(x_0)T'(t) = X''(x_0)T(t)$ and $X(x)T'(t_0)
= X''(x)T(t_0)$. So we have
	\begin{IEEEeqnarray}{l}
		T'(t) = \lambda T(t) \text{ and } \label{heat4} \\
		X''(x) = \lambda X(x) \label{heat5}
	\end{IEEEeqnarray}

%% Possibilities for lambda

%% lambda > 0
There are three possibilities for $\lambda$. Either
$\lambda > 0, \lambda = 0$ or $\lambda < 0$. If
$\lambda > 0$ then by equation \ref{heat5} we must have
	\begin{displaymath}
		X(x) = \alpha e^{\sqrt{\lambda}x}
			+ \beta e^{-\sqrt{\lambda}x}
	\end{displaymath}
But by condition \ref{heat2}, this implies $\alpha =
\beta = 0$ which is impossible.

%% lambda = 0
If $\lambda = 0$ then $X''(x) = 0$ so $X(x) = ax$
for some constant $a$. Again, condition \ref{heat2}
implies $a = 0$ so this is impossible.

%% lambda < 0
Therefore $\lambda < 0$. Say $\lambda = -k^2$ for some
$k \in \real, k > 0$. Equation \ref{heat5} implies
	\begin{displaymath}
		X(x) = \alpha \cos kx + \beta \sin kx
	\end{displaymath}
	
But condition \ref{heat2} implies $\alpha = 0$ and $k
\in \natrl$.

Substituting $\lambda$ in equation \ref{heat4}, we obtain
	\begin{displaymath}
		T(t) = \gamma e^{-k^2t} \text{ for } t \geq 0
	\end{displaymath}

%% Final form of heat equation
Therefore, $u$ is of the form
	\begin{displaymath}
		u(x,t) = b e^{-k^2t} \sin kx \text{ for }
			(x,t) \in [0,\pi] \times [0,\infty)
	\end{displaymath}
where $b$ is a constant. It follows that for any
choice of constants $b_1, b_2, \ldots, b_n$ the
function
	\begin{equation}
		u(x,t) = \sum_{k=1}^n b_k e^{-k^2 t} \sin kx
			\label{heat7}
	\end{equation}
satisfies conditions \ref{heat2} and \ref{heat3}
above. For condition \ref{heat1} to hold, we must
be able to write $f$ in the form
	\begin{equation}
		f(x) = \sum_{k=1}^n b_k \sin kx \label{heat6}
	\end{equation}
for some constants $b_1, b_2, \ldots, b_n$.

Another possibility for $f$ and $u$ is that there
exist real scalars $b_1, b_2, \ldots$ such that
	\begin{equation}
		f(x) = \sum_{k=1}^\infty b_k \sin kx \label{heat8}
	\end{equation}
and
	\begin{equation}
		u(x,t) = \sum_{k=1}^\infty b_k e^{-k^2 t} \sin kx
			\label{heat9}
	\end{equation}
Of course, the series in \ref{heat9} must be
$\ctsdiff[2]$ and we must be able to
differentiate it term by term on $(0,\pi) \times (0,\infty)$.

We will show in this chapter the conditions that $f$ must satisfy
for equation \ref{heat8} to hold and hence, for the
function $u$ in \ref{heat9} to represent a solution to
the heat problem.

%% Uniqueness
In the next section, we will also show that the solution
to the heat problem derived here is unique.

Before concluding this section, we will show that the function
$u$, defined by \ref{heat9} is $\ctsdiff$.

%%%%%%%%%%%%%%%%%%%%%%%%%%%%%%%%%%%%%%%%%%%%%%%%%%%%%
%% lemma of u is C-infinity
%%%%%%%%%%%%%%%%%%%%%%%%%%%%%%%%%%%%%%%%%%%%%%%%%%%%%

\begin{lemma}\label{lemma:uniformconv}
	Suppose $\{b_k\}_{k=1}^\infty$ is a bounded real
	sequence.
	\begin{enumerate}[i)]
		\item
			Let $\tau > 0$ and $x \in \real$ and define
			$\varphi_k(t):[\tau,\infty) \to \real$ by 
			$\varphi_k(t) = b_k e^{-k^2 t} \sin kx$. Then
				\begin{displaymath}
					\sum_{k=1}^\infty k^n \varphi_k(t)
				\end{displaymath}
			converges uniformly on $[\tau,\infty)$ for fixed
			$0 \leq n \in \intgr$.
			
		\item
			Let $t > 0$ and define $\psi_k(x):\real \to
			\real$ by $\psi_k(x) = b_k e^{-k^2 t} \sin kx$.
			Then
				\begin{displaymath}
					\sum_{k=1}^\infty k^n \psi_k(x)
				\end{displaymath}
			converges uniformly on $\real$ for fixed
			$0 \leq n \in \intgr$.
	\end{enumerate}
\end{lemma}

\begin{proof}
	Let $M$ be an upper bound for the set
	$\{\modulus{b_k}: k \geq 1\}$.
	\begin{enumerate}[i)]
		\item
			For $t \geq \tau > 0$ we have
				\begin{IEEEeqnarray*}{rCl}
					\modulus{k^n \varphi_k(t)}
						& = & \modulus{k^n b_k
						e^{-k^2 t} \sin kx} \\
					& \leq & M k^n e^{-k^2 t} \\
					& \leq & M k^n e^{-k^2 \tau} \\
					& \leq & M k^n e^{-k \tau} \\
					& = & M k^n a^k \text{ where } a = e^{-\tau} < 1
				\end{IEEEeqnarray*}
			By the ratio test, $\sum_{k=1}^\infty {M k^n a^k}$ converges
			and so, by the Weierstrass M-test, $\sum_{k=1}^\infty
			k^n \varphi_k(t)$ converges uniformly on $[\tau,\infty)$.
			
		\item
			The proof is similar to that in part i).
	\end{enumerate}
\end{proof}

%%%%%%%%%%%%%%%%%%%%%%%%%%%%%%%%%%%%%%%%%%%%%%%%%%%%%
%% Theorem of u is C-infinity
%%%%%%%%%%%%%%%%%%%%%%%%%%%%%%%%%%%%%%%%%%%%%%%%%%%%%

\begin{thrm}\label{thrm:HeatSolnIsCInfty}
	Let $u:\real \times (0,\infty)$ be defined by
		\begin{displaymath}
			u(x,t) = \sum_{k=1}^\infty b_k e^{-k^2 t} \sin kx
		\end{displaymath}
	where $\{b_k\}_{k=1}^\infty$ is a bounded real sequence.
	Then $u$ is $\ctsdiff$ on its domain.
\end{thrm}

\begin{proof}
	Fix $x \in \real$ and $\tau > 0$ and let $\varphi_k(t)
	:[\tau,\infty) \to \real$ be defined by $\varphi_k(t)
	= b_k e^{-k^2 t} \sin kx$. Then by Lemma \ref{lemma:uniformconv},
	the function $\varphi:[\tau,\infty) \to
	\real$ defined by
		\begin{displaymath}
			\varphi(t) = \sum_{k=1}^\infty \varphi_k(t)
		\end{displaymath}
	converges uniformly on its domain. Also by Lemma
	\ref{lemma:uniformconv},
		\begin{displaymath}
			\sum_{k=1}^\infty \varphi_k'(t)
				= \sum_{k=1}^\infty -k^2 b_k e^{-k^2 t}
				\sin kx
		\end{displaymath}
	converges uniformly on its domain.
	
	Therefore, by Theorem \ref{thrm:derivative}, $\varphi$
	is $\ctsdiff[1]$ on $[\tau,\infty)$ and $\varphi'(t) =
	\sum_{k=1}^\infty \varphi_k'(t)$.
	
	Assume $\varphi$ is $\ctsdiff[n]$ for $n > 1$ and that
	$\varphi^{(n)}(t) = \sum_{k=1}^\infty \varphi_k^{(n)}(t)$.
	Then, noting that $\varphi_k^{(n+1)}(t) = (-1)^{(n+1)}
	k^{2(n+1)} b_k e^{-k^2 t} \sin kx$ and using Lemma
	\ref{lemma:uniformconv} combined with Theorem \ref
	{thrm:derivative} again, it is easy to see that $\varphi$
	is $\ctsdiff[n+1]$ on $[\tau,\infty)$. Therefore, by induction,
	$\varphi$ is $\ctsdiff$ on $[\tau,\infty)$ for all
	$\tau > 0$. It follows that $\varphi$ is $\ctsdiff$ on
	$(0,\infty)$.
	
	Now let $\psi_k(x) = b_k e^{-k^2 t} \sin kx$ for $x \in \real$
	and fixed $t > 0$. An argument identical to that used above
	shows that the function $\psi(x) = \sum_{k=1}^\infty \psi_k(x)$
	is $\ctsdiff$ on $\real$.
	
	Thus, $u$ is $\ctsdiff$ on $\real \times (0,\infty)$.
\end{proof}

\end{section}
