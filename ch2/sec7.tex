\begin{section}{The Heat Problem (Revisited)}

Recall the Heat Problem stated in the first
section of this chapter: Given a continuous 
function $f:[0,\pi] \to \real$
which satisfies $f(0) = f(\pi) = 0$ and represents
the temperature at time zero of a wire of length
$\pi$, is it possible to find a
continuous function $u:[0,\pi] \times [0,\infty)
\to \real$ which represents the temperature of the
wire at time $t \geq 0$?

We stated that $u$ must have partial derivatives
on $(0,\pi) \times (0,\infty)$ and furthermore, it must
satisfy the following conditions

%% Conditions for the heat equation
	\begin{IEEEeqnarray*}{l}
		u(x,0) = f(x) \text{ for all } x \in [0,\pi] \; (\ast) \\
		u(0,t) = u(\pi,t) = 0 \text{ for all } t \geq 0 \; (\dagger) \\
		\frac{\partial u}{\partial t}(x,t) = c^2 \,
			\frac{\partial^2 u}{\partial x^2}(x,t)
			\text{ for all } (x,t) \in 
			(0,\pi) \times (0,\infty) \; (\ddagger)
	\end{IEEEeqnarray*}
where $c$ is a real constant.

We showed that the funtion $u$ defined by
	\begin{displaymath}
		u(x,t) = \sum_{k=1}^\infty b_k e^{-k^2t}\sin kx
			\; (\ast)
	\end{displaymath}
for some real scalars $b_k, k \in \natrl$ satisfies
$(\dagger)$ and $(\ddagger)$ and that
this $u$ is $\ctsdiff$ on $\real \times (0,\infty)$.

Therefore, to show that $u$ represents the solution
to the Heat Problem corresponding to $f$, we only
have to show that $u$ is continuous on $[0,\pi] \times
\{0\}$ and that $f$ can be written in the form
	\begin{displaymath}
		f(x) = \sum_{k=1}^\infty b_k\sin kx
	\end{displaymath}

In this section we present one condition, which,
if satisfied, allows $f$ to have a solution $u$
for the Heat Problem.

%% Existence of soln to heat problem
\begin{thrm}
	Suppose $f$ belongs to $C_0[0,\pi]$ and that
	$f \in \ctsdiff[2]$ on $[0,\pi]$. Then the
	Heat Problem has a (unique) solution (namely,
	the function $u$ stated above.)
\end{thrm}

\begin{proof}
	By Proposition \ref{prop:periodic}, there exists
	a unique, odd, $2\pi$-periodic extension of $f$
	to $\real$. Call this extension $g$. Then by
	Corollary \ref{cor:EvenOdd} the Fourier series
	of $g$ has the form
		\begin{displaymath}
			\sum_{k=1}^\infty b_k \sin kx
		\end{displaymath}
	and this series converges uniformly to $g$ on
	$\real$ by Corollary \ref{cor:FourierUniformConv2}.
	Therefore, the function $u$ given in $(\ast)$
	represents a solution to the Heat Problem for $f$
	provided we can show it is continuous at $(x_0,0)$
	for $x_0 \in [0,\pi]$. In fact, we will show that
	$u$ is continuous on $\real \times \{0\}$.
	
	Let $M > 0$ be an upper bound for the $b_k$'s.
	Fix $\epsilon > 0$. Given $x_0 \in \real$
	choose $\delta' > 0$ such that $\modulus{g(x)-g(x_0)}
	< \epsilon/2$ whenever $\modulus{x-x_0} < \delta'$.
	Next choose $N \in \natrl$ such that
		\begin{displaymath}
			M\sum_{k=N+1}^\infty \frac{2}{k^2}
				< \epsilon/4
		\end{displaymath}
	
	Finally, choose $\tau > 0$ such that
		\begin{displaymath}
			M\sum_{k=1}^N \frac{1-e^{-k^2\tau}}{k^2}
				< \epsilon/4
		\end{displaymath}
	Let $\delta = \min(\delta',\tau)$. Then if
	$\norm[]{(x,t)-(x_0,0)} < \delta$ we have
		\begin{IEEEeqnarray*}{rCl}
			\modulus{u(x,t)-u(x_0,0)} & \leq &
				\modulus{u(x,t)-u(x,0)} +
				\modulus{u(x,0)-u(x_0,0)} \\
			& = & \modulus{\sum_{k=1}^\infty
				b_k e^{-k^2t}\sin kx - \sum_{k=1}^\infty
				b_k \sin kx} + \modulus{g(x)-g(x_0)} \\
			& \leq & \sum_{k=1}^\infty \modulus{b_k}
				(1-e^{-k^2t}) + \epsilon/2 \\
			& \leq & M \sum_{k=1}^\infty \frac{1-e^{-k^2t}}{k^2}
				+ \epsilon/2
		\end{IEEEeqnarray*}
	By Theorem \ref{thrm:coeffbounds}. Therefore,
		\begin{IEEEeqnarray*}{rCl}
			\modulus{u(x,t)-u(x_0,0)} & \leq &
				M \sum_{k=1}^N \frac{1-e^{-k^2t}}{k^2}
				+ M \sum_{k=N+1}^\infty \frac{1-e^{-k^2t}}{k^2}
				+ \epsilon/2 \\
			& \leq & M \sum_{k=1}^N \frac{1-e^{-k^2\tau}}{k^2}
			 	+ M \sum_{k=N+1}^\infty \frac{2}{k^2} + \epsilon/2 \\
			& \leq & \epsilon/4 + \epsilon/4 + \epsilon/2 \\
			& = & \epsilon
		\end{IEEEeqnarray*}
	And so $u$ is continuous on $\real \times [0,\infty)$. Thus
	the restriction of this $u$ to $[0,\pi] \times [0,\infty)$
	represents the solution to the Heat Problem for $f$. (This is
	solution is unique by Theorem \ref{thrm:HeatUnique}.)
\end{proof}

\end{section}
