\begin{section}{Three Convergence Theorems}

In this section, we establish three convergence
theorems which will be used in later sections. The
first two theorems deal with derivatives and the
second with integrals.

%%%%%%%%%%%%%%%%%%%%%%%%%%%%%%%%%%%%%%%%%%%%%%%%%%%%%%%%%
%% First Derivative convergence theorem
%%%%%%%%%%%%%%%%%%%%%%%%%%%%%%%%%%%%%%%%%%%%%%%%%%%%%%%%%

\begin{thrm}
\label{thrm:derivative}
	\begin{enumerate}[i)]
		\item
			Suppose $\{f_k\}_{k=1}^\infty$ is a sequence of complex-valued,
			continuously differentiable functions on a set $X \subset \real$
			which converge pointwise on their domain and whose derivatives
			converge uniformly on $X$. Then $\{f_k\}_{k=1}^\infty$ converges
			pointwise on $X$ to a $\ctsdiff[1]$ function which satisfies
				\begin{displaymath}
					\frac{d}{dx}\lim_{k \rightarrow \infty}f_k(x)
						= \lim_{k \rightarrow \infty}f_k'(x)
				\end{displaymath}
			
		\item
			Suppose $\{f_k\}_{k=1}^\infty$ is a sequence of complex-valued,
			continuously differentiable functions on a set $X \subset \real$.
			Futhermore, suppose $\sum_{k=1}^\infty f_k$ converges pointwise
			on $X$ and $\sum_{k=1}^\infty f_k'$ converges uniformly
			on $X$. Then $\sum_{k=1}^\infty f_k$ converges pointwise on $X$ to
			a $\ctsdiff[1]$ function which satisfies
				\begin{displaymath}
					\frac{d}{dx}\sum_{k=1}^\infty f_k(x)
						= \sum_{k=1}^\infty f_k'(x)
				\end{displaymath}
	\end{enumerate}
\end{thrm}

\begin{proof}
	\begin{enumerate}[i)]
		\item
			Let $g = \lim_{k \rightarrow \infty} f_k'$. Then for $x \in X$,
				\begin{IEEEeqnarray*}{rCl}
					f(x) & = & \lim_{k \rightarrow \infty} f_k(x) \\
					& = & \lim_{k \rightarrow \infty}
						\left( f_k(x_0) + \myinta{x_0}{x}{f_k'(t)}{t}
						\right) \\
					& = & f(x_0) + \lim_{k \rightarrow \infty}
						\myinta{x_0}{x}{f_k'(t)}{t} \\
					& = & f(x_0) + \myinta{x_0}{x}{g(t)}{t}
				\end{IEEEeqnarray*}
			But $g$ is continuous on $X$, so the result follows by 
			the Fundamental Theorem of Calculus.
				
		\item
			Apply part i) to the sequence of partial sums
				\begin{displaymath}
					s_n(x) = \sum_{k=1}^n f(x)
				\end{displaymath}
	\end{enumerate}			
\end{proof}

%%%%%%%%%%%%%%%%%%%%%%%%%%%%%%%%%%%%%%%%%%%%%%%%%%%%%%%%%
%% Second Derivative convergence theorem
%%%%%%%%%%%%%%%%%%%%%%%%%%%%%%%%%%%%%%%%%%%%%%%%%%%%%%%%%

\begin{thrm}\label{thrm:DerivUniConv}
	Suppose $\{f_k\}_{k=1}^\infty$ is a sequence of complex-valued,
	continuously differentiable functions on a bounded interval $[a,b]$.

		\begin{enumerate}[i)]
			\item
			If the sequence $\{f_k\}_{k=1}^\infty$ converges pointwise 
			at some $x_0 \in [a,b]$ and $\{f_k'\}_{k=1}^\infty$ converges
			uniformly to some $g \in \ctsab{a}{b}$ then $\{f_k\}_{k=1}^\infty$
			converges uniformly on $[a,b]$ to some $f \in \ctsdiff[1]([a,b])$
			and $f'(x) = g(x)$ for all $x \in [a,b]$.
			
			\item
			If $\sum_{k=1}^\infty f_k$ converges pointwise at some $x_0 \in
			[a,b]$ and $\sum_{k=1}^\infty f_k'$ converges uniformly on $[a,b]$
			then $\sum_{k=1}^\infty f_k$ converges uniformly on $[a,b]$ to a
			$\ctsdiff[1]$ function which satisfies
				\begin{displaymath}
					\frac{d}{dx}\sum_{k=1}^\infty f_k(x)
						= \sum_{k=1}^\infty f_k'(x)
				\end{displaymath}
		\end{enumerate}
\end{thrm}

\begin{proof}
	\begin{enumerate}[i)]
		\item
		Fix $\epsilon > 0$. Choose $N_1$ such that
		$\modulus{f_m(x_0) - f_n(x_0)} < \epsilon/2$ for
		$n, m > N_1$. Choose $N_2$ such that $\modulus{f_m'(t) -
		f_n'(t)} < \epsilon/2(b-a)$ for $n, m > N_2$ and all
		$t \in [a,b]$. Let $N =	\max(N_1, N_2)$. Then for $n, m > N$,
	
			\begin{IEEEeqnarray*}{rCl}
				\modulus{f_m(x) - f_n(x)} & = &
					\modulus{\left(f_m(x_0) + \myinta{a}{b}{f_m'(t)}{t}\right)
					- \left(f_n(x_0) + \myinta{a}{b}{f_n'(t)}{t}\right)} \\
				& \leq & \modulus{f_m(x_0) - f_n(x_0)}
					+ \modulus{\myinta{a}{b}{f_m'(t)}{t}
					- \myinta{a}{b}{f_n'(t)}{t}} \\
				& < & \epsilon/2 + \myinta{a}{b}{\modulus{f_m'(t)-f_n'(t)}}{t} \\
				& < & \epsilon/2 + \myinta{a}{b}{\frac{\epsilon}{2(b-a)}}{t} \\
				& = & \epsilon/2 + \epsilon/2 \\
				& = & \epsilon
			\end{IEEEeqnarray*}
	
		Hence, $\{f_k\}_{k=1}^\infty$ satisfies the uniform Cauchy criterion
		and so converges uniformly to some $f:[a,b] \to \cmplx$. The remaining
		assertion can be proven identically as in Theorem \ref{thrm:derivative}.
		
		\item
		Apply part i) to the sequence of partial sums
			\begin{displaymath}
				s_n(x) = \sum_{k=1}^n f(x)
			\end{displaymath}
	\end{enumerate}
\end{proof}

%%%%%%%%%%%%%%%%%%%%%%%%%%%%%%%%%%%%%%%%%%%%%%%%%%%%%%%%%
%% Integral convergence theorem
%%%%%%%%%%%%%%%%%%%%%%%%%%%%%%%%%%%%%%%%%%%%%%%%%%%%%%%%%

\begin{thrm}\label{thrm:IntegralConv}
	\begin{enumerate}[i)]
	\item
	Suppose $h:[a,b] \to \cmplx$ is bounded on its domain and
	is Riemann Integrable on $[c,b]$ for all $a < c < b$. Then
	$h$ is Riemann Integrable on $[a,b]$ and $\int_a^b h =
	\lim_{c \rightarrow a^+}\int_c^b h$.
	
	\item
	Suppose $h:[a,b] \to \cmplx$ is bounded on its domain and
	is Riemann Integrable on $[a,c]$ for all $a < c < b$. Then
	$h$ is Riemann Integrable on $[a,b]$ and $\int_a^b h =
	\lim_{c \rightarrow b^-}\int_a^c h$.
	\end{enumerate}
\end{thrm}

\begin{proof}
	We will only prove part i). The second part follows by a similar
	argument. First note that we may assume, without loss of generality, that $h$
	is real-valued. (For the general case, just apply the following argument
	to the real and imaginary parts of of $h$.) Fix $\epsilon > 0$. Let
		\begin{displaymath}
			M = \sup\{\modulus{h(x)}:	a < x < b\}
		\end{displaymath}
	Choose $c \in (a,b)$ such that $(c-a) < \epsilon/4M$.
	Since $h$ is Riemann Integrable on $[c,b]$, we can find a partition
	$c = x_1 < x_2 < \ldots < x_n = b$ such that
	\begin{displaymath}
	\sum_{k=2}^n M_k(x_k - x_{k-1}) - \sum_{k=2}^n m_k(x_k - x_{k-1}) < \epsilon/2
	\end{displaymath}
	where
		\begin{displaymath}
			M_k = \sup\{h(x): x_{k-1} \leq x \leq x_k\}
		\end{displaymath}
	and
		\begin{displaymath}
			m_k = \inf\{h(x): x_{k-1} \leq x \leq x_k\}
		\end{displaymath}
	Let $x_0 = a$,
		\begin{displaymath}
			M_1 = \sup\{h(x): x_0 = a \leq x \leq x_1 = c\}
		\end{displaymath}
	and
		\begin{displaymath}
			m_1 = \inf\{h(x): x_0 = a \leq x \leq x_1 = c\}
		\end{displaymath}
	Then
	\begin{IEEEeqnarray*}{l}
	\sum_{k=1}^n M_k(x_k - x_{k-1}) - \sum_{k=1}^n m_k(x_k - x_{k-1}) \\
	= (M_1 - m_1)(c - a) + \sum_{k=2}^n (M_k - m_k)(x_k - x_{k-1}) \\
	< \frac{2M\epsilon}{4M} + \epsilon/2 = \epsilon
	\end{IEEEeqnarray*}
	
	It follows that $h$ is Riemann Integrable on $[a,b]$. Moreover,
	\begin{displaymath}
	\modulus{\int_a^b h - \int_c^b h} = \modulus{\int_a^c h}
	\leq \int_a^c \modulus{h} \leq M(c-a) \rightarrow 0
	\end{displaymath}
	as $c \rightarrow a^+$.
\end{proof}


%%%%%%%%%%%%%%%%%%%%%%%%%%%%%%%%%%%%%%%%%%%%%%%%%%%%%%%%%
%% Corollary to the integral convergence theorem
%%%%%%%%%%%%%%%%%%%%%%%%%%%%%%%%%%%%%%%%%%%%%%%%%%%%%%%%%

\begin{cor}\label{cor:IntegralConv}
	\begin{enumerate}[i)]
	\item
	Suppose $h:[a,b] \to \cmplx$ is right continous at $a$ and Riemann
	Integrable on $[c,b]$ for all $c \in (a,b)$. Then $h$ is Riemann
	Integrable on $[a,b]$ and $\int_a^b h =
	\lim_{c \rightarrow a^+}\int_c^b h$.
	
	\item
	Suppose $h:[a,b] \to \cmplx$ is left continuous at $b$ and Riemann
	Integrable on $[a,c]$ for all $c \in (a,b)$. Then $h$ is Riemann
	Integrable on $[a,b]$ and $\int_a^b h =
	\lim_{c \rightarrow b^-}\int_a^c h$.
	\end{enumerate}
\end{cor}

\begin{proof}
	Again, we will only prove part i). The second part follows by a similar
	argument.
	Choose $c > a$ such that $\modulus{h(t) - h(a)} \leq 1$ for $t \in [a,c]$.
	Then $\modulus{h(t)} \leq \modulus{h(a)} + 1$ on for $t \in [a,c]$. So $h$
	is bounded on $[a,c]$. Also, it is bounded on $[c,b]$ because it is Riemann
	Integrable on this interval. Therefore, $h$ satisfies the hypotheses of
	Theorem \ref{thrm:IntegralConv} and the result follows.
\end{proof}

\end{section}