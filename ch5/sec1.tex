\begin{section}{Properties of Convolutions}

	In this section we define the convolution of two
	functions and present some relevant results.
	
%%%%%%%%%%%%%%%%%%%%%%%%%%%%%%%%%%%%%%%%%%%%%%%%%%%%%%%%%
%% Some preliminary Defns and Lemmas
%%%%%%%%%%%%%%%%%%%%%%%%%%%%%%%%%%%%%%%%%%%%%%%%%%%%%%%%%

\begin{defn}
	If $f:\real \rightarrow \cmplx$ and $\isreal$, define
	$\tau_x f:\real \rightarrow \cmplx$ by
		\begin{displaymath}
			(\tau_x f)(t) = f(t-x)
		\end{displaymath}
	for $\isreal[t]$.
\end{defn}

	The following lemmas will be stated without proof.
	
\begin{lemma}
	Suppose $\isreal$, $V$ is one of $\ctszeror, \ctscompact$
	or $\ctsdiff[p](\real)$ and $f \in V$. Then $\tau_x f \in
	V$ and moreover the map $f \mapsto \tau_x f$ is a vector
	space isomorphism between $V$ and itself.
\end{lemma}

\begin{lemma}
	\begin{enumerate}[i)]
	
		\item
			For each $\isreal[h]$ the restriction of $\tau_h$ to $\absint$
			is a norm-preserving, isomorphism of $\absint$ onto itself.
			When we say norm-preserving, we mean that $\norm[1]{f-g} =
			\norm[1]{\tau_h f - \tau_h g}$ for all $f,g \in \absint$.
			Moreover,
				\begin{displaymath}
					\myintb{(\tau_h f)(t)}{t} = \myintb{f(t)}{t}
				\end{displaymath}
		
		\item
			For each $\isreal[h]$ the restriction of $\tau_h$ to $\ctszeror$
			is a norm-preserving, isomorphism of $\ctszeror$ onto itself.
			That is,
				\begin{displaymath}
					\supnorm{\tau_h f - \tau_h g} = \supnorm{f-g}
				\end{displaymath}
			In fact,
				\begin{displaymath}
					\F[\tau_h f](x) = e^{-ihx} \f(x)
				\end{displaymath}
			
	\end{enumerate}
\end{lemma}

\begin{lemma}
		Suppose $V$ is one of $\ctszeror$ or $\absint$ (with their natural
		norms). Then the mapping
			\begin{displaymath}
				x \mapsto \tau_x \bigg \vert_V
			\end{displaymath}
		is a group homomorphism from $\real$ under addition to the group
		of linear operators on $V$ under composition.
\end{lemma}

\begin{defn}
	If $f:\real \rightarrow \cmplx$, define $f^-:\real \rightarrow \cmplx$
	by $f^-(t)=f(-t)$ and write $If=f^-$.
\end{defn}

\begin{lemma}
	\begin{enumerate}[i)]
		\item
			If $V$ is either $\absint$ or $\ctszeror$ then the restriction of
			$I$ to $V$ is a norm-preserving isomorphism of $V$ onto itself.
		\item
			If $f \in \absint$ then
				\begin{displaymath}
					\F[If](x)=\f(-x)=(I\f)(x)=\check{f}(x)
				\end{displaymath}
	\end{enumerate}
\end{lemma}

\begin{defn}
	Let $\Rloc$ be the set of all functions from $\real$ to $\cmplx$ which
	are Riemann integrable on $[a,b]$ whenever $a,b \in \real$ and $a < b$.
	Note that $\Rloc = \ctsr \cup \absint$. Let $\Rc$ be the subset of
	functions in $\Rloc$ which vanish outside a compact set. Note that $\Rc
	\in \absint$.
\end{defn}
	
%%%%%%%%%%%%%%%%%%%%%%%%%%%%%%%%%%%%%%%%%%%%%%%%%%%%%%%%%
%% Defn of Convolution
%%%%%%%%%%%%%%%%%%%%%%%%%%%%%%%%%%%%%%%%%%%%%%%%%%%%%%%%%

\begin{defn}
	Suppose $f,g \in \Rloc$ and that, for all $\isreal$, the map
		\begin{displaymath}
			t \mapsto f(t)g(x-t)
		\end{displaymath}
	belongs to $\absint$. Then we say \emph{$f$ is convolvable with $g$}
	and	we define $\conva{f}{g}:\real \rightarrow \cmplx$ by
		\begin{displaymath}
			\convb{f}{g}{x} = \myintb{f(t)g(x-t)}{t}
		\end{displaymath}
	for all $\isreal$. We call $\conva{f}{g}$ the \emph{convolution of
	$f$ with $g$}.
\end{defn}
	
%%%%%%%%%%%%%%%%%%%%%%%%%%%%%%%%%%%%%%%%%%%%%%%%%%%%%%%%%
%% First result of convolutions
%%%%%%%%%%%%%%%%%%%%%%%%%%%%%%%%%%%%%%%%%%%%%%%%%%%%%%%%%

\begin{prop}
	Suppose $f,g \in \Rloc$ and $\conva{f}{g}$ is defined. Then
	$\conva{g}{f}$ is defined and the two functions are equal.
\end{prop}

\begin{proof}
	Let $\isreal$ and $A > 0$. Then
		\begin{IEEEeqnarray*}{rCl}
			\myintb[A]{\modulus{g(t)f(x-t)}}{t} & = &
				-\myinta{x+A}{x-A}{\modulus{g(x-s)f(s)}}{s} \; \;
				(s=x-t) \\
			& = & \myinta{x-A}{x+A}{\modulus{f(s)g(x-s)}}{s} \\
			& \leq & \myintb{\modulus{f(t)g(x-t)}}{t} \\
			& < & \infty
		\end{IEEEeqnarray*}
	Hence the map $t \mapsto g(t)f(x-t)$ belongs to $\absint$ and
		\begin{displaymath}
			\myintb{\modulus{g(t)f(x-t)}}{t} \leq \myintb
				{\modulus{f(t)g(x-t)}}{t}
		\end{displaymath}
	Similarly,
		\begin{displaymath}
			\lim_{A \rightarrow \infty}\myintb[A]{g(t)f(x-t)}{t}
				= \myintb{f(t)g(x-t)}{t}
		\end{displaymath}
	ie. $\convb{g}{f}{x} = \convb{f}{g}{x}$ for all $\isreal$.
\end{proof}
	
%%%%%%%%%%%%%%%%%%%%%%%%%%%%%%%%%%%%%%%%%%%%%%%%%%%%%%%%%
%% Second result of convolutions
%%%%%%%%%%%%%%%%%%%%%%%%%%%%%%%%%%%%%%%%%%%%%%%%%%%%%%%%%

\begin{lemma}
	If $f_1,f_2,g \in \Rloc, \iscmplx[\alpha]$ and both
	$\conva{f_1}{g}$ and $\conva{f_2}{g}$ are defined
	then $\conva{(\alpha f_1+f_2)}{g}$ is defined and
	$\conva{(\alpha f_1+f_2)}{g} = \alpha \conva{f_1}{g}
	+ \conva{f_2}{g}$.
\end{lemma}

\begin{proof}
	Trivial.
\end{proof}

%%%%%%%%%%%%%%%%%%%%%%%%%%%%%%%%%%%%%%%%%%%%%%%%%%%%%%%%%
%% Third result of convolutions
%%%%%%%%%%%%%%%%%%%%%%%%%%%%%%%%%%%%%%%%%%%%%%%%%%%%%%%%%

\begin{lemma}
	If $f,g \in \Rloc, \conva{f}{g}$ is defined and $\isreal[h]$
	then $\conva{(\tau_h f)}{g}$ is defined and $\conva{(\tau_h f)}
	{g} = \tau_h(\conva{f}{g})$.
\end{lemma}

\begin{proof}
	For any $A > 0$,
		\begin{IEEEeqnarray*}{rCl}
			\myintb[A]{(\tau_h f)(t)g(x-t)}{t} & = &
				\myintb[A]{f(t-h)g(x-t)}{t} \\
			& = & \myinta{-A-h}{A-h}{f(s)g(x-s-h)}{s} \; \;
				(s=t-h) \\
			& = & \myinta{-A-h}{A-h}{f(s)g((x-h)-s)}{s} \\
			& \rightarrow & \convb{f}{g}{x-h} \; \; \text{as }
				A \rightarrow \infty \\
			& = & \tau_h(\convb{f}{g}{x})
		\end{IEEEeqnarray*}
\end{proof}

%%%%%%%%%%%%%%%%%%%%%%%%%%%%%%%%%%%%%%%%%%%%%%%%%%%%%%%%%
%% Fourth result of convolutions
%%%%%%%%%%%%%%%%%%%%%%%%%%%%%%%%%%%%%%%%%%%%%%%%%%%%%%%%%

\begin{lemma}\label{lemma:Conv1}
	If $f \in \Rc$ and $g \in \Rloc$ then $\conva{f}{g}$
	is defined. Moreover, if either $f \in \ctsr$ or $g
	\in \ctsr$ then $\conva{f}{g} \in \ctsr$ as well.
\end{lemma}

\begin{proof}
	We will divide the proof into steps.
		\begin{enumerate}[i)]
		
			%% f*g is defined
			\item
				Choose $a > 0$ such that $f(t)=0$ for $\modulus{t} > a$.
				Then for $a \leq A \in \real$ and $\isreal$,
					\begin{displaymath}
						\myintb[A]{f(t)g(x-t)}{t} = \myintb[a]{f(t)g(x-t)}{t}
					\end{displaymath}
				So $\conva{f}{g}$ is defined and
					\begin{displaymath}
						\convb{f}{g}{x} = \myintb[a]{f(t)g(x-t)}{t}
					\end{displaymath}
				for all $\isreal$.
				
			%% Upper bound on f*g
			\item
				Now notice that for all $\isreal$,
					\begin{IEEEeqnarray*}{rCl}
						\modulus{\convb{f}{g}{x}} & \leq &
							\myintb[a]{\modulus{f(t)}\modulus{g(x-t)}}{t} \\
						& \leq & \supnorm{f} \myinta{x-a}{x+a}{\modulus{g(s)}}{s}
							\; \; (s=x-t)
					\end{IEEEeqnarray*}
			
			%% Final step
			\item
				Suppose $f \in \ctsr$. If $\isreal[h]$ and $\modulus{h} < 1$
				then for all $\isreal$,
					\begin{IEEEeqnarray*}{rCl}
						\modulus{\convb{f}{g}{x-h}-\convb{f}{g}{x}} & = &
							\modulus{\tau_h \convb{f}{g}{x}-\convb{f}{g}{x}} \\
						& = & \modulus{\convb{(\tau_h f - f)}{g}{x}} \\
						& \leq & \supnorm{\tau_h f - f} \myinta{x-a-1}{x+a+1}
							{\modulus{g(t)}}{t} \; \; \text{by Step 2} \\
						& \rightarrow & 0 \; \; \text{as } h \rightarrow 0
					\end{IEEEeqnarray*}
				(for all fixed $\isreal$) because $f$ is uniformly continuous on
				$[x-1,x+1]$.
			
		\end{enumerate}
\end{proof}

%%%%%%%%%%%%%%%%%%%%%%%%%%%%%%%%%%%%%%%%%%%%%%%%%%%%%%%%%
%% Fifth result of convolutions
%%%%%%%%%%%%%%%%%%%%%%%%%%%%%%%%%%%%%%%%%%%%%%%%%%%%%%%%%

\begin{lemma}
	Suppose $f \in \Rloc \cap B(\real)$ (where $B(\real)$ is the
	set of bounded functions on $\real$). If $g \in \absint$ then
	$\conva{f}{g}$ is defined and belongs to $B(\real)$. Moreover,
	$\supnorm{\conva{f}{g}} \leq \supnorm{f} \norm[1]{g}$.
\end{lemma}

\begin{proof}
	This follows from the proof of Lemma \ref{lemma:Conv1}.
\end{proof}

%%%%%%%%%%%%%%%%%%%%%%%%%%%%%%%%%%%%%%%%%%%%%%%%%%%%%%%%%
%% Sixth result of convolutions
%%%%%%%%%%%%%%%%%%%%%%%%%%%%%%%%%%%%%%%%%%%%%%%%%%%%%%%%%

\begin{lemma}\label{lemma:Conv2}
	If $f \in \ctsr$ and $g \in \cts_c^p(\real)$ for some $\isnatrl[p]$
	then $\conva{f}{g} \in \ctsdiff[p](\real)$ and
		\begin{displaymath}
			(\conva{f}{g})^{(k)}(x) = (\conva{f}{g^{(k)}}){x}
		\end{displaymath}
	for all $\isreal$, $1 \leq k \leq p$.
\end{lemma}

\begin{proof}
	Let $h(x)=\convb{f}{g}{x}$ for $\isreal$. Choose $a > 0$ such that
	$g(t)=0$ for $\modulus{t} \geq a$. Then
		\begin{displaymath}
			h(x) = \myinta{x-a}{x+a}{f(t)g(x-t)}{t}
		\end{displaymath}
	For $b > 0, -b < x < b$ we have
		\begin{displaymath}
			h(x) = \myinta{-b-a}{b+a}{f(t)g(x-t)}{t}
		\end{displaymath}
	If we let
		\begin{displaymath}
			F(x,t) = f(t)g(x-t)
		\end{displaymath}
	for $\modulus{t} \leq a+b$ and $-b < x < b$ then we may apply
	Theorem \ref{thrm:Leibniz1} to conclude that $h \in \ctsdiff[1]$
	on $(-b,b)$ and
		\begin{IEEEeqnarray*}{rCl}
			h'(x) & = & \myinta{-b-a}{b+a}{f(t)g'(x-t)}{t} \\
			& = & \myintb{f(t)g'(x-t)}{t}
		\end{IEEEeqnarray*}
	It follows that $h \in \ctsdiff[1](\real)$ and by induction that
	$h \in \ctsdiff[p](\real)$ and that $h^{(k)} = \conva{f}{g^{(k)}}$
	for $1 \leq k \leq p$.
\end{proof}

%%%%%%%%%%%%%%%%%%%%%%%%%%%%%%%%%%%%%%%%%%%%%%%%%%%%%%%%%
%% Seventh result of convolutions
%%%%%%%%%%%%%%%%%%%%%%%%%%%%%%%%%%%%%%%%%%%%%%%%%%%%%%%%%

\begin{lemma}
	If $f \in \ctsdiff[p](\real)$ for some $\isnatrl[p]$ and $g \in
	\ctscompact$ then $\conva{f}{g} \in \ctsdiff[p](\real)$ and
		\begin{displaymath}
			(\conva{f}{g})^{(k)} = \conva{f^{(k)}}{g}
		\end{displaymath}
	for $1 \leq k \leq p$.
\end{lemma}

\begin{proof}
	The proof is similar to that used in Lemma \ref{lemma:Conv2}.
\end{proof}

\end{section}