\begin{section}{Absolute Integrability}

The concept of absolute integrability will be dealt with
in this section. It will be used in the next section to
prove Fubini's Theorem.

%%%%%%%%%%%%%%%%%%%%%%%%%%%%%%%%%%%%%%%%%%%%%%
%% Absolute Integrability
%%%%%%%%%%%%%%%%%%%%%%%%%%%%%%%%%%%%%%%%%%%%%%

\begin{defn}
	Suppose $a \in \real$, $f:[a,+\infty) \to \cmplx$ and $f$
	is Riemann Integrable on $[a,\lambda]$ for all $a < \lambda
	\in \real$. We say $f$ is \emph{absolutely integrable} on
	$[a,+\infty)$ provided there exists $M > 0$ such that

	\begin{displaymath}
	\myinta{a}{\lambda}{\modulus{f(t)}}{t} \leq M \text{ for all }
	a \leq \lambda \in \real
	\end{displaymath}
	
	Or equivalently,
	\begin{displaymath}
	\lim_{\lambda \rightarrow +\infty}
	\myinta{a}{\lambda}{\modulus{f(t)}}{t} \text{ exists.}
	\end{displaymath}
	
	If $b \in \real$, absolute integrability on $(-\infty, b]$ is
	defined analagously.
\end{defn}

If $f$ is absolutely integrable on $[a,+\infty)$ and
$a \leq \lambda < \mu$ then
\begin{IEEEeqnarray*}{l}
	\modulus{\myinta{a}{\mu}{f(t)}{t} - 
	\myinta{a}{\lambda}{f(t)}{t}} \\
	\leq \myinta{\lambda}{\mu}{\modulus{f(t)}}{t} \\
	= \myinta{a}{\mu}{\modulus{f(t)}}{t} - 
	\myinta{a}{\lambda}{\modulus{f(t)}}{t}
\end{IEEEeqnarray*}
$\rightarrow 0$ as $\lambda$, $\mu \rightarrow +\infty$. By the
Cauchy criteriion,
\begin{displaymath}
	\lim_{\lambda \rightarrow +\infty}
	\myinta{a}{\lambda}{f(t)}{t}
\end{displaymath}
exists. Denote it by $\myinta{a}{+\infty}{f(t)}{t}$.

%%%%%%%%%%%%%%%%%%%%%%%%%%%%%%%%%%%%%%%%%%%%%%%%%%
%% Example of fn which is not A.I.
%%%%%%%%%%%%%%%%%%%%%%%%%%%%%%%%%%%%%%%%%%%%%%%%%%

\begin{ex}\label{ex:SinTOverT}
	Define $f:[0,+\infty) \to \real$ by
	\begin{displaymath}
	f(t) = 
		\begin{cases}
		\displaystyle{\frac{\sin t}{t}} & \text{if } t > 0 \\
		1 & \text{if } t = 0
		\end{cases}
	\end{displaymath}
	Show that $\lim_{\lambda \rightarrow +\infty}
	\myinta{0}{\lambda}{f(t)}{t}$ exists but
	$f$ is not absolutely integrable on $[0,+\infty)$.
\end{ex}

\begin{soln}
	It will be shown in Theorem \ref{thrm:Dini1}, using techniques from
	complex analysis, that
	\begin{displaymath}
	\lim_{\lambda \rightarrow +\infty}
	\myinta{0}{\lambda}{f(t)}{t} = \pi/2
	\end{displaymath}
	
	However, we can show that
	\begin{displaymath}
	\lim_{\lambda \rightarrow +\infty}
	\myinta{0}{\lambda}{\modulus{f(t)}}{t} = +\infty
	\end{displaymath}
	
	Define $g:[\pi,+\infty)$ by
	\begin{displaymath}
		g(t) =
		\begin{cases}
		\displaystyle{\frac{1}{\pi/2(k\pi + \pi/2)}}(t - k\pi) & k\pi \leq t \leq k\pi + \pi/2 \\
		\displaystyle{\frac{1}{\pi/2(k\pi + \pi/2)}}((k+1)\pi - t) & k\pi + \pi/2 \leq t \leq (k+1)\pi
		\end{cases}
	\end{displaymath}
	for $k \in \natrl$. It is easy to see that $\modulus{f(t)} \geq g(t)$ 
	for all $\pi \leq t \in \real$.
	
	Then for $n \in \natrl$
	\begin{IEEEeqnarray*}{rCl}
		\myinta{0}{(n+1)\pi}{\modulus{f(t)}}{t}
		& \geq & \myinta{\pi}{(n+1)\pi}{\modulus{f(t)}}{t} \\
		& = & \sum_{k=1}^n \myinta{k\pi}{(k+1)\pi}{\modulus{\frac{\sin t}{t}}}{t} \\
		& \geq & \sum_{k=1}^n \myinta{k\pi}{(k+1)\pi}{g(t)}{t} \\
		& = & \sum_{k=1}^n {\frac{1}{2} \left(\frac{\pi}{k\pi + \pi/2}\right)} \\
		& = & \sum_{k=1}^n \frac{1}{2k + 1}
	\end{IEEEeqnarray*}
	which goes to $+\infty$ as $n \rightarrow \infty$.
\end{soln}	

%%%%%%%%%%%%%%%%%%%%%%%%%%%%%%%%%%%%%%%%%%%%%%%%%%%%%
%% Defn of absolute integrability on R
%%%%%%%%%%%%%%%%%%%%%%%%%%%%%%%%%%%%%%%%%%%%%%%%%%%%%	

\begin{defn}
	Suppose $f:\real \to \cmplx$ and $f$ is Riemann Integrable
	on $[a,b] \subset \real$. We say $f$ is \emph{absolutely
	integrable on $\real$} provided it is absolutely integrable
	on $(-\infty, 0]$ and $[0, +\infty)$. In this case we define
		\begin{displaymath}
		\int_\real f = \myintb{f(x)}{x}
		= \myinta{-\infty}{0}{f(x)}{x} + \myinta{0}{+\infty}{f(x)}{x}
		\end{displaymath}
	and conclude that
		\begin{displaymath}
		\modulus{\myintb{f(x)}{x}} \leq
		\myintb{\modulus{f(x)}}{x}
		\end{displaymath}
	as well as
		\begin{displaymath}
		\myintb{f(x)}{x}
		= \myinta{-\infty}{a}{f(x)}{x} +
		\myinta{a}{b}{f(x)}{x} +
		\myinta{b}{+\infty}{f(x)}{x}
		\end{displaymath}
	for all $a < b$, $a, b \in \real$.
\end{defn}

Denote the set of all absolutely integrable functions on $\real$
by $\absint$. Clearly $\absint$ is a complex vector space equipped
with the linear functional
	\begin{displaymath}
	f \mapsto \int_\real f
	\end{displaymath}
Moreover, if $f \in \absint$ and $f > 0$, then
$\myintb{f(x)}{x} > 0$.

%%%%%%%%%%%%%%%%%%%%%%%%%%%%%%%%%%%%%%%%%%%%%%%%%%%%
%% Characterization of a.i.
%%%%%%%%%%%%%%%%%%%%%%%%%%%%%%%%%%%%%%%%%%%%%%%%%%%%

\begin{prop}
	Suppose $f:\real \to \cmplx$ is Riemann Integrable
	on $[a,b]$ for all $[a,b] \subset \real$. Then the
	following are equivalent
	
		\begin{enumerate}[i)]
			\item
			$f \in \absint$
			
			\item
			There exists $M > 0$ such that
				\begin{displaymath}
					\myinta{a}{b}{\modulus{f(x)}}{x}
						\leq M
				\end{displaymath}
			whenever $a < b$.
			
			\item
			For all $\epsilon > 0$ there exists $A > 0$
			such that
				\begin{displaymath}
					\myinta{a}{b}{\modulus{f(x)}}{x}
						< \epsilon
				\end{displaymath}
			whenever $[a,b] \cap [-A,A] = \varnothing$.
		\end{enumerate}
\end{prop}

\begin{proof}
	\begin{enumerate}[]
	\item
	i) $\Rightarrow$ ii)
	If $0 \leq a < b$ then, since $f$ is absolutely
	integrable on $[0,+\infty)$, we have
		\begin{displaymath}
			\myinta{a}{b}{\modulus{f(x)}}{x}
				\leq \myinta{0}{b}{\modulus{f(x)}}{x}
				\leq M_1
		\end{displaymath}
	for some $M_1 > 0$.
	
	Similarly, since $f$ is absolutely integrable on
	$(-\infty,0]$, there exists $M_2 > 0$ such that
		\begin{displaymath}
			\myinta{a}{b}{\modulus{f(x)}}{x}
				\leq M_2
		\end{displaymath}
	whenever $a < b \leq 0$.

	Let $M = M_1 + M_2$. Then ii) is satisfied with
	this choice of $M$ whenever $0 \leq a < b$ or
	$a < b \leq 0$. If $a < 0 < b$ then we also have
		\begin{IEEEeqnarray*}{rCl}
				\myinta{a}{b}{\modulus{f(x)}}{x}
					& = & \myinta{a}{0}{\modulus{f(x)}}{x}
					+ \myinta{0}{b}{\modulus{f(x)}}{x} \\
				& \leq & M_1 + M_2 \\
				= M
		\end{IEEEeqnarray*}

	\item
	ii) $\Rightarrow$ i)
	There exists $M > 0$ such that
		\begin{displaymath}
			\myinta{0}{\lambda}{\modulus{f(x)}}{x} \leq M
		\end{displaymath}
	for all $\lambda > 0$. Thus, by definition, $f$ is
	absolutely integrable on $[0,+\infty)$. Similarly, $f$
	is absolutely integrable on $(-\infty,0]$. So $f \in
	\absint$.
	
	\item
	i) $\Rightarrow$ iii)
	If $f \in \absint$ then it is absolutely integrable on $[0,+\infty)$.
	This means
		\begin{displaymath}
			\lim_{b \rightarrow \infty}
				\myinta{0}{b}{\modulus{f(x)}}{x}
		\end{displaymath}
	exists. Fix $\epsilon > 0$. By the Cauchy criterion, there exists
	$A_1 > 0$ such that
		\begin{displaymath}
			\myinta{a}{b}{\modulus{f(x)}}{x} = 
				\myinta{0}{b}{\modulus{f(x)}}{x}
				- \myinta{0}{a}{\modulus{f(x)}}{x}
				< \epsilon
		\end{displaymath}
	whenever $A_1 < a < b \in \real$. A similar argument shows that if $f \in
	\absint$ then there exists $A_2 > 0$ such that
		\begin{displaymath}
			\myinta{a}{b}{\modulus{f(x)}}{x}
				= \myinta{a}{0}{\modulus{f(x)}}{x}
				- \myinta{b}{0}{\modulus{f(x)}}{x}
				< \epsilon
		\end{displaymath}
	whenever $a < b < -A_2$. Let $A = \max (A_1, A_2)$. Then $[a,b] \cap
	[-A,A] = \varnothing$ iff $a < b < -A \leq -A_2$
	or $A_1 \leq A < a < b$. The result then follows.
	
	\item
	iii) $\Rightarrow$ i)
	Fix $\epsilon > 0$. Then by assumption there exists $A > 0$ such that
		\begin{displaymath}
			\myinta{a}{b}{\modulus{f(x)}}{x} < \epsilon
		\end{displaymath}
	whenever $b > a > A$. Hence
		\begin{displaymath}
			\lim_{a,b \rightarrow \infty} \myinta{a}{b}{\modulus{f(x)}}{x}
				= 0
		\end{displaymath}
	So by the Cauchy criterion, $f$ is absolutely integrable on $[0,+\infty)$.
	A similar argument shows that $f$ is absolutely integrable on $(-\infty,0]$.
	Therefore, $f \in \absint$ as required.
			
	\end{enumerate}
\end{proof}

%%%%%%%%%%%%%%%%%%%%%%%%%%%%%%%%%%%%%%%%%%%%%%%%%%%%
%% Defn of 1-norm
%%%%%%%%%%%%%%%%%%%%%%%%%%%%%%%%%%%%%%%%%%%%%%%%%%%%

Sometimes, the quantity
	\begin{displaymath}
		\myintb{\modulus{f(x)}}{x}
	\end{displaymath}
is denoted by $\norm[1]{f}$. This is called the 
\emph{1-norm} of $f$. This terminology is indeed
justified, as the next theorem shows.

%%%%%%%%%%%%%%%%%%%%%%%%%%%%%%%%%%%%%%%%%%%%%%%%%%%%
%% 1-norm is a norm
%%%%%%%%%%%%%%%%%%%%%%%%%%%%%%%%%%%%%%%%%%%%%%%%%%%%

\begin{thrm}
	The function $\norm[1]{\cdot}: \absint \to \real$ is
	a norm provided we identify functions which agree
	almost everywhere.
\end{thrm}

\begin{proof}
	Clearly, if $f \in \absint$ then
		\begin{displaymath}
			\norm[1]{f}
				= \myintb{\modulus{f(x)}}{x} \geq 0
		\end{displaymath}
	with equality iff $f = 0$ almost everywhere.
	
	If in addition, $\lambda \in \cmplx$ then
		\begin{displaymath}
			\norm[1]{\lambda f}
				= \myintb{\modulus{\lambda f(x)}}{x}
				= \modulus{\lambda}
				\myintb{\modulus{f(x)}}{x}
				= \modulus{\lambda} \norm[1]{f}
		\end{displaymath}

	Finally, the triangle inequality holds because if
	$f,g \in \absint$ then
		\begin{IEEEeqnarray*}{rCl}
			\norm[1]{f + g} & = &
				\myintb{\modulus{f(x)+g(x)}}{x} \\
			& \leq & \myintb{[\modulus{f(x)}
				+ \modulus{g(x)}]}{x} \\
			& = & \myintb{\modulus{f(x)}}{x}
				+ \myintb{\modulus{g(x)}}{x} \\
			& = & \norm[1]{f} + \norm[1]{g}
		\end{IEEEeqnarray*}
\end{proof}

%%%%%%%%%%%%%%%%%%%%%%%%%%%%%%%%%%%%%%%%%%%%%%%%%%%%
%% Convergence of fn to f with respect to 1-norm
%%%%%%%%%%%%%%%%%%%%%%%%%%%%%%%%%%%%%%%%%%%%%%%%%%%%

\begin{prop}
	Given $f \in \absint$ define
		\begin{displaymath}
			f_n(x) =
				\begin{cases}
					f(x) & \text{ if } \modulus{x} \leq n \\
					0 & \text{ if } \modulus{x} > n
				\end{cases}
		\end{displaymath}
	for $n \in \natrl$. Then $f_n \rightarrow f$
	in $\absint$ with respect to the 1-norm.
\end{prop}

\begin{proof}
	\begin{IEEEeqnarray*}{rCl}
		\norm[1]{f - f_n} & = &
			\myintb{\modulus{f(x)-f_n(x)}}{x} \\
		& = & \myinta{-\infty}{-n}{\modulus{f(x)}}{x}
			+ \myinta{n}{\infty}{\modulus{f(x)}}{x} \\
		& = & \myintb{\modulus{f(x)}}{x}
			- \myintb[n]{\modulus{f(x)}}{x} \\
	\end{IEEEeqnarray*}
	$\rightarrow 0$ as $n \rightarrow \infty$ since
	$f \in \absint$.
\end{proof}
			
%%%%%%%%%%%%%%%%%%%%%%%%%%%%%%%%%%%%%%%%%%%%%%%%%%%%
%% The set of fns with compact support is dense
%% in the set of a.i. fns
%%%%%%%%%%%%%%%%%%%%%%%%%%%%%%%%%%%%%%%%%%%%%%%%%%%%

\begin{thrm}
	A function $\varphi:\real \to \cmplx$ is said to
	have \emph{compact support} if it vanishes outside
	a compact set. Let
		\begin{displaymath}
			C_c(\real) = \{\varphi:\real \to \cmplx:
				\varphi \text{ is continuous and has compact
				support} \}
		\end{displaymath}
	Then, $C_c(\real)$ is a dense linear subspace of
	$\absint$.
\end{thrm}

\begin{proof}
	Let $f \in \absint$. Then given $\epsilon > 0$
	there exists $N_1 > 0$ such that
		\begin{displaymath}
			\myintb{\modulus{f(x)}}{x}
				- \myintb[n]{\modulus{f(x)}}{x}
				= \myinta{-\infty}{-n}{\modulus{f(x)}}{x}
				+ \myinta{n}{\infty}{\modulus{f(x)}}{x}
				< \epsilon/2
		\end{displaymath}
	whenever $n > N_1$.
	
	Also, given $n \in \natrl$, there exist a step
	function $s_n(x):\real \to \cmplx$ and a continuous
	function $\varphi_n(x):\real \to \cmplx$, both
	of which vanish outside $[-n,n]$, such that
		\begin{displaymath}
			\myintb[n]{\modulus{f(x) - s_n(x)}}{x}
				< 1/2n
		\end{displaymath}
	and
		\begin{displaymath}
			\myintb[n]{\modulus{s_n(x) - \varphi_n(x)}}{x}
				< 1/2n
		\end{displaymath}
	
	Choose $N_2 > 0$ such that $1/N_2 < \epsilon/2$ and let
	$N = \max(N_1,N_2)$. Then for $n > N$, we have
		\begin{IEEEeqnarray*}{rCl}
			\norm[1]{f - \varphi_n} & = &
				\myintb{\modulus{f(x) - \varphi_n(x)}}{x} \\
			& = & \myinta{-\infty}{-n}{\modulus{f(x)}}{x}
				+ \myinta{n}{\infty}{\modulus{f(x)}}{x}
				+ \myintb[n]{\modulus{f(x) - \varphi_n(x)}}{x} \\
			& < & \epsilon/2 + \myintb[n]{\modulus{f(x) - s_n(x)}}{x}
				+ \myintb[n]{\modulus{s_n(x) - \varphi_n(x)}}{x} \\
			& < & \epsilon/2 + 1/2n + 1/2n \\
			& < & \epsilon/2 + \epsilon/2 \\
			& = & \epsilon
		\end{IEEEeqnarray*}
	and so $\norm[1]{f - \varphi_n} \rightarrow 0$ as $n \rightarrow
	\infty$.
\end{proof}

%%%%%%%%%%%%%%%%%%%%%%%%%%%%%%%%%%%%%%%%%%%%%%%%%%%%
%% A function bounded by a a.i function is a.i
%%%%%%%%%%%%%%%%%%%%%%%%%%%%%%%%%%%%%%%%%%%%%%%%%%%%

\begin{prop}
	Suppose $f:\real \to \cmplx$ and $f$ is Riemann Integrable on $[a,b]
	\subset \real$ for all $a < b$. Furthermore, suppose there is a
	function $h \in \absint$ such that $\modulus{f(x)} \leq h(x)$ for
	all $x \in \real$. Then $f \in \absint$.
\end{prop}

\begin{proof}
	For $0 < \lambda \in \real$ we have
	\begin{displaymath}
	\myinta{0}{\lambda}{\modulus{f(x)}}{x}
	\leq \myinta{0}{\lambda}{h(x)}{x}
	\end{displaymath}
	
	Hence,
	\begin{displaymath}
	\lim_{\lambda \rightarrow +\infty}
	\myinta{0}{\lambda}{\modulus{f(x)}}{x}
	\leq \lim_{\lambda \rightarrow +\infty}
	\myinta{0}{\lambda}{h(x)}{x}
	\end{displaymath}
	and this latter limit exists since $h \in \absint$. Therefore,
	$f$ is absolutely integrable on $[0,+\infty)$. A Similar argument
	shows that $f$ is absolutely integrable on $(-\infty,0]$. So,
	by the definition of absolute integrability on $\real$, $f \in
	\absint$.
\end{proof}

\end{section}
