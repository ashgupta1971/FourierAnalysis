\begin{section}{Dini's Theorem}

Dini's Theorem gives another set of conditions
for a function $f$ to satisfy under which its
Fourier series converges. But before proving this
theorem, we need some preliminary results.

%% 1st preliminary result to prove Dini's theorem
\begin{thrm}\label{thrm:Dini1}
	If
		\begin{displaymath}
			g(t) =
				\begin{cases}
					\displaystyle{
						\frac{\sin t}{t}} & t > 0 \\
					1 & t = 0
				\end{cases}
		\end{displaymath}
	then
		\begin{displaymath}
			\myinta{0}{\infty}{g(t)}{t}
				= \lim_{R \rightarrow \infty}
				\myinta{0}{R}{\frac{\sin t}{t}}{t}
				= \pi/2
		\end{displaymath}
\end{thrm}

\begin{proof}
	Let $f(z) = e^{iz}/z$ for $z \in \cmplx \diagdown
	\{0\}$. For $r > 0$ let $\gamma_r(t) = r e^{it},
	0 \leq t \leq \pi$. We will now divide the proof 
	into several steps.
	\begin{enumerate}[{Step} 1.]
		\item
			First of all, we have
				\begin{IEEEeqnarray*}{rCl}
					\myinta{\epsilon}{R}{f(z)}{z}
						+ \myinta{-R}{-\epsilon}{f(z)}{z}
						& = & \myinta{\epsilon}{R}{\frac{e^{it}}{t}}{t}
						+ \myinta{-R}{-\epsilon}{\frac{e^{it}}{t}}{t} \\
					& = & \myinta{\epsilon}{R}{\frac{e^{it}}{t}}{t}
						+ \myinta{R}{\epsilon}{-\frac{e^{-is}}{-s}}{s} \\
					& = & \myinta{\epsilon}{R}{\frac{e^{it}}{t}}{t}
						- \myinta{\epsilon}{R}{\frac{e^{-is}}{s}}{s} \\
					& = & 2i \myinta{\epsilon}{R}{\frac{\sin t}{t}}{t}
				\end{IEEEeqnarray*}
		
		\item
			Now note that for $\epsilon > 0$,
				\begin{IEEEeqnarray*}{l}
					\modulus{\myintc[\gamma_\epsilon]{\frac{e^{iz}}{z}}{z}
						- \myintc[\gamma_\epsilon]{\frac{1}{z}}{z}} \; (\ast) \\
					= \modulus{\myintc[\gamma_\epsilon]
						{\frac{e^{iz}-1}{z}}{z}} \\
					\leq M \pi \epsilon
				\end{IEEEeqnarray*}
			where
				\begin{displaymath}
					M = \max \left\{ \modulus{\frac{e^{iz}-1}{z}}:
						\modulus{z}=\epsilon \right\}
				\end{displaymath}
			But
				\begin{displaymath}
					\lim_{z \rightarrow 0} \frac{e^{iz}-1}{z} = i
				\end{displaymath}
			so the expression in $(\ast)$ goes to zero as
			$\epsilon \rightarrow 0$. Noting that
				\begin{displaymath}
					\myintc[\gamma_\epsilon]{\frac{1}{z}}{z}
						= \myinta{0}{\pi}{\frac{
						\epsilon i e^{it}}{\epsilon e^{it}}}{t}
						= \pi i
				\end{displaymath}
			we get
				\begin{displaymath}
					\lim_{\epsilon \rightarrow 0+}
						\myintc[\gamma_\epsilon]{\frac{e^{iz}}{z}}{z}
						= \pi i
				\end{displaymath}

		\item
			The final integral we need to evaluate is
				\begin{IEEEeqnarray*}{rCl}
					\modulus{\myintc[\gamma_R]{f(z)}{z}} & = &
						\modulus{\myinta{0}{\pi}{\frac
						{e^{i(R\cos t + iR\sin t)}}{R e^{it}}
						i R e^{it}}{t}} \\
					& \leq & \myinta{0}{\pi}{e^{-R\sin t}}{t} \\
					& = & 2 \myinta{0}{\pi/2}{e^{-R\sin t}}{t}
				\end{IEEEeqnarray*}
			But $2/\pi \leq (\sin t)/2 \leq 1$ for $0 < t \leq \pi/2$.
			So $-R\sin t \leq -2Rt/\pi$ for $0 \leq t \leq \pi/2$.
			Hence
				\begin{IEEEeqnarray*}{rCl}
					2 \myinta{0}{\pi/2}{e^{-R\sin t}}{t}
						& \leq & 2 \myinta{0}{\pi/2}{e^{-2Rt/\pi}}{t} \\
					& = & \frac{\pi}{R} (1 - e^{-R})
				\end{IEEEeqnarray*}
			$\rightarrow 0$ as $R \rightarrow \infty$.
	
		\item
			Since $f$ is analytic on the starlike region	$\cmplx
			\diagdown \{iy: y \leq 0\}$, Cauchy's Theorem says that
				\begin{IEEEeqnarray*}{l}
					0 = \myinta{\epsilon}{R}{f(z)}{z} +
						\myintc[\gamma_R]{f(z)}{z} \\
					+ \; \myinta{-R}{-\epsilon}{f(z)}{z}
						+ \myintc[\gamma_\epsilon]{f(z)}{z}
				\end{IEEEeqnarray*}
			So putting the above results together we get
				\begin{IEEEeqnarray*}{rCl}
					2i \myinta{\epsilon}{R}{\frac{\sin t}{t}}{t}
						& = & \myintc[\gamma_\epsilon]{f(z)}{z}
						- \myintc[\gamma_R]{f(z)}{z} \\
					& = & \pi i - \myintc[\gamma_R]{f(z)}{z}
				\end{IEEEeqnarray*}
			Letting $R \rightarrow \infty$ we see that
				\begin{displaymath}
					2i \myinta{\epsilon}{\infty}{\frac{\sin t}{t}}{t}
						= \pi i
				\end{displaymath}
			But from Corollary \ref{thrm:IntegralConv},
				\begin{displaymath}
					\myinta{0}{\infty}{\frac{\sin t}{t}}{t}
						= \lim_{\epsilon \rightarrow 0^+}
						\myinta{\epsilon}{\infty}{\frac{\sin t}{t}}{t}		
				\end{displaymath}
			and so
				\begin{displaymath}
					\myinta{0}{\infty}{\frac{\sin t}{t}}{t}
						= \pi/2
				\end{displaymath}
			as required.
	\end{enumerate}
\end{proof}

%% Corollary to previous theorem
\begin{cor}\label{cor:Dini2}
	\begin{enumerate}[i)]
		\item
			For any $a > 0$
				\begin{displaymath}
					\lim_{\lambda \rightarrow \infty}
						\myinta{0}{a}{\frac{\sin \lambda t}{t}}{t}
						= \pi/2
				\end{displaymath}
		\item
			For any $b > 0$
				\begin{displaymath}
					\myinta{0}{\infty}{\frac{\sin bt}{t}}{t}
						= \pi/2
				\end{displaymath}
	\end{enumerate}
\end{cor}

\begin{proof}
	\begin{enumerate}[i)]
		\item
			\begin{IEEEeqnarray*}{rCl}
				\myinta{0}{a}{\frac{\sin \lambda t}{t}}{t}
					& = & \myinta{0}{\lambda a}{\frac
					{\sin x}{x/\lambda}\frac{1}{\lambda}}{x}
					\text{ where } x = \lambda t \\
				& = & \myinta{0}{\lambda a}{\frac{\sin x}{x}}{x}
			\end{IEEEeqnarray*}
			$\rightarrow \pi/2$ as $\lambda \rightarrow \infty$
			by Theorem \ref{thrm:Dini1}.
		\item
			This proven similarly to that in part i).
	\end{enumerate}
\end{proof}

%% 2nd corollary to Dini's mini theorem
\begin{cor}\label{cor:Dini3}
	Suppose $\delta > 0$ and $g:[0,\delta] \to \cmplx$ is Riemann
	Integrable on its domain. Furthermore, suppose $g(0+) =
	\lim_{t \rightarrow 0^+} g(t)$ exists and
		\begin{displaymath}
			\myinta{0}{\delta}{\frac{g(t)-g(0+)}{t}}{t}
		\end{displaymath}
	exists. Then
		\begin{displaymath}
			\lim_{\substack{\lambda \rightarrow \infty \\
				\lambda \in \real}} \myinta{0}{\delta}
				{\frac{g(t)}{t}\sin \lambda t}{t}
				= \frac{\pi}{2} g(0+)
		\end{displaymath}
\end{cor}

\begin{proof}
	By Theorem \ref{thrm:RLL}
		\begin{IEEEeqnarray*}{rCl}
			0 & = & \lim_{\substack{\lambda \rightarrow \infty \\
				\lambda \in \real}} \myinta{0}{\delta}
				{\frac{g(t)-g(0+)}{t}\sin \lambda t}{t} \\
			& = & \lim_{\substack{\lambda \rightarrow \infty \\
				\lambda \in \real}} \myinta{0}{\delta}
				{\frac{g(t)}{t}\sin \lambda t}{t}
				- \lim_{\substack{\lambda \rightarrow \infty \\
				\lambda \in \real}} g(0+) \myinta{0}{\delta}
				{\frac{\sin \lambda t}{t}}{t}
		\end{IEEEeqnarray*}
	But, by Corollary \ref{cor:Dini2},
		\begin{displaymath}
			\lim_{\substack{\lambda \rightarrow \infty \\
				\lambda \in \real}} \myinta{0}{\delta}
				{\frac{\sin \lambda t}{t}}{t}
				= \pi/2
		\end{displaymath}
	and so the result follows.
\end{proof}

%%%%%%%%%%%%%%%%%%%%%%%%%%%%%%%%%%%%%%%%%%%%%%%%%%%
%% Dini's Theorem
%%%%%%%%%%%%%%%%%%%%%%%%%%%%%%%%%%%%%%%%%%%%%%%%%%%

\begin{thrm}[Dini]\label{thrm:Dini}
	Suppose $f$ is a $2\pi$-periodic, Riemann Integrable
	function on $\real$. Furthermore suppose for some $x_0 \in
	\real$ and $\delta > 0$ the following all exist:
		\begin{enumerate}[i)]
			\item
				$f(x_0+) = \lim_{x \rightarrow x_0^+} f(x)$
			\item
				$f(x_0-) = \lim_{x \rightarrow x_0^-} f(x)$
			\item
				\begin{displaymath}
					\myinta{0}{\delta}{\frac{f(x_0+t)-f(x_0+)}{t}}{t}
				\end{displaymath}
			\item
				\begin{displaymath}
					\myinta{0}{\delta}{\frac{f(x_0-t)-f(x_0-)}{t}}{t}
				\end{displaymath}
		\end{enumerate}
	Then
		\begin{displaymath}
			\lim_{n \rightarrow \infty} s_n(x_0)
				= \frac{f(x_0+)+f(x_0-)}{2}
		\end{displaymath}
	where $s_n$ is the $n$th partial sum of the Fourier series
	of $f$.
\end{thrm}

\begin{proof}
	Since $D_n$ is even, Theorem \ref{thrm:Dirch} implies
	$\myinta{0}{\pi}{D_n(t)}{t} = \pi$. Therefore,
		\begin{IEEEeqnarray*}{l}
			\frac{1}{\pi} \myinta{0}{\pi}
				{f(x_0+t)D_n(t)}{t} - f(x_0+) \\
			= \frac{1}{\pi} \myinta{0}{\pi}
				{f(x_0+t)D_n(t)}{t} - \frac{f(x_0+)}{\pi}
				\myinta{0}{\pi}{D_n(t)}{t} \\
			= \frac{1}{\pi} \myinta{0}{\pi}
				{[f(x_0+t)-f(x_0+)]D_n(t)}{t} \\
			= \frac{1}{\pi} \myinta{0}{\pi}
				{g(t) \sin (n+1/2)t}{t}
		\end{IEEEeqnarray*}
	where
		\begin{displaymath}
			g(t) = 2 \left( \frac{f(x_0+t)-f(x_0+)}{t}
				\right) \frac{t/2}{\sin t/2}
		\end{displaymath}
	Now the hypothesis implies $g$ is Riemann Integrable
	on $[0,\pi]$ so by Theorem \ref{thrm:RLL}
		\begin{displaymath}
			\lim_{n \rightarrow \infty}
				\frac{1}{\pi} \myinta{0}{\pi}
				{g(t) \sin (n+1/2)t}{t}
				= 0
		\end{displaymath}
	Therefore,
		\begin{displaymath}
			\lim_{n \rightarrow \infty}
				\frac{1}{\pi} \myinta{0}{\pi}
				{f(x_0+t)D_n(t)}{t} = f(x_0+)
		\end{displaymath}
	A similar argument shows that
		\begin{displaymath}
			\lim_{n \rightarrow \infty}
				\frac{1}{\pi} \myinta{0}{\pi}
				{f(x_0-t)D_n(t)}{t} = f(x_0-)
		\end{displaymath}
	To complete the proof,
		\begin{IEEEeqnarray*}{rCl}
			s_n(x_0) & = & \frac{1}{2\pi} \left(
				\myinta{0}{\pi}{f(x_0+t)D_n(t)}{t}
				+ \myinta{0}{\pi}{f(x_0-t)D_n(t)}{t}
				\right) \\
			& = & \frac{1}{2} \left(
				\frac{1}{\pi} \myinta{0}{\pi}{f(x_0+t)D_n(t)}{t}
				+ \frac{1}{\pi} \myinta{0}{\pi}{f(x_0-t)D_n(t)}{t}
				\right)
		\end{IEEEeqnarray*}
	$\rightarrow (f(x_0+)+f(x_0-))/2$ as $n \rightarrow \infty$.
\end{proof}

%% Final corollary to Dini's Theorem
\begin{cor}
	\begin{enumerate}[i)]
		\item
			Suppose $\delta > 0$ and $g:[0,\delta] \to
			\cmplx$ is Riemann Integrable on its domain.
			Furthermore suppose $g(0+) = \lim_{x \rightarrow
			0^+} g(x)$ exists and $\modulus{g(t)-g(0+)} \leq
			M \modulus{t}$ for some $M > 0$. Then the conditions
			of Corollary \ref{cor:Dini3} are satisfied.
		\item
			Suppose that $f$ is a $2\pi$-periodic, Riemann Integrable
			function on $\real$ and for some $x_0 \in \real$ and $M,
			\delta > 0$ we have $\modulus{f(x_0+t)-f(x_0)} \leq M
			\modulus{t}, -\delta \leq t \leq \delta$. Then the
			conditions of Theorem \ref{thrm:Dini} are satisfied.
		\item
			If $f$ satisfies the hypothesis of Theorem \ref{thrm:Jordan}
			then it also satisfies the hypothesis of Theorem
			\ref{thrm:Dini}.
	\end{enumerate}
\end{cor}

\begin{proof}
	\begin{enumerate}[i)]
		\item
			Let $h(t) = (g(t)-g(0+))/t$. Then $\modulus{h(t)} \leq
			M$ for $t \in [0,\delta]$. Also, $h$ is Riemann Integrable
			on $[\epsilon,\delta]$ for all $0 < \epsilon < \delta$.
			Therefore, by Theorem \ref{thrm:IntegralConv} $h$ is
			Riemann Integrable on $[0,\delta]$ and we have satsified
			all the conditions of Corollary \ref{cor:Dini3}.
		\item
			Since
				\begin{displaymath}
					\lim_{t \rightarrow 0} \modulus{f(x_0+t)-f(x_0)}
						\leq \lim_{t \rightarrow 0} M \modulus{t}
						= 0
				\end{displaymath}
			we have $f(x_0+) = f(x_0-) = f(x_0)$. Futhermore, the
			function $g:(0,\delta] \to \cmplx$ defined by $g(t)
			= (f(x_0+t)-f(x_0+))/t$ is bounded on its domain and
			Riemann Integrable on $[\epsilon,\delta]$ and
			$[-\delta,-\epsilon]$, for all
			$0 < \epsilon < \delta$. Therefore, by Theorem
			\ref{thrm:IntegralConv} $g$ is Riemann Integrable
			on $[0,\delta]$ and and $[-\delta,0]$ and we have satisfied
			all the conditions of Theorem \ref{thrm:Dini}.
		\item
			This follows from Theorem \ref{thrm:Jordan} and Corollary
			\ref{cor:IntegralConv}.
	\end{enumerate}
\end{proof}

\end{section}
