\begin{section}{The Representation Problem}
	The \emph{Representation Problem} asks if
	the Fourier series of a given function $f$
	converges pointwise to $f$. In this section,
	we will deduce two results which answer this
	question in certain cases. Also, the Riemann
	Localization Theorem will be presented.
	
%%%%%%%%%%%%%%%%%%%%%%%%%%%%%%%%%%%%%%%%%%%%%%%%
%% Dirichlet's convergence theorem
%%%%%%%%%%%%%%%%%%%%%%%%%%%%%%%%%%%%%%%%%%%%%%%%

\begin{thrm}\label{thrm:DirchConv}
	Suppose $f$ is a $2\pi$-periodic, Riemann
	Integrable function and $s_n$ is the $n$th
	partial sum of its Fourier series. If $f$
	is differentiable at some $x \in \real$
	then $s_n$ converges pointwise to $f$
	at $x$.
\end{thrm}

%% Intuitive proof of Dirichlet's convergence theorem
	Before presenting the proof, we will give an intuitive
	justification of this result. If $n \in \natrl$ is large
	and $0 < \delta < \pi$ then
		\begin{displaymath}
			\myinta{\delta}{\pi}{f(x+t) D_n(t)}{t}
				\simeq 0
		\end{displaymath}
	and
		\begin{displaymath}
			\myinta{-\pi}{-\delta}{f(x+t) D_n(t)}{t}
				\simeq 0
		\end{displaymath}
	by Theorem \ref{thrm:Dirch}. Therefore, if
	$0 < \delta < \pi$ is small and $n \in \natrl$
	is large, we can use the continuity of $f$
	at $x$ to deduce that
		\begin{IEEEeqnarray*}{rCl}
			f(x) & = & \frac{1}{2\pi} \myintb[\pi]
				{f(x) D_n(t)}{t} \\
			& \simeq & \frac{1}{2\pi}
				\myinta{-\pi}{-\delta}{f(x+t) D_n(t)}{t}
				+ \frac{1}{2\pi} \myintb[\delta]{f(x+t) D_n(t)}{t} \\
			& & + \; \frac{1}{2\pi} \myinta{\delta}{\pi}
				{f(x+t) D_n(t)}{t} \\
			& = & \frac{1}{2\pi} \myintb[\pi]{f(x+t) D_n(t)}{t} \\
			& = & s_n(x)
		\end{IEEEeqnarray*}

%% Formal proof of convergence theorem
\begin{proof}
	\begin{IEEEeqnarray*}{rCl}
		s_n(x) - f(x) & = & \frac{1}{2\pi} \myintb[\pi]
			{f(x+t)D_n(t)}{t}	- \frac{f(x)}{2\pi}
			\myintb[\pi]{D_n(t)}{t} \\
		& = & \frac{1}{2\pi} \myintb[\pi]{[f(x+t)-f(x)]
			D_n(t)}{t} \\
		& = & \frac{1}{2\pi} \myintb[\pi]{g(t) \sin (n+1/2)t}{t}
	\end{IEEEeqnarray*}
	where $g:[-\pi,\pi] \to \cmplx$ is defined by
		\begin{displaymath}
			g(t) =
				\begin{cases}
					\displaystyle{
						\frac{f(x+t)-f(x)}{\sin t/2}}
						& \text{if } 0 < \modulus{t} \leq \pi \\
					2f'(x) & \text{if } t = 0
				\end{cases}
		\end{displaymath}
	Note that $g$ is continous at zero because
		\begin{displaymath}
			g(t) = 2 \left( \frac{f(x+t)-f(x)}{t} \right)
				\frac{t/2}{\sin x/2}
		\end{displaymath}
	$\rightarrow 2f'(x) = g(0)$ as $t \rightarrow 0$.
	Now $g$ is Riemann Integrable on $[\delta,\pi]$ and
	$[-\pi,-\delta]$ for all $0 < \delta < \pi$. So by
	Corollary \ref{cor:IntegralConv} $g$ is Riemann
	Integrable on $[-\pi,\pi]$. Therefore by Theorem
	\ref{thrm:RLL},
		\begin{displaymath}
			s_n(x) - f(x) =
				\frac{1}{2\pi} \myintb[\pi]{g(t) \sin (n+1/2)t}{t}
		\end{displaymath}
	$\rightarrow 0$ as $n \rightarrow \infty$.
\end{proof}

%% Corollary to Dirichlet's convergence theorem
\begin{cor}\label{cor:FourierUniformConv2}
	If $f \in \ctsdiff[2]$ is $2\pi$-periodic then the Fourier
	series of $f$ converges uniformly on $\real$ to $f$.
\end{cor}

\begin{proof}
	This follows immediately from Theorem \ref{thrm:DirchConv}
	and Corollary \ref{cor:FourierUniformConv}.
\end{proof}

%%%%%%%%%%%%%%%%%%%%%%%%%%%%%%%%%%%%%%%%%%%%%%%%
%% Jordan's convergence theorem
%%%%%%%%%%%%%%%%%%%%%%%%%%%%%%%%%%%%%%%%%%%%%%%%

\begin{thrm}[Jordan]\label{thrm:Jordan}
	Suppose $f$ is a $2\pi$-periodic, Riemann Integrable
	function and $s_n$ the $n$th partial sum of its
	Fourier series. Furthermore, suppose there exists
	$x_0 \in \real$ and $\delta > 0$ such that $f$ is
	differentiable on $(x_0,x_0+\pi)$ and $(x_0-\pi,x_0)$.
	If $\lambda_+ = \lim_{x \rightarrow x_0^+}f'(x)$ and
	$\lambda_- = \lim_{x \rightarrow x_0^-}f'(x)$ both
	exist then
		\begin{enumerate}[i)]
			\item
				$f(x_0+) = \lim_{x \rightarrow x_0^+}f(x)$
				exists.
			\item
				$f(x_0-) = \lim_{x \rightarrow x_0^-}f(x)$
				exists.
			\item
				\begin{displaymath}
					\lim_{x \rightarrow x_0^+}\frac{f(x)-f(x_0+)}
						{x-x_0} = \lambda_+
				\end{displaymath}
			\item
				\begin{displaymath}
					\lim_{x \rightarrow x_0^-}\frac{f(x)-f(x_0-)}
						{x-x_0} = \lambda_-
				\end{displaymath}
			\item
				\begin{displaymath}
					\lim_{n \rightarrow \infty}s_n(x)
						= \frac{f(x_0+)+f(x_0-)}{2}
				\end{displaymath}
		\end{enumerate}
\end{thrm}

\begin{proof}
	\begin{enumerate}[i)]
		\item
			Since $\lim_{x \rightarrow x_0^+}f'(x)$ exists,
			there are $M > 0$ and $\delta_0 > 0$ such 
			that $\modulus{f'(x)} <	M$ for $x_0 < x \leq x_0
			+ \delta_0$. Fix $\epsilon > 0$ and let $\delta'
			= \min(\delta_0,\epsilon/2M)$. Then, by the Mean
			Value Theorem, if $x,y \in (x_0,x_0+\delta']$
			with $x < y$,
				\begin{IEEEeqnarray*}{rCl}
					\modulus{f(x)-f(y)} & = & \modulus{f'(t)}
						\modulus{x-y} \text{ where }
						t \in (x,y) \subset (x_0,x_0+\delta') \\
					& < & M \frac{\epsilon}{2M} \\
					& = & \epsilon/2
				\end{IEEEeqnarray*}
			Therefore, if we let $z = x_0 + \delta'$, then
				\begin{displaymath}
					f(z) - \epsilon/2 \leq
						\limsup_{x \rightarrow x_0^+}f(x)
						\leq f(z) + \epsilon/2
				\end{displaymath}
			Similarly, 
				\begin{displaymath}
					f(z) - \epsilon/2 \leq
						\liminf_{x \rightarrow x_0^+}f(x)
						\leq f(z) + \epsilon/2
				\end{displaymath}
			So we have $\limsup_{x \rightarrow x_0^+}f(x)
			- \liminf_{x \rightarrow x_0^+}f(x) \leq \epsilon$
			and the result follows.
		\item
			This limit is handled similarly as that in part i).
		\item
			Let $g(x) = f(x)$ for $x \in (x_0,x_0+\delta)$ and
			$g(x_0) = f(x_0+)$. Fix $\epsilon > 0$ and choose
			$0 < \delta_0 < \delta$ so that $\modulus{f'(x)-\lambda_+}
			< \epsilon$ if $x_0 < x < x_0+\delta_0$. Now by
			part i), $g$ is continuous on $[x_0,x_0+\delta_0]$ and
			differentiable on $(x_0,x_0+\delta_0)$. So by the
			Mean Value Theorem, if $x \in (x_0,x_0+\delta_0)$
				\begin{displaymath}
					\frac{f(x) - f(x_0+)}{x-x_0} =
						\frac{g(x) - g(x_0)}{x-x_0} =
						g'(t)
				\end{displaymath}
			for some $t \in (x_0,x) \subset (x_0,x_0+\delta_0)$.
			So $g'(t) = f'(t)$ and hence,
				\begin{displaymath}
					\modulus{\frac{f(x) - f(x_0+)}{x-x_0}
						- \lambda_+} = \modulus{f'(t) - \lambda_+}
						< \epsilon
				\end{displaymath}
			if $x_0 < x < x_0+\delta_0$.
		\item
			This result is handled similarly to part iii).
		\item
			Since $D_n$ is even, Theorem \ref{thrm:Dirch}
			implies $\myinta{0}{\pi}{D_n(t)}{t} = \pi$.
			Therefore,
				\begin{IEEEeqnarray*}{rCl}
					\frac{1}{\pi}\myinta{0}{\pi}
						{f(x_0+t)D_n(t)}{t} - f(x_0+) & = &
						\frac{1}{\pi}\myinta{0}{\pi}
						{f(x_0+t)D_n(t)}{t} \\
					& & - \; \frac{1}{\pi} \myinta{0}{\pi}
						{f(x_0+)D_n(t)}{t} \\
					& = & \frac{1}{\pi}\myinta{0}{\pi}
						{[f(x_0+t) - f(x_0+)]D_n(t)}{t} \\
					& = & \frac{1}{\pi}\myinta{0}{\pi}
						{g(t)D_n(t)}{t}
				\end{IEEEeqnarray*}
			where
				\begin{displaymath}
					g(t) = 
						\begin{cases}
							\displaystyle{
								\frac{f(x_0+t)-f(x_0+)}{\sin t/2}}
									& \text{if } 0 < t < \pi \\
								2\lambda_+ & \text{if } t = 0
						\end{cases}
				\end{displaymath}
			As was argued in the proof of Theorem \ref{thrm:DirchConv},
			$g$ is Riemann Integrable on $[0,\pi]$ and
				\begin{displaymath}
					\lim_{n \rightarrow \infty} \frac{1}{\pi}
						\myinta{0}{\pi}{g(t)D_n(t)}{t} = 0
				\end{displaymath}
			Therefore,
				\begin{displaymath}
					\lim_{n \rightarrow \infty}
						\frac{1}{\pi}\myinta{0}{\pi}
						{f(x_0+t)D_n(t)}{t} = f(x_0+)
				\end{displaymath}
			A similar argument shows that
				\begin{displaymath}
					\lim_{n \rightarrow \infty}
						\frac{1}{\pi}\myinta{-\pi}{0}
						{f(x_0+t)D_n(t)}{t} = f(x_0-)
				\end{displaymath}
			To complete the proof, note that
				\begin{IEEEeqnarray*}{rCl}
					s_n(x_0) & = & \frac{1}{2\pi} \myintb[\pi]
						{f(x+t)D_n(t)}{t} \\
					& = & \frac{1}{2} \left( \frac{1}{\pi}
						\myinta{-\pi}{0}{f(x+t)D_n(t)}{t}
						+ \frac{1}{\pi} \myinta{0}{\pi}
						{f(x+t)D_n(t)}{t} \right)
				\end{IEEEeqnarray*}
			$\rightarrow (f(x_0+)+f(x_0-))/2$ as $n \rightarrow
			\infty$.
	\end{enumerate}
\end{proof}

%% Corollary to Jordan's Theorem
\begin{cor}
	Suppose $f$ is $2\pi$-periodic, Riemann Integrable
	and $f$ is $\ctsdiff[1]$ on $(x_{k-1},x_k)$ where
	$-\pi = x_0 < x_1 < \ldots < x_n = \pi$. If
	$\lim_{x \rightarrow x_k^+}f'(x)$ exists for
	$0 \leq k \leq n-1$ and $\lim_{x \rightarrow x_k^-}
	f'(x)$ exists for $1 \leq k \leq n$ then $s_n(x)
	\rightarrow (f(x+)+f(x-))/2$ for all $x \in \real$.
	(Here, $s_n$ denotes the $n$th partial sum of the
	Fourier series of $f$.)
\end{cor}

\begin{proof}
	This follows immediately from Theorem \ref{thrm:Jordan}.
\end{proof}

%%%%%%%%%%%%%%%%%%%%%%%%%%%%%%%%%%%%%%%%%%%%%%%%%%
%% Examples of Fourier series
%%%%%%%%%%%%%%%%%%%%%%%%%%%%%%%%%%%%%%%%%%%%%%%%%%

We now present several examples of Fourier series
of various functions.

%% first example
\begin{ex}\label{ex:Jordan1}
	Prove that
		\begin{displaymath}
			x^2 = \frac{\pi^2}{3} + 4\sum_{k=1}^\infty
				\frac{(-1)^k}{k^2}\cos kx
		\end{displaymath}
	for $-\pi \leq x \leq \pi$.
\end{ex}

\begin{soln}
	By Proposition \ref{prop:periodic}, the function
	$f(x) = x^2, -\pi \leq x \leq \pi$ has a unique
	$2\pi$-periodic extension to $\real$. Moreover,
	this extension is even so by Corollary
	\ref{cor:EvenOdd} $b_n = 0$ for all $n \in \natrl$.
	Now this extension of $f$ satisfies the hypothesis
	of Theorem \ref{thrm:Jordan} and is also continuous
	on $\real$ so the Fourier series of $f$ converges
	pointwise to $f$. The $a_n$'s are computed as follows:
		\begin{IEEEeqnarray*}{rCl}
			a_0 & = & \frac{1}{\pi} \myintb[\pi]{x^2}{x} \\
			& = & \evalat{\frac{x^3}{3\pi}}{x}{-\pi}{\pi} \\
			& = & \frac{2\pi^2}{3}
		\end{IEEEeqnarray*}
	and for $0 < n \in \intgr$ we have (using integration by
	parts twice)
		\begin{IEEEeqnarray*}{rCl}
			a_n & = & \frac{1}{\pi} \myintb[\pi]
				{x^2 \cos nx}{x} \\
			& = & \frac{1}{\pi} \left( \evalat{\frac
				{x^2 \sin nx}{n}}{x}{-\pi}{\pi}
				- \frac{2}{n} \myintb[\pi]{x \sin nx}{x}
				\right) \\
			& = & \frac{1}{\pi} \left( \evalat{\frac
				{2}{n^2} x \cos nx}{x}{-\pi}{\pi} +
				\myintb[\pi]{\cos nx}{x} \right) \\
			& = & \frac{1}{\pi} \left(
				\frac{(-1)^n 4\pi}{n^2} \right) \\
			& = & \frac{(-1)^n 4}{n^2}
		\end{IEEEeqnarray*}
	as required.
\end{soln}

%% second example
\begin{ex}
	Prove that
		\begin{displaymath}
			\modulus{x} = \frac{\pi}{2} - \frac{4}{\pi}
				\sum_{k=1}^\infty \frac{\cos (2k-1)x}
				{(2k-1)^2}
		\end{displaymath}
\end{ex}

\begin{soln}
	As in Example \ref{ex:Jordan1}, we consider the
	unique (even) $2\pi$-periodic extension of
	$f(x) = \modulus{x}, -\pi \leq x \leq \pi$ to
	$\real$. Then $b_n = 0$ for all $n \in \natrl$
	and we compute the $a_n$'s as follows:
		\begin{IEEEeqnarray*}{rCl}
			a_0 & = & \frac{1}{\pi} \myintb[\pi]
				{\modulus{x}}{x} \\
			& = & \frac{2}{\pi} \myinta{0}{\pi}{x}{x} \\
			& = & \frac{2}{\pi} \left( \frac{\pi^2}{2}
				\right) \\
			& = & \pi
		\end{IEEEeqnarray*}
	and for $a_n, n > 0$ we have (again, using integration
	by parts)
		\begin{IEEEeqnarray*}{rCl}
			a_n & = & \frac{1}{\pi} \myintb[\pi]
				{\modulus{x}\cos nx}{x} \\
			& = & \frac{2}{\pi} \myinta{0}{\pi}
				{x \cos nx}{x} \\
			& = & \frac{2}{\pi} \left( \evalat
				{\frac{x \sin nx}{n}}{x}{0}{\pi}
				- \frac{1}{n} \myinta{0}{\pi}
				{\sin nx}{x} \right) \\
			& = & \frac{2}{\pi} \left( \evalat
				{\frac{\cos nx}{n^2}}{x}{0}{\pi}
				\right) \; (\ast)
		\end{IEEEeqnarray*}
	Now 
		\begin{displaymath}
			\evalat{\frac{\cos nx}{n^2}}{x}{0}{\pi} =
				\begin{cases}
					0 & \text{if } n \text{ is even} \\
					\displaystyle{\frac{-2}{n^2}}
						& \text{if } n \text{ is odd}
				\end{cases}
		\end{displaymath}
	Therefore, from $(\ast)$ we get $a_{2n} = 0$ and
		\begin{displaymath}
			a_{2n-1} = -\frac{4}{\pi (2n-1)^2}
		\end{displaymath}
	Since $f$ is continuous on its domain, the Fourier
	series of $f$ converges pointwise to $f$ by Theorem
	\ref{thrm:Jordan}. 
\end{soln}

%% third example
\begin{ex}
	Define $f:[-\pi,\pi] \to \real$ by
		\begin{displaymath}
			f(x) =
				\begin{cases}
					1 & 0 < x < \pi \\
					-1 & -\pi < x < 0 \\
					0 & x=-\pi,0,\pi
				\end{cases}
		\end{displaymath}
	Find the Fourier series of $f$ and show that
	it converges pointwise to $f$.
\end{ex}

\begin{soln}
	Consider the $2\pi$-periodic extension of $f$ to
	$\real$. This extension is continuous for all
	$x \in \real, x \neq \pi l, l \in \intgr$.
	Also, for $x = \pi l, l \in \intgr$
	we have $f(x+)+f(x-) = 0 = f(x)$. Now $f$ satisfies
	the hypothesis of Theorem \ref{thrm:Jordan} and so
	the Fourier series of $f$ converges pointwise to $f$.
	Noting that $f$ is odd we get $a_n = 0$ for all $0
	\leq n \in \intgr$ and
		\begin{IEEEeqnarray*}{rCl}
			b_n & = & \frac{1}{\pi} \myintb[\pi]{f(x)\sin nx}{x} \\
			& = & \frac{2}{\pi} \myinta{0}{\pi}{f(x)\sin nx}{x} \\
			& = & \frac{2}{\pi} \myinta{0}{\pi}{\sin nx}{x} \\
			& = & \evalat{-\frac{2 \cos nx}{\pi n}}{x}{0}{\pi} \\
			& = & \frac{2}{\pi n} \left( 1 - \cos n \pi \right)
		\end{IEEEeqnarray*}
	Now
		\begin{displaymath}
			\frac{2}{\pi n} \left( 1 - \cos n \pi \right) =
				\begin{cases}
					\displaystyle{\frac{4}{\pi n}} &
						\text{if } n \text{ is odd} \\	
					0 & \text{if } n \text{ is even}
				\end{cases}
		\end{displaymath}
	Therefore,
		\begin{displaymath}
			f(x) = \frac{4}{\pi}\sum_{k=1}^\infty
				\frac{\sin (2k-1)x}{2k-1}
		\end{displaymath}
\end{soln}

%% fourth example
\begin{ex}
	Prove that
		\begin{displaymath}
			\cos ax = \frac{\sin \pi a}{\pi a}
				+ \sum_{k=1}^\infty (-1)^k
				\frac{2a\sin \pi a}{\pi (a^2-k^2)}
				\cos kx
		\end{displaymath}
	where $a \in \real, a \notin \natrl$ and 
	$-\pi \leq x \leq \pi$.
\end{ex}

\begin{soln}
	As in Example \ref{ex:Jordan1}, we consider the
	unique (even) $2\pi$-periodic extension of
	$f(x) = \cos ax, -\pi \leq x \leq \pi$ to
	$\real$. Then $b_n = 0$ for all $n \in \natrl$
	and we compute the $a_n$'s as follows:
		\begin{IEEEeqnarray*}{rCl}
			a_0 & = & \frac{1}{\pi}\myintb[\pi]{\cos ax}{x} \\
			& = & \frac{2}{\pi}\myinta{0}{\pi}{\cos ax}{x} \\
			& = & \evalat{\frac{2\sin ax}{\pi a}}{x}{0}{\pi} \\
			& = & \frac{2\sin \pi a}{\pi a}
		\end{IEEEeqnarray*}
	and for $n > 0$ we have (using integration by parts)
		\begin{IEEEeqnarray*}{rCl}
			a_n & = & \frac{1}{\pi}\myintb[\pi]
				{\cos ax\cos nx}{x} \\
			& = & \frac{2}{\pi}\myinta{0}{\pi}
				{\cos ax\cos nx}{x} \\
			& = & \frac{2}{\pi} \left( \evalat
				{\frac{\cos ax \sin nx}{n}}{x}{0}{\pi}
				+ \frac{a}{n}\myinta{0}{\pi}
				{\sin ax \sin nx}{x} \right) \\
			& = & \frac{2}{\pi} \left( \evalat
				{-\frac{a\sin ax\cos nx}{n^2}}{x}{0}{\pi}
				+ \frac{a^2}{n^2}\myinta{0}{\pi}
				{\cos ax\cos nx}{x} \right) \\
			& = & (-1)^{n+1}\frac{2a\sin \pi a}{\pi n^2}
				+ \frac{2a^2}{\pi n^2}\myinta{0}{\pi}
				{\cos ax\cos nx}{x}
		\end{IEEEeqnarray*}
	Rearranging this equation we get
		\begin{IEEEeqnarray*}{rCl}
			\frac{2}{\pi}\left(1 - \frac{a^2}{n^2}\right)
				\myinta{0}{\pi}{\cos ax\cos nx}{x}
				& = & (-1)^{n+1}\frac{2a\sin \pi a}{\pi n^2} \\
		\end{IEEEeqnarray*}
	and so
		\begin{IEEEeqnarray*}{rCl}
			a_n & = & \frac{2}{\pi}\myinta{0}{\pi}
				{\cos ax\cos nx}{x} \\
			& = & (-1)^{n+1}\frac{2a n^2 \sin \pi a}
				{\pi n^2 (n^2 - a^2)} \\
			& = & (-1)^n\frac{2a\sin \pi a}{\pi(a^2-n^2)}
		\end{IEEEeqnarray*}
	as required. (Note that by Theorem \ref{thrm:Jordan}
	this Fourier series converges to $f$ since $f$ is
	continuous on $\real$.)
\end{soln}

%% fifth example
\begin{ex}
	Find real scalars $a_0, a_1, \ldots$ such that
		\begin{displaymath}
			e^{-\modulus{x}} = \frac{a_0}{2} +
				\sum_{k=1}^\infty a_k\cos kx
		\end{displaymath}
	for $-\pi \leq x \leq \pi$.
\end{ex}

\begin{soln}
	As in Example \ref{ex:Jordan1}, we consider the
	unique (even) $2\pi$-periodic extension of
	$f(x) = e^{-\modulus{x}}, -\pi \leq x \leq \pi$ to
	$\real$. Then $b_n = 0$ for all $n \in \natrl$
	and we compute the $a_n$'s as follows:
		\begin{IEEEeqnarray*}{rCl}
			a_0 & = & \frac{1}{\pi}\myintb[\pi]
				{e^{-\modulus{x}}}{x} \\
			& = & \frac{2}{\pi}\myinta{0}{\pi}
				{e^{-x}}{x} \\
			& = & \frac{2(1-e^{-\pi})}{\pi}
		\end{IEEEeqnarray*}
	and for $n > 0$ we have (using integration by parts)
		\begin{IEEEeqnarray*}{rCl}
			a_n & = & \frac{1}{\pi}\myintb[\pi]
				{e^{-\modulus{x}}\cos nx}{x} \\
			& = & \frac{2}{\pi}\myinta{0}{\pi}
				{e^{-x}\cos nx}{x} \\
			& = & \frac{2}{\pi} \left( \evalat
				{\frac{e^{-x}\sin nx}{n}}{x}{0}{\pi}
				+ \frac{1}{n}\myinta{0}{\pi}
				{e^{-x}\sin nx}{x} \right) \\
			& = & \frac{2}{\pi} \left( \evalat
				{-\frac{e^{-x}\cos nx}{n^2}}{x}{0}{\pi}
				- \frac{1}{n^2}\myinta{0}{\pi}
				{e^{-x}\cos nx}{x} \right) \\
			& = & \frac{2(1 - (-1)^n e^{-\pi})}{\pi n^2}
				- \frac{2}{\pi n^2}\myinta{0}{\pi}
				{e^{-x}\cos nx}{x} 
		\end{IEEEeqnarray*}
	Rearranging this equation we get
		\begin{IEEEeqnarray*}{rCl}
			\frac{2}{\pi} \left(1 + \frac{1}{n^2}\right)
				\myinta{0}{\pi}{e^{-x}\cos nx}{x}
				& = & \frac{2(1 - (-1)^n e^{-\pi})}{\pi n^2}
		\end{IEEEeqnarray*}
	and so
		\begin{IEEEeqnarray*}{rCl}
			a_n & = & \frac{2}{\pi}\myinta{0}{\pi}
				{e^{-x}\cos nx}{x} \\
			& = & \frac{2 n^2 (1 - (-1)^n e^{-\pi})}{\pi n^2(n^2+1)} \\
			& = & \frac{2(1 - (-1)^n e^{-\pi})}{\pi(n^2+1)}
		\end{IEEEeqnarray*}
	Therefore because of continuity and Theorem \ref{thrm:Jordan},
	we have
		\begin{displaymath}
			e^{-\modulus{x}} = \frac{(1-e^{-\pi})}{\pi}
				+ 2\sum_{k=1}^\infty \left(
				\frac{1 - (-1)^k e^{-\pi}}{\pi(k^2+1)} \right)
				\cos kx
		\end{displaymath}
	for $-\pi \leq x \leq \pi$. In fact, this series converges
	uniformly because
		\begin{displaymath}
			\modulus{\left( 2\frac{(-1)^k e^{-\pi}-1}{\pi(k^2+1)} \right)
				\cos kx} \leq \frac{4}{\pi k^2}
		\end{displaymath}
	and the series
		\begin{displaymath}
			\sum_{k=1}^\infty \frac{4}{\pi k^2}
		\end{displaymath}
	converges.
\end{soln}
	
%%%%%%%%%%%%%%%%%%%%%%%%%%%%%%%%%%%%%%%%%%%%%%%%%%
%% Riemann Localization Theorem
%%%%%%%%%%%%%%%%%%%%%%%%%%%%%%%%%%%%%%%%%%%%%%%%%%

\begin{thrm}[Riemann's Localization Theorem]
	Suppose $f$ and $g$ are $2\pi$-periodic Riemann
	Integrable functions which agree on $(x_0-\delta,
	x_0+\delta)$ for some $x_0 \in \real$ and $\delta
	> 0$. Let $s_n$ be the $n$th partial sum of the
	Fourier series of $f$ and $t_n$ the $n$th partial
	sum of the Fourier series of $g$. If $s_n(x_0)$
	converges to some $l$ then so does $t_n(x_0)$.
\end{thrm}

\begin{proof}
	Let $h = f - g$ and $u_n = s_n - t_n$. Then $u_n$
	is the $n$th partial sum of the Fourier series of
	$h$ and we have
		\begin{IEEEeqnarray*}{rCl}
			u_n(x_0) & = & \frac{1}{2\pi}\myintb[\pi]
				{h(x_0+t)D_n(t)}{t} \\
			& = & \frac{1}{2\pi}\myinta{-\pi}{-\delta}
				{h(x_0+t)D_n(t)}{t} + \frac{1}{2\pi}
				\myintb[\delta]{h(x_0+t)D_n(t)}{t} \\
			& & + \; \frac{1}{2\pi}\myinta{\delta}{\pi}
				{h(x_0+t)D_n(t)}{t} \\
			& = & \frac{1}{2\pi}\myinta{-\pi}{-\delta}
				{h(x_0+t)D_n(t)}{t} + \frac{1}{2\pi}
				\myinta{\delta}{\pi}{h(x_0+t)D_n(t)}{t}
		\end{IEEEeqnarray*}
	since $h(x) = 0$ on $(x_0-\delta,x_0+\delta)$. Now
		\begin{IEEEeqnarray*}{l}
			\frac{1}{2\pi}\myinta{-\pi}{-\delta}
				{h(x_0+t)D_n(t)}{t} + \frac{1}{2\pi}
				\myinta{\delta}{\pi}{h(x_0+t)D_n(t)}{t} \\
			= \frac{1}{2\pi}\myinta{-\pi}{-\delta}
				{\frac{h(x_0+t)}{\sin t/2}\sin (n+1/2)t}{t} \\
			+ \; \frac{1}{2\pi}\myinta{\delta}{\pi}
				{\frac{h(x_0+t)}{\sin t/2}\sin (n+1/2)t}{t}
		\end{IEEEeqnarray*}
	$\rightarrow 0$ as $n \rightarrow \infty$ by Theorem
	\ref{thrm:RLL}.
\end{proof}

\end{section}
