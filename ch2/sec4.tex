\begin{section}{Dirichlet's Kernel}

	Suppose $a_0, a_1, b_1, \ldots$ are the
	trigonometric Fourier coefficients of a
	given function $f$ and $\ldots, c_{-1},
	c_0, c_1, \ldots$ its exponential Fourier
	coefficients. The \emph{Fourier Series of
	f} is the series with partial sums
		\begin{IEEEeqnarray*}{rCl}
			s_n(x) & = & \frac{a_0}{2} + 
				\sum_{k=1}^n a_k\cos kx + b_k\sin kx \\
			& = & c_0 + \sum_{k=1}^n (c_k e^{ikx}
				+ c_{-k} e^{-ikx})
		\end{IEEEeqnarray*}
	In this section, we will derive some properties
	of this series.
	
%%%%%%%%%%%%%%%%%%%%%%%%%%%%%%%%%%%%%%%%%%%%%%%%%%%
%% Derivation of Dirichlet Kernel
%%%%%%%%%%%%%%%%%%%%%%%%%%%%%%%%%%%%%%%%%%%%%%%%%%%

	Let $s_n$ be the $n$th partial sum of the Fourier
	series of $f$. Then
		\begin{IEEEeqnarray*}{rCl}
			s_n(x) & = & \sum_{k=-n}^n c_k e^{ikx} \\
			& = & \sum_{k=-n}^n \left( \frac{1}{2\pi}
				\myintb[\pi]{f(t)e^{-ikt}}{t} \right)
				e^{ikx} \\
			& = & \frac{1}{2\pi} \myintb[\pi]{ \left(
				\sum_{k=-n}^n f(t)e^{-ikt}e^{ikx} \right)}{t} \\
			& = & \frac{1}{2\pi} \myintb[\pi]{\left( f(t)
				\sum_{k=-n}^n e^{ik(x-t)} \right)}{t} \\
			& = & \frac{1}{2\pi} \myintb[\pi]{f(t) D_n(x-t)}{t}
		\end{IEEEeqnarray*}
	where $D_n(u) = \sum_{k=-n}^n e^{iku}$ for all $u \in \real$.
	We call $\{D_n\}_{n=0}^\infty$ the \emph{Dirichlet Kernel}.
		
%%%%%%%%%%%%%%%%%%%%%%%%%%%%%%%%%%%%%%%%%%%%%%%%%%%
%% Properties of Dirichlet Kernel
%%%%%%%%%%%%%%%%%%%%%%%%%%%%%%%%%%%%%%%%%%%%%%%%%%%
	
\begin{prop}\label{prop:Dirch}
	For each $n \in \intgr$, $n \geq 0$, we have
		\begin{displaymath}
			D_n(x) =
				\begin{cases}
					2n + 1 & \text{if } x = 2\pi l, l \in \intgr \\
					\displaystyle{
						\frac{\sin(n+1/2)x}{\sin x/2}}
						& \text{otherwise}
				\end{cases}
		\end{displaymath}
\end{prop}

\begin{proof}
	Clearly, if $x = 2 \pi l$ for $l \in \intgr$ then
	$e^{ikx} = 1$ so $\sum_{k=-n}^n e^{ikx} = 2n + 1$.
	Otherwise, we have
		\begin{IEEEeqnarray*}{rCl}
			D_n(x) & = & \sum_{k=-n}^n e^{ikx} \\
			& = & \sum_{k=-n}^n (e^{ix})^k \\
			& = & e^{-inx} \sum_{k=0}^{2n} (e^{ix})^k \\
			& = & e^{-inx} \left( \frac{1 - (e^{ix})^{2n+1}}
				{1 - e^{ix}} \right) \\
			& = & \frac{e^{-inx}}{e^{ix/2}} \left( \frac
				{1 - (e^{ix})^{2n+1}}{e^{-ix/2} - 
				e^{ix/2}} \right) \\
			& = & e^{-i(n+1/2)x} \left( \frac{1 - (e^{ix})^{2n+1}}
				{-2i\sin x/2} \right) \\
			& = & \frac{e^{-i(n+1/2)x} - e^{i(n+1/2)x}}
				{-2i\sin x/2} \\
			& = & \frac{-2i \sin (n+1/2)x}{-2i \sin x/2} \\
			& = & \frac{\sin (n+1/2)x}{\sin x/2}
		\end{IEEEeqnarray*}
\end{proof}

\begin{thrm}\label{thrm:Dirch}
	For all $n \in \natrl$,
		\begin{enumerate}[i)]
			\item
				$D_n$ is real-valued, $\ctsdiff$, $2\pi$-periodic,
				even and satisfies $\modulus{D_n(x)} \leq 2n+1$.
			\item
				$\myintb[\pi]{D_n(t)}{t} = 2\pi$
			\item
				For all $\delta \in (0,\pi)$,
					\begin{displaymath}
						\lim_{n \rightarrow \infty}
							\myinta{\delta}{\pi}{D_n(t)}{t}
							= \myinta{-\pi}{-\delta}{D_n(t)}{t}
							= 0
					\end{displaymath}
		\end{enumerate}
\end{thrm}

\begin{proof}
	\begin{enumerate}[i)]
		\item
			That $D_n$ is real-valued, $\ctsdiff$, $2\pi$-periodic, even
			and satisfies $\modulus{D_n(x)} \leq 2n+1$ follows
			from its definition.
		\item
			\begin{IEEEeqnarray*}{rCl}
				\myintb[\pi]{D_n(t)}{t} & = &
					\myintb[\pi]{\sum_{k=-n}^n e^{ikt}}{t} \\
				& = & \myintb[\pi]{1}{t} + \sum_{\substack{
					k=-n \\ k \neq 0}}^n \myintb[\pi]{e^{ikt}}{t} \\
				& = & 2\pi
			\end{IEEEeqnarray*}
		\item
			If $0 < \delta < \pi$ then
				\begin{displaymath}
					\myinta{\delta}{\pi}{D_n(t)}{t}
						= \myinta{\delta}{\pi}{g(t)
						\sin (n+1/2)t}{t}
				\end{displaymath}
			where $g(t) = 1 / (\sin t/2)$. Since $g$ is
			continuous on $[\delta, \pi]$, it is Riemann
			Integrable on this interval and hence this
			integral goes to zero as $n \rightarrow \infty$
			by Theorem \ref{thrm:RLL}. The other integral can
			be deduced by noting that $D_n$ is even.
	\end{enumerate}
\end{proof}

%% More properties
\begin{prop}
	Let $f$ be a $2\pi$-periodic, Riemann Integrable function
	and $s_n$ the $n$th partial sum of its Fourier series.
	Then
		\begin{enumerate}[i)]
			\item
				\begin{displaymath}
					s_n(x) = \frac{1}{2\pi}\myintb[\pi]
						{f(x-t) D_n(t)}{t}
				\end{displaymath}
			\item
				\begin{displaymath}
					s_n(x) = \frac{1}{2\pi}\myintb[\pi]
						{f(x+t) D_n(t)}{t}
				\end{displaymath}
		\end{enumerate}
\end{prop}

\begin{proof}
	\begin{enumerate}[i)]
		\item
			\begin{IEEEeqnarray*}{rCl}
				s_n(x) & = & \frac{1}{2\pi} \myintb[\pi]
					{f(t) D_n(x-t)}{t} \\
				& = & - \frac{1}{2\pi} \myinta{x+\pi}
					{x-\pi}{f(x-s) D_n(s)}{s}
					\text{ where } s = x - t \\
				& = & \frac{1}{2\pi} \myinta{x-\pi}
					{x+\pi}{f(x-s) D_n(s)}{s} \\
				& = & \frac{1}{2\pi} \myintb[\pi]
					{f(x-s) D_n(s)}{s}
			\end{IEEEeqnarray*}
		by Proposition \ref{prop:periodic} and the fact
		that $f$ is $2\pi$-periodic.
		\item
			By part i) we have
				\begin{IEEEeqnarray*}{rCl}
					s_n(x) & = & \frac{1}{2\pi} \myintb[\pi]
						{f(x-t) D_n(t)}{t} \\
					& = & - \frac{1}{2\pi} \myinta{\pi}{-\pi}
						{f(x+s) D_n(-s)}{s} \text{ where } s = -t \\
					& = & \frac{1}{2\pi} \myintb[\pi]
						{f(x+s) D_n(s)}{s} \text{ since } D_n
						\text{ is even}
				\end{IEEEeqnarray*}
	\end{enumerate}
\end{proof}

\end{section}
