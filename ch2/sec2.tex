\begin{section}{Maximum Minimum Priniple}

Roughly stated, the Maximum Minimum Principle
says the maximum (minimum) value of a function
on a bounded region, under certain conditions,
occurs along the boundary of the region. This
result will be used to prove the uniqueness of
the solution to the heat problem. But we begin
with a lemma that will be used to prove the
principle.

%%%%%%%%%%%%%%%%%%%%%%%%%%%%%%%%%%%%%%%%%%%%%%%%%%
%% lemma used to prove MM Principle
%%%%%%%%%%%%%%%%%%%%%%%%%%%%%%%%%%%%%%%%%%%%%%%%%%

\begin{lemma}\label{lemma:maxmin}
	Suppose $v:[a,b] \times [0,\infty) \to \real$ 
	is continuous on its domain and $\ctsdiff[2]$
	on $(a,b) \times (0,\infty)$. Futhermore, suppose
		\begin{displaymath}
			\frac{\partial v}{\partial t}(x,t)
				- \frac{\partial^2 v}{\partial x^2}(x,t)
				< 0
		\end{displaymath}
	for $(x,t) \in (a,b) \times (0,\infty)$.
	
	Fix $T > 0$ and define
		\begin{displaymath}
			B_T = \{(a,t):0 \leq t \leq T\}
				\cup \{(b,t):0 \leq t \leq T\}
				\cup \{(x,0):a \leq x \leq b\}
		\end{displaymath}
		
	Then $\max \{v(x,t):a \leq x \leq b, 0 \leq t \leq T\}
	= \max \{v(x,t):(x,t) \in B_T\}$.
\end{lemma}

\begin{proof}
	Since $[a,b] \times [0,T]$ is compact and $v$ is
	continuous, $v$ attains an absolute maximum on this
	region. Say this absolute maximum occurs at the point
	$(x_0,t_0)$. If $a < x_0 < b$ or $0 < t_0 \leq T$ then
	by the first derivative test we must have
		\begin{displaymath}
			\frac{\partial v}{\partial t}(x_0,t_0)
				= \frac{\partial v}{\partial x}(x_0,t_0)
				= 0
		\end{displaymath}
				
	By the second derivative test we must have
		\begin{displaymath}
			\frac{\partial^2 v}{\partial x^2}(x_0,t_0) < 0
		\end{displaymath}
	
	But then
		\begin{displaymath}
			\frac{\partial v}{\partial t}(x,t)
				- \frac{\partial^2 v}{\partial x^2}(x,t)
				> 0
		\end{displaymath}
	which is a contradiction.
\end{proof}

%%%%%%%%%%%%%%%%%%%%%%%%%%%%%%%%%%%%%%%%%%%%%%%%%%
%% MM Principle
%%%%%%%%%%%%%%%%%%%%%%%%%%%%%%%%%%%%%%%%%%%%%%%%%%

\begin{thrm}[Maximum Minimum Principle]\label{thrm:maxmin}	
	Suppose $u:[a,b] \times [0,\infty) \to \real$ is
	continous on its domain, $\ctsdiff[2]$ on $(a,b) \times
	(0,\infty)$ and satisfies
		\begin{displaymath}
			\frac{\partial u}{\partial t}(x,t)
				= \frac{\partial^2 u}{\partial x^2}(x,t)
		\end{displaymath}
	for $(x,t) \in (a,b) \times (0,\infty)$.
	
	Then for all $T > 0$,
		\begin{enumerate}[i)]
			\item
				\begin{displaymath}
					\max \{u(x,t):a \leq x \leq b, 0 \leq t \leq T\}
						= \max \{u(x,t):(x,t) \in B_T\}
				\end{displaymath}

			\item
				\begin{displaymath}
					\min \{u(x,t):a \leq x \leq b, 0 \leq t \leq T\}
						= \min \{u(x,t):(x,t) \in B_T\}
				\end{displaymath}
		\end{enumerate}		
	where $B_T$ is defined as in Lemma \ref{lemma:maxmin}.
\end{thrm}

\begin{proof}
	\begin{enumerate}[i)]
		\item
			Given $T > 0$, let $M_T = \max \{u(x,t):(x,t) \in B_T\}$.
			Fix $\epsilon > 0$ and define
				\begin{displaymath}
					v(x,t) = u(x,t) + \epsilon x^2
				\end{displaymath}
			for $(x,t) \in [a,b] \times [0,\infty)$. Then $v$ is
			continuous on its domain, $\ctsdiff[2]$ on $(a,b)
			\times (0,\infty)$ and satisfies
				\begin{IEEEeqnarray*}{l}
					\frac{\partial v}{\partial t}(x,t)
						- \frac{\partial^2 v}{\partial x^2}(x,t) \\
					= \frac{\partial u}{\partial t}(x,t)
						- \frac{\partial^2 u}{\partial x^2}(x,t)
						- 2 \epsilon \\
					= -2 \epsilon \\
					< 0
				\end{IEEEeqnarray*}
			for all $(x,t) \in (a,b) \times (0,\infty)$.
	
			By Lemma \ref{lemma:maxmin} $v(x,t) \leq M_T$ when
			$a \leq x \leq b$ and $0 \leq t \leq T$. So
				\begin{displaymath}
					u(x,t) + \epsilon x^2 \leq M_T
				\end{displaymath}
			on this same region. Since this is true for all
			$\epsilon > 0$ and $T > 0$, the result follows.
		
		\item
			Apply part i) with $u$ replaced by $-u$.
	\end{enumerate}
\end{proof}		

%%%%%%%%%%%%%%%%%%%%%%%%%%%%%%%%%%%%%%%%%%%%%%%%%%
%% Corollaries to MM Principle
%%%%%%%%%%%%%%%%%%%%%%%%%%%%%%%%%%%%%%%%%%%%%%%%%%

\begin{defn}
	Let $C_0[0,\pi]$ be defined by the set of all
	continuous, real-valued functions on $[0,\pi]$
	which satisfy $f(0) = f(\pi) = 0$.
\end{defn}

\begin{cor}
	Suppose $f_1, f_2 \in C_0[0,\pi]$ and $u_1, u_2$
	are the solutions to the heat problem corresponding
	to $f_1, f_2$, respectively. Furthermore, suppose
	$f_1(x) \leq f_2(x)$ for all $x \in [0,\pi]$. Then
	$u_1(x,t) \leq u_2(x,t)$ for all $(x,t) \in [0,\pi]
	\times [0,\infty)$.
\end{cor}

\begin{proof}
	Let $f = f_1 - f_2$. Then $u = u_1 - u_2$ is the
	solution to the heat problem corresponding to $f$.
	Now $u$ satisfies the hypotheses of Theorem
	\ref{thrm:maxmin} so $u(x,t) \leq M_T$ for all
	$(x,t) \in [0,\pi] \times [0,\infty)$ (where $M_T$
	is defined as in the proof of Theorem \ref{thrm:maxmin}).
	But $M_T \leq 0$. So the result follows.
\end{proof}

\begin{cor}\label{cor:maxmin2}
	Suppose $f \in C_0[0,\pi]$ and $u$ is the solution to
	the heat problem corresponding to $f$. If $\modulus{
	f(x)} \leq M$ for some $M > 0$ and all $x \in [0,\pi]$
	then $\modulus{u(x,t)} \leq M$ for all $(x,t) \in [0,\pi]
	\times [0,\infty)$.
\end{cor}

\begin{proof}
	Since $-M \leq f(x) \leq M$ for all $x \in [0,\pi]$ the
	value of $M_T$ in the proof of Theorem \ref{thrm:maxmin}
	satisfies $-M \leq M_T \leq M$ as well. Therefore, by
	Theorem \ref{thrm:maxmin}, $-M \leq u(x,t) \leq M$ for
	all $(x,t) \in [0,\pi] \times [0,\infty)$.
\end{proof}

\begin{cor}\label{cor:maxmin3}
	Suppose $f_1, f_2 \in C_0[0,\pi]$, $u_1, u_2$ are the 
	solutions to the heat problem corresponding to $f_1,
	f_2$ respectively and for some $\epsilon > 0$ we have
	$\modulus{f_1(x) - f_2(x)} \leq \epsilon$. Then
	$\modulus{u_1(x,t) - u_2(x,t)} \leq \epsilon$ for all
	$(x,t) \in [0,\pi] \times [0,\infty)$.
\end{cor}

\begin{proof}
	Let $f = f_1 - f_2$ and $u = u_1 - u_2$. Then $u$ is the
	solution to the heat problem corresponding to $f$. So the
	result is an immediate consequence of Corollary \ref{cor:maxmin2}.
\end{proof}

\begin{thrm}[Uniqueness of the Heat Equation]\label{thrm:HeatUnique}
	If $f \in C_0[0,\pi]$ then there is at most one
	solution to the heat problem corresponding to $f$.
\end{thrm}

\begin{proof}
	Suppose $u_1$ and $u_2$ are two solutions of the heat
	problem corresponding to $f$. Apply Corollary \ref{cor:maxmin3}
	with $f_1 = f_2 = f$.
\end{proof}

\end{section}
